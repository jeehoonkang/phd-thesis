\section{Results}
\label{sec:results}

In this section, we outline a number of important results we have proven to hold of our
``promising'' model.
%
% \ori{we say too much ``our model''; should we give it a name?
% The ``promise model''?}
%
All the results of this section \textbf{are fully validated in Coq} except
for % \Cref{sec:compilation_POWER} and
\Cref{thm:drfra,thm:drflock}, for which we provide hand-written proofs in \Cref{sec:relaxed:proofs}.
The {Coq} development is available at~\cite{kang-phd-thesis-web}.

\subsection{Compiler Transformations}
\label{sec:transformations}

A transformation $\prog_\text{src} \leadsto \prog_\text{tgt}$ is \emph{sound} 
if it does not introduce new behaviors under any (parallel and sequential) context, that is, 
for every context ${\cal C}$,
every behavior of ${\cal C}[\prog_\text{tgt}]$ is a behavior of ${\cal C}[\prog_\text{src}]$.

Next, we list the program transformations proven to be sound in our model.
To streamline the presentation, we refer to transformations on the \emph{semantic} level, 
as if they are applied to \emph{actions}, namely fences and (valueless) memory accesses.
Thus, we presuppose adequate syntactic manipulations on the program level
that implement these semantic transformations.
For example, a syntactic transformation implementing \\
${\rlab^x_\rlx;\rlab^y_\rlx \leadsto \rlab^y_\rlx;\rlab^x_\rlx}$
is a reordering ${a:=x; b:=y \leadsto b:=y; a:=x}$
on the program code (assuming $a \neq b$);
while a merge of a write and an update correspond, \eg  to a transformation
of the form $x:=a;\fai{x,1}  \leadsto  x:=a+1$.
Nevertheless, our formal development
proves soundness of transformations on the purely \emph{syntactic} level, assuming a simple programming language with memory operations, conditionals, and loops.


%To this end, we identify the commands with thread events of the form $\rlab_o^x$,
%$\wlab_o^x$, $\ulab_{o_\lr,o_\lw}^x$, $\flab_\relo$, $\flab_\acqo$, or $\flab_\sco$.

\paragraph{Trace-Preserving Transformations}
Transformations that do not change the set of traces of actions in a given thread are clearly sound.
For example, $y:=a+1-a \leadsto y:=1$ is a sound transformation (recall that $a$ denotes a local register; see~\Cref{sec:introduction}:\ref{eq:LBfalsedep}).
Indeed, this is the crucial property that distinguishes a memory model for a higher-level language from
a hardware memory model.

 \paragraph{Strengthening}
A simple transformation that is sound in our model is strengthening of access modes.
A read/write action $\xlab_o$ can be transformed to $\xlab_{o'}$
provided that $o \sqsubseteq o'$. Similarly, 
it is sound to replace  $\ulab_{o_\lr,o_\lw}$ by $\ulab_{o'_\lr,o'_\lw}$ provided that 
$o_\lr \sqsubseteq o'_\lr$ and $o_\lw \sqsubseteq o'_\lw$,
or to strengthen $\flab_\relo$ or $\flab_\acqo$ to $\flab_\sco$.

\paragraph{Reordering}
Next we consider transformations of the form  $\xlab ; \ylab \leadsto \ylab;\xlab$,
and specify the set of \emph{reorderable} pairs, that is
the set of pairs $\xlab;\ylab$ for which we proved this reordering transformation to be sound in our model. 
First, the following pairs  are reorderable (where $x$ and $y$ denote distinct locations):
%\begin{tabbing}
%\hspace{5cm}\=\kill
%\sbul $\wlab^x_{o_1};\rlab^y_{o_2}$ where $x\neq y$, and either $o_1 \neq \sco$ or $o_2 \neq \sco$\\[1mm]
%\sbul  $\wlab^x;\wlab^y_o$, where $x\neq y$ and $o \sqsubseteq \rlx$ \\[1mm]
%\sbul $\rlab^x_{o_1};\rlab^y_{o_2}$, where $o_1 \sqsubseteq \rlx$, and either $x\neq y$ or $o_2=\pln$ \\[1mm]
%\sbul $\rlab^x_{o_1};\wlab^y_{o_2}$ where $x\neq y$, $o_1 \sqsubseteq \rlx$, and $o_2 \sqsubseteq \rlx$ \\[1mm]
%\sbul $\rlab_o;\flab_\acqo$ where $o\neq\rlx$ \> \sbul$\wlab;\flab_\acqo$ \\[1mm]
%\sbul  $\flab_\relo;\wlab_o$ where $o\neq\rlx$  \> \sbul $\flab_\relo;\rlab$  
%\end{tabbing} 
{\begin{tabbing}
\hspace{5cm}\= \hspace{3cm}\=\kill
%\sbul $\wlab^x_{o_1};\rlab^y_{o_2}$ unless $o_1 =o_2= \sco$ \> \sbul  $\wlab^x;\wlab^y_{\sqsubseteq \rlx}$ \\[1mm]
\sbul $\wlab^x;\rlab^y$ \> 
\sbul  $\wlab^x;\wlab^y_{\sqsubseteq \rlx}$  \>
\sbul$\wlab;\flab_\acqo$ \\[1mm]
\sbul $\rlab^x_{\sqsubseteq \rlx};\rlab^y$ and $\rlab^x_{\pln};\rlab^x_{\pln}$ \>
\sbul $\rlab^x_{\sqsubseteq \rlx};\wlab^y_{\sqsubseteq \rlx}$ \>
\sbul $\rlab_{\neq \rlx};\flab_\acqo$ \\[1mm]
\sbul  $\flab_\relo;\wlab_{\neq\rlx}$   \> \sbul $\flab_\relo;\rlab$  \> \sbul $\flab_\relo;\flab_\acqo$
\end{tabbing} }


In addition, for the purpose of specifying reorderable pairs, an update is just a combination of a read and a write.
Thus, $\xlab;\ulab_{o_\lr,o_\lw}$ is reorderable if both $\xlab;\rlab_{o_\lr}$ and $\xlab;\wlab_{o_\lw}$ are reorderable,
and symmetrically  $\ulab_{o_\lr,o_\lw};\xlab$ is reorderable if both $\rlab_{o_\lr};\xlab$ and $\wlab_{o_\lw};\xlab$ are reorderable.
In particular, a pair $\ulab^x_{o^x_\lr,o^x_\lw}; \ulab^y_{o^y_\lr,o^y_\lw}$
is reorderable if $x\neq y$, $o_\lr^x \sqsubseteq \rlx$, $o_\lw^y \sqsubseteq \rlx$.
%, and either $o_\lw^x \neq \sco$ or $o_\lr^y \neq \sco$.

The set of reorderable pairs in our model contains all pairs that are intended to be reorderable in the C/C++ 
and Java memory models,
including in particular all ``roach-motel reorderings''~\cite{c11comp,sevcik:jmm}.

%\jeehoon{\cite{c11comp} presented the reordering result with a table that contains instruction-by-instruction reorderable criteria.  I think it is more readable.  Similarly to other results.  How do you think?}
%\ori{no space, I think, and I actually find this somewhat easier to read...}

\paragraph{Merging}
These are transformations that completely eliminate an action.
Clearly, the two actions in \emph{mergeable} pairs (pairs for which we proved the merge to be sound in our model)
should access the same location. The following three kinds of pairs are mergeable:
%\begin{tabbing}
%\hspace{1.5cm}\=\hspace{3cm}=\=\hspace{1.5cm}=\=\hspace{1.5cm}=\kill
% $\rlab$-after-$\rlab$: \>  $\rlab_o;\rlab_o$ \>  $\wlab$-after-$\wlab$: \>   $\wlab_o;\wlab_o$ \\[1mm]
%  $\rlab$-after-$\wlab$: \> $\wlab;\rlab$ %and  $\wlab_{\sco}; \rlab_{\sco}$
%%  $\rlab$-after-$\wlab$: \> $\wlab;\rlab_\ra$ %and  $\wlab_{\sco}; \rlab_{\sco}$
%\end{tabbing} 
\begin{tabbing}
\hspace{1.4cm}\=\hspace{1.5cm}=\=\hspace{1.4cm}\=\hspace{1.5cm}=\=\hspace{1.4cm}=\=\hspace{1.5cm}=\kill
 $\rlab$-after-$\rlab$: \>  $\rlab_o;\rlab_o$ \>  $\wlab$-after-$\wlab$: \>   $\wlab_o;\wlab_o$ \>  $\rlab$-after-$\wlab$: \> $\wlab;\rlab$
\end{tabbing} 

Using the strengthening transformation, the access modes here can be read as upper bounds
(\eg $\rlab_\ra;\rlab_\rlx$ can be first strengthened to $\rlab_\ra;\rlab_\ra$ and then merged).
Note that the  $\rlab$-after-$\wlab$ merge allows 
even to eliminate a redundant acquire read after a plain/relaxed write (as in Example \ref{eq:acq_merge} in \Cref{sec:relacq}).
%Nevertheless, an SC-read cannot be eliminated in this case, unless it follows an SC-write.
%Indeed, elimination of a an SC-read after a non-SC-write is unsound in our model.
%We note that while this elimination is allowed by a certain fix of of C++ described in \cite{c11comp},
%its effectiveness seems to be low, and, in fact,  
%it is already unsound in the strengthening of the SC semantics in C++ that was proposed in~\cite{Batty:2016}

In addition, the following pairs involving updates are mergeable:
\begin{tabbing}
\hspace{1.5cm}\=\hspace{4.4cm}\=\hspace{1.5cm}\=\hspace{1.5cm}\kill
 $\rlab$-after-$\ulab$: \> $\ulab_{\rlx, o}; \rlab_{\rlx}$, and $\ulab_{\ra, o}; \rlab_{\ra}$ \> %, and $\ulab_{\sco,\sco}; \rlab_{\sco}$  \\[1mm]
  $\ulab$-after-$\wlab$: \>   $\wlab_o; \ulab_{o_\lr, o}$ \\[1mm] %, provided that $\wlab_o; \rlab_{o_\lr}$ is mergeable 
 $\ulab$-after-$\ulab$: \>   $\ulab_{o_1, o}; \ulab_{o_2, o}$, provided that $\ulab_{o_1, o}; \rlab_{o_2}$ is mergeable 
\end{tabbing} 
Note that read-after-update does not allow the read to be an acquire read unless the update includes an acquire read
(unlike read-after-write elimination). This is due to release sequences:
 eliminating an acquire read after a relaxed update may remove the synchronization 
due to a release sequence ending in this update.
 
Finally, two fences of the same type can obviously be merged.

%With the exception of ``SC-read after non-SC-write'' mentioned above, 
The set of mergeable pairs in our model contains all pairs intended to be mergeable in the C/C++ and Java models~\cite{c11comp,sevcik:jmm}.
In particular, we support $\rlab$-after-$\wlab$ merging, which is the effect of local satisfaction of reads
in hardware like TSO, Power, and ARM.

\paragraph{Introducing and Eliminating Unused Reads}
Introduction of irrelevant read accesses is sound in our model, unlike in the Java memory model~\cite{sevcik:jmm}.
Eliminating plain read accesses whose read values are never used in the program is also sound in our model.
In contrast, eliminating relaxed or acquire reads is not generally sound because it may remove synchronization.

%\ori{why not relaxed?}
%\jeehoon{\url{https://github.com/jeehoonkang/memory-model-explorer/blob/operational/formalization/src/opt/UnusedLoad.v\#L32}}

\paragraph{Proof Technique}

Our proof of these results employs the well-known approach of simulation relations between the
target and the source programs.  We prove the adequacy of simulation up-to context, or in other
words, we can prove simulations between code fragments and compose them.  Our definitions are fairly
standard, except that they ensure thread-locality, thus allowing us to define a simulation relation
on thread configurations, which (as we prove) can be composed into a simulation relation on full
machine states.  The thread-locality is crucially used in proving the soundness of transformations
because they are performed not for a full machine but for a single thread.  Our thread-local
simulation relation is based on novel ideas for reasoning about the shared memory and consistency.
For more details, we refer the reader to \Cref{sec:relaxed:proofs:simulation}.

% it is difficult to employ the standard simulation relations presented in
% \Cref{sec:background:correctness} for proving compiler optimizations because optimizations are
% usually performed for each thread, while simulation relations are defined over full machine states
% in which the other threads may affect the shared resources (\ie{} the memory and the global SC
% timemap) at any time.  To address this difficulty, we define thread-local simulations on thread
% configurations, and prove that they can be composed into a simulation relation on full machine
% states.


% Additionally, for thread configurations, our relations let us ignore the certification processes and
% just focus on thread steps in the source and the target.



\subsection{Compilation to TSO}
\label{sec:compilation_TSO}

Like C18/C++17, our model can be efficiently compiled to x86-TSO.
Since this architecture provides relatively strong guarantees,
every memory access can be compiled to a primitive hardware instruction.
Moreover, release/acquire fences are ignored during compilation,
and SC fences are mapped to an $\mfence$ instruction.
Correctness of this mapping follows from a recent result by Lahav and Vafeiadis~\cite{fm16},
which shows that all weak behaviors of TSO are explained by store-load reordering and merging.
Accordingly, it reduces the correctness proof of compilation to TSO to:
$(i)$ supporting write-read reordering and write-read merge;
and $(ii)$ a correctness proof of compilation to SC.
Since we proved the soundness of write-read reordering and merge 
(regardless of the access modes of the two events),
and since clearly our model is weaker than SC,
we immediately derive the correctness of compilation to TSO.
% \derek{What does it mean to be not limited by access modes?}
% \jeehoon{Meaning a write and a read is reorderable regardless of their access modes.  I think it would better to just remove the phrase.}
% \ori{I changed to regardless of their access modes}

% \subsection{Compilation to Power}
% \label{sec:compilation_POWER}

% \begin{figure}[t]
% \centering
% \small
% $\begin{array}{@{}l@{}c@{{}={}}l@{\quad}l@{}c@{{}={}}l@{}}
% \compile{\rlab_{\sqsubseteq\rlx}}&&\texttt{ld}&
% \compile{\rlab_\ra} && \texttt{ld; lwsync} \\
% \compile{\wlab_{\sqsubseteq\rlx}} && \texttt{st} &
% \compile{\wlab_\ra} && \texttt{lwsync; st} \\
% \compile{\flab_{\sqsubset \sco}}  && \texttt{lwsync} & 
% \compile{\flab_\sco} && \texttt{sync} \\
% \multicolumn{6}{@{}l@{}}{
% \compile{\ulab_{{\sqsubseteq\rlx},{\sqsubseteq\rlx}}} = \texttt{L: lwarx; cmp; bc  Lout; stwcx.; bc  L;  Lout:}}\\ 
% \compile{\ulab_{\ra,{\sqsubseteq\rlx}}} && \compile{\ulab_{\rlx,\rlx}} \texttt{; lwsync} &
% \compile{\ulab_{{\sqsubseteq\rlx},\ra}} &&  \texttt{lwsync; }\compile{\ulab_{\rlx,\rlx}} \\ 
% \compile{\ulab_{\ra,\ra}} &&  \multicolumn{4}{@{}l@{}}{\texttt{lwsync; } \compile{\ulab_{\rlx,\rlx}}\texttt{; lwsync}}
% \end{array}$
% %\begin{tabular}{@{}l@{~~~~~}l@{}}
% %$\rlab_{\sqsubseteq\rlx}$ & \texttt{ld} \\ 
% %$\wlab_{\sqsubseteq\rlx}$ & \texttt{st} \\ 
% %$\rlab_\ra$ & \texttt{ld; lwsync}\\ 
% %$\wlab_\ra$ & \texttt{lwsync; st} \\ 
% %$\ulab_{{\sqsubseteq\rlx},{\sqsubseteq\rlx}}$ & \texttt{\_loop: lwarx; cmp; bc  \_exit;}\\ 
% %& \texttt{stwcx.; bc  \_loop;  \_exit:} \\ 
% %$\ulab_{\ra,{\sqsubseteq\rlx}}$ & $\compile{\ulab_{\rlx,\rlx}}$\texttt{; lwsync} \\ 
% %$\ulab_{{\sqsubseteq\rlx},\ra}$ &  \texttt{lwsync; }$\compile{\ulab_{\rlx,\rlx}}$ \\ 
% %$\ulab_{\ra,\ra}$ &  \texttt{lwsync; }$\compile{\ulab_{\rlx,\rlx}}$\texttt{; lwsync} \\ 
% %%$\ulab^\rel$ & \texttt{lwsync;} $\compile{\ulab^\rlx}$ \\
% %%$\ulab^\acqrel$ & \texttt{lwsync;} $\compile{\ulab^\rlx}$\texttt{; isync} \\ 
% %$\flab_{\sqsubset \sco}$  & \texttt{lwsync} \\ 
% %$\flab_\sco$ & \texttt{sync} \\
% %\end{tabular} 
% %% \vskip -2mm
% \caption{Compilation to Power.}
% \label{fig:comp}
% \end{figure}

% \Cref{fig:comp} provides the compilation scheme of our model to Power, following the C++11 one.
% We denote by $\compile{\prog}$ the Power program obtained by applying this mapping to a source program $\prog$.
% To prove compilation correctness, we again utilize a result of \cite{fm16},
% which shows that every allowed behavior of a Power program $\progb$  
% (assuming its recent declarative model of Alglave \etal~\cite{herding-cats})
% is a behavior of a Power program $\progb'$ under a stronger model, called ``StrongPower'',
% where $\progb'$ is obtained from $\progb$ by applying a sequence of local reorderings of independent memory accesses to distinct locations.
% Accordingly, it suffices to show that, given a source program $\prog$, our model allows all behaviors 
% that StrongPower allows for some reordering of $\compile{\prog}$.
% We split this into two steps, outlined below.

% \paragraph{Compilation to StrongPower}
% First, we show that the compilation to StrongPower is correct, that is: every behavior of $\compile{\prog}$ under StrongPower is allowed for $\prog$ in our model.
% StrongPower strengthens the Power model by forbidding ``load buffering'' behaviors
% (formally, it disallows cycles in the entire program order together with the reads-from relation).
% Consequently, we do not actually need promises in order to explain
% StrongPower behaviors---instead, we can just show that behaviors of $\compile{\prog}$
% under StrongPower are allowed for $\prog$ under our \emph{promise-free} machine (see end of \Cref{sec:full-formal}).
% % This allows us to actually prove that behaviors of StrongPower are allowed
% % even without promises.
% (Note that this does not contradict the fact that
% promises are necessary to explain weak behaviors of the (non-strong) Power model, such as the \ref{eq:LB}.) 
% To ease the proof, we use an alternative, declarative presentation of our promise-free machine
% (\Cref{sec:axiomatic}),
% which can straightforwardly be shown to be weaker than
% the StrongPower model under the compilation scheme in \Cref{fig:comp}.
% See \cite{appendix} for details.

% %(that places a lightweight fence after every acquire or SC read).
% %We believe that this axiomatic semantics is also sufficient for proving the correctness of the more efficient compilation scheme
% %(that uses a control dependency and an isync fence), but we leave this to future work.

% \paragraph{Reorderings in the Compiled Program}
% Second, to account for sequences of reorderings of independent memory accesses to distinct locations in  $\compile{\prog}$,
% we would like to relate them to the reorderings in the source program $\prog$
% that we proved sound in \Cref{sec:transformations}.
% But there is a subtle complication here: some reorderings in $\compile{\prog}$ do not correspond 
% to reorderings in $\prog$! For example, if $\prog$ is $x:=_\ra 1;y:=_\rlx 2$, its compilation $\compile{\prog}$
% has the form $\texttt{lwsync;st}\ x\ 1\texttt{;st}\ y\ 2$, but reordering the stores in $\compile{\prog}$ would produce $\texttt{lwsync;st}\ y\ 2\texttt{;st}\ x\ 1$, which
% does not correspond to the reordering of $\prog$ to $y:=_\rlx 2;x:=_\ra 1$ (compiling the latter would yield $\texttt{st}\ y\ 2\texttt{;lwsync;st}\ x\ 1$).

% To solve this issue, we slightly extend our model and consider compilation as if it happens in two stages.
% First, all release/acquire accesses in $\prog$ are split into weaker accesses and corresponding fences as follows:
% \squishlist
% \item $\rlab_\ra \leadsto \rlab_\rlx ; \flab_\lacq$  and $\wlab_\ra \leadsto \flab_\lrel ; \wlab_\srlx$ 
% \item $\ulab_{\ra,o} \leadsto \ulab_{\rlx,o} ; \flab_\lacq$ and $\ulab_{o,\ra} \leadsto \flab_\lrel ; \ulab_{o,\srlx}$ where $o \sqsubseteq\rlx$
% \squishend
% Here, we introduced a new $\srlx$ (``strong relaxed'') mode, which has the same semantics as $\rlx$
% but blocks promises like $\ra$ writes (\ie in write/update steps, we require
% that  ${\forall m' \in \lprom(x).~ m'.\lmrel = \bot}$ if $o \sqsupseteq \srlx$).
% The reason for this is technical: we would have liked to just use $\rlx$ rather
% than $\srlx$, but at least for $\ulab_{o,\ra}$, the mapping to $\flab_\lrel ; \ulab_{o,\rlx}$ is unsound.  Using $\srlx$, however, we have proved the soundness of the above source-to-source mappings (in Coq) using a straightforward 
% simulation argument (the target program's promises are subject to the same constraints as the source's), and $\srlx$ is sufficient for the rest of the proof.
% % \footnote{We believe that $\wlab_\ra$ can be mapped to $\flab_\lrel ; \wlab_\rlx$ (\ie using $\wlab_\rlx$ rather than $\wlab_\srlx$), but we have not formally established the soundness of this mapping.  However, for $\ulab_{o,\ra}$, the mapping to $\flab_\lrel ; \ulab_{o,\rlx}$ is actually unsound.}
% %\TODO{show the counter example?}


% After this first step, we obtain a program $\prog_1$, which has no
% $\ra$ accesses except possibly for $\ulab_{\ra,\ra}$'s (which are
% surrounded by fences after compilation), and which has $\srlx$
% accesses only immediately after release fences. (The relevance of this
% property will become clearer
% below.)  % \derek{Why is everything after ``except possibly'' in
%   % the above sentence important?}  \jeehoon{Because (1) the split
%   % transformation shown above does not cover the case for
%   % $\ulab_{\ra,\ra}$, and (2) we proved the reordering of strong
%   % relaxed writes past other events, \emph{not} the other way around.
%   % Hence it is important to guarantee that strong relaxed writes occur
%   % only after a release fence, whose mapping to Power is clearly not
%   % reorderable.} 
% For instance, in the example given above, $\prog_1$
% would be $\fencerel;x:=_\srlx 1;y:=_\rlx 2$.  We then apply the
% compilation scheme of \Cref{fig:comp}, where $\srlx$ is compiled like
% $\rlx$.  Clearly, by construction of $\prog_1$, the result
% $\compile{\prog_1}$ is identical to $\compile{\prog}$.  Moreover,
% sequences of reorderings of accesses applied to $\compile{\prog}$ now
% correspond directly to sequences of reorderings in $\prog_1$.

% Thus, suppose Power program $\progb$ is the result of applying some
% sequence of reorderings of accesses to the compilation result
% $\compile{\prog} = \compile{\prog_1}$.  Then, there exists a source
% program $\prog_2$ such that $\compile{\prog_2}=\progb$, and $\prog_2$
% is obtained from $\prog_1$ by applying a sequence of reorderings at
% the source level.  Crucially, the reorderings that take $\prog_1$ to
% $\prog_2$ are all sound in our model: in addition to those already
% covered in \Cref{sec:transformations}, the reorderings of
% $\wlab^x_\srlx/\ulab^x_{\rlx,\srlx}$ past a
% $\rlab_\rlx^y/\wlab_\rlx^y/\ulab_{\rlx,\rlx}^y$ (where $x\neq y$) have
% also been proven sound in Coq.  (Note that the reverse
% reorderings---moving accesses \emph{past} a $\srlx$---are not needed
% because of the aforementioned property that all $\srlx$'s in $\prog_1$
% come immediately after a release fence.)  Returning to the above
% example, when matching the reordering of
% $\texttt{lwsync;st}\ x\ 1\texttt{;st}\ y\ 2$ to
% $\texttt{lwsync;st}\ y\ 2\texttt{;st}\ x\ 1$ in the target, these new
% reorderings validate the transformation of
% $\fencerel;x:=_\srlx 1;y:=_\rlx 2$ to
% $\fencerel;y:=_\rlx 2;x:=_\srlx 1$ in the source.

% Putting it all together: by the compilation to StrongPower result, we
% know that any behavior of $\progb$ under StrongPower is also a
% behavior of $\prog_2$ in our model; by soundness of the reorderings,
% we know this is also a behavior of $\prog_1$; and by soundness of the
% source-to-source mappings, it is also a behavior of the original
% $\prog$.

% % \paragraph{Putting It Together} Suppose $\prog_1$ is the result of
% % translating release/acquire accesses in $\prog$ into fences and
% % relaxed accesses (the $\leadsto$ relation defined above), and suppose


% % transforming for any reordering $\progb$ of
% % $\compile{\prog}$, there is a reordering $\prog_2$ of
% % $\prog_1$ at the source level.  every behavior allowed by StrongPower
% % for a reordering of
% % $\compile{\prog}$ is allowed for a reordering of
% % $\prog_1$ by our model, which, by soundness of the reorderings and the
% % source-to-source mappings above, implies that it is allowed for
% % $\prog$.

% % Putting all pieces together, 
% % every behavior allowed by StrongPower for a reordering of $\compile{\prog}$ is allowed for a reordering of $\prog_1$ by our model,
% % which, by soundness of the reorderings and the  source-to-source mappings above, implies that it is allowed for $\prog$.

% Finally, we note that for C18/C++17, there is a more efficient compilation scheme of acquire reads and updates, which uses a control dependency and an \texttt{isync} fence instead of a lightweight fence (\texttt{lwsync}).
% We believe that our model is a<lso correctly compiled using this scheme.
% Nevertheless, this will require a more direct proof (\texttt{isync} fences are beyond the reach of~\cite{fm16}),
% which we leave to future work.


\subsection{DRF Theorems}
\label{sec:drf}

We proceed with an explanation of our DRF theorems.
These theorems provide ways of restricting attention to better-behaved subsets
of the model assuming certain conditions on programs.

Evidently, the most complicated part of our semantics is the promises.
Without promises, our model amounts to a usual operational model,
where thread steps only arise because of program instructions.
Hence, our first DRF result (and the one that is by far the most challenging to prove)
identifies a set of programs for which promises cannot introduce additional behaviors.
Specifically, we show that this holds for programs in which all
racy accesses are release/acquire, % or SC, 
assuming a promise-free semantics.
Crucially, as usual in DRF guarantees, the races are considered under the stronger
semantics (promise-free), not the full model, thus allowing programmers to adhere to this programming discipline
while being completely ignorant of the weak semantics (promises).

More precisely, we say that a machine state $\mconf$ is \emph{$o$-race-free},
if whenever two different threads may take a (non-promise) step accessing the same location,
then both accesses are reads or both have access mode strictly stronger than $o$.

\begin{theorem}[Promise-Free DRF]
\label{thm:drfpf}
Let $\bstep{}$ denote the steps of the promise-free machine 
(see end of \ref{sec:full-formal}).
Suppose that every machine state that is $\bstep{}$-reachable from the initial state of a program $\prog$ is $\rlx$-race-free.
Then, the behaviors of $\prog$ according to the full machine 
coincide with those according to the $\bstep{}$-machine.
%Let $\mconf=\tup{\gts,\gsco,\mem}$ be a promise-free machine state (that is, $\gts(i).\lprmem=\emptyset$ for all $i\in\Tid$),
%and suppose that any state $\mconf'$ such that $\mconf \bstep{}^* \mconf'$ is $\rlx$-race-free.
%Then, starting from $\mconf$, the usual step $\astep{}$ and the promise-free step $\bstep{}$ admit the same behaviors.
\end{theorem}

Putting promises aside, a counter-intuitive part of weak memory models are the relaxed accesses,
which allow threads to observe writes without observing previous writes to other locations.
Removing $\pln/\rlx$ accesses, namely keeping only $\ra$, % and $\sco$, 
substantially simplifies our machine (in particular, 
its thread views would consist of just one view, the $\lcur$ one).
Accordingly, our second DRF result strengthens \Cref{thm:drfpf}
and states that it suffices to show
that there are only races on $\ra$ %or $\sco$-
accesses \emph{under release/acquire semantics} 
to conclude that a program has only release/acquire behaviors.

\begin{theorem}[DRF-RA]
\label{thm:drfra}
Let $\bstep{\ra}$ be identical to $\bstep{}$ in \Cref{thm:drfpf}, except for interpreting 
$\rlx$ and $\pln$ accesses in program transitions as if they are all $\ra$-accesses.
% (that is, \textsc{read-helper} are \textsc{write-helper} are invoked with $\max\{o,\ra\}$ instead of $o$).
Suppose that every machine state that is $\bstep{\ra}$-reachable from the initial state of a program $\prog$ is $\rlx$-race-free.
Then, the behaviors of $\prog$ according to the full machine 
coincide with those according to the $\bstep{\ra}$-machine.
\end{theorem}

To state a more standard DRF theorem, 
we assume programs are \emph{well-locked}:
(1) locations are partitioned into \emph{normal} and \emph{lock} locations, and
(2) lock locations are accessed only by matching pairs of the following lock/unlock operations:
%$$\inparaII{
%lock(l)\ \{ \\
%\quad \while{!\cas{l}{0}{1}{\acqr\relw}}{\skipc;} \\
%\}
%}{
%unlock(l)\ \{ \\
%\quad l_{\relw} := 0; \\
%\}
%}
%$$
\begin{align*}
lock(l) & : \quad \while{!\cas{l}{0}{1}{\acqr\relw}}{\skipc;} \\
unlock(l) & : \quad l_{\relw} := 0; 
\end{align*}

The theorem forbids any weak behavior in programs that, under SC semantics, race only on lock locations.
For the SC semantics, we consider ``an interleaving machine'', 
where reads read from the latest write to the appropriate location
(regardless of the access modes).

\begin{theorem}[DRF-LOCK]
\label{thm:drflock}
Let $\bstep{\sco}$ denote the steps of the interleaving machine.
Suppose that every machine state that is $\bstep{\sco}$-reachable from
the initial state of a well-locked program $\prog$ is race-free on normal locations.
Then, the behaviors of $\prog$ according to the full machine coincide with those
according to the $\bstep{\sco}$-machine.
\end{theorem}


\paragraph{Necessity of Re-certification}

We observe that promise re-certification is necessary for \Cref{thm:drfpf,thm:drfra} to hold, as
shown in the following example:
\[
\inarrIII{ w_{\relw}:=1; }
         { a:=y_{\acqr}; \\
           \kw{if}~(a == 1)~\{ \\
           ~~b:=z; \\
           ~~\kw{if}~(b == 1)~\{ \\
           ~~~~x:=1; \\
           ~~\} \\
           \}
         }
         { a:=w_{\acqr}; \\
           \kw{if}~(a == 1)~\{ \\
           ~~z:=1; \\
           \kw{else}~\{ \\
           ~~y_{\relw}:=1; \\
           \} \\
           b:=x; \comment{1} \\
           \kw{if}~(b == 1)~\{ \\
           ~~z:=1; \\
           \}
         }
\]

\noindent This behavior contradicts both of \Cref{thm:drfpf,thm:drfra}, and yet it is allowed in the
absence of promise re-certification.  Consider the following execution.  First, Thread 1 writes $w=1$
and Thread 3 promises to write $z=1$.  Now Thread 3 reads $w=0$, after which it can no longer
certify to write the promised value $z=1$.  Then Thread 3 writes $y=1$, Thread 2 reads $y=1$ and
$z=1$, Thread 2 writes $x=1$, Thread 3 reads $x=1$, and then Thread 3 fulfills the promise to write
$z=1$.



\subsection{An Invariant-Based Program Logic}
\label{sec:invariant}
Besides the DRF guarantees, to demonstrate that our model does not suffer from
the disastrous consequences of OOTA, 
we prove soundness of a very simple program logic for concurrent programs with respect to our model.
In particular, it can be trivially used to show that \ref{eq:LBdep} must return $a=0$,
and more generally, that programs cannot read values they never wrote.
Note that even this basic logic is unsound for C/C++'s relaxed accesses
(whereas it is sound in our model even if all accesses are plain).

%As we proved strengthening to be a sound transformation (see~\Cref{sec:transformations}), 
%it suffices to show soundness for programs that have only plain accesses.
%In addition, since updates are clearly stronger than a corresponding pair of a read followed by a write,
%it suffices to consider programs without updates.

We take a \emph{program proof} to be a tuple $\tup{J,S_1,S_2,\ldots}$, 
where $J$ is a global invariant over the shared variables
and each $S_i \subseteq \textsf{State}_i$ is a set of local states 
(intuitively describing the reachable states of thread $i$) such that the following conditions hold:
\begin{itemize}
  \setlength\itemsep{-2pt}
\item $\sigma_i^0 \in S_i$
and $\bigwedge_{x\in\Loc} x=0 \vdash J$.
\item
If $\sigma_i \astep{\rlab(o,x,v)} \sigma_i'$ then
$\sigma_i\in S_i \land J \land x = v \vdash \sigma_i'\in S_i$.
\item
If $\sigma_i \astep{\wlab(o,x,v)} \sigma_i'$ then
$\sigma_i\in S_i \land J \vdash \sigma_i'\in S_i \land J[v/x]$.
\item
If $\sigma_i \astep{\ulab(o_\lr,o_\lw,x,v_\lr,v_\lw)} \sigma_i'$ then \\
$\sigma_i\in S_i \land J \land x = v_\lr \vdash \sigma_i'\in S_i \land J[v_\lw/x]$.
\item
For $e\in \{\flab_\acqo,\flab_\relo,\flab_\sco,\silentt,\syscallt\}$,
if $\sigma_i \astep{e} \sigma_i'$ then
 $\sigma_i\in S_i \vdash \sigma_i'\in S_i$.
\end{itemize}
\Cref{fig:logic} provides an illustration of a program proof showing that \ref{eq:LBdep}
 does not exhibit weak behaviors.

Now, given a program proof for a program $\prog$, we can show that
all the reachable states $\mconf$ from the initial state $\mconf^0$ of $\prog$
satisfy the global invariant $J$:

\begin{theorem}[Soundness]
Let $\tup{J,S_1,S_2,\ldots}$ be a program proof, 
and let $\mconf=\tup{\lts,\gsco,\mem}$ such that $\mconf^0 \astep{}^* \mconf$.
Then, $\lts(i).\lstate \in S_i$ for every thread $i$, and 
$\bigwedge_{x\in\Loc} x=f(x).\lval \vdash J$
for every function $f$ that assigns to every location $x$ a message $m\in\mem$ such that $m.\lloc=x$.
\end{theorem}

Our Coq proof of this theorem is simple:
it holds trivially for promise-free executions,
and extends easily to promise steps, since every promise step has a promise-free certification.

\begin{figure}[t]
\centering
\small
\begin{minipage}{0.55\columnwidth}
 \[
\begin{array}{c}
\begin{array}{c||c}
 \begin{array}{l}
\asrt{J } \\
a:=x; \\
\asrt{J \land (a=0)} \\
y:=a; \\
\asrt{J} 
 \end{array}&
 \begin{array}{l}
\asrt{J } \\
x:=y;  \\
\asrt{J } 
 \end{array}
 \end{array}
 \end{array}
 \]
 \end{minipage}
\begin{minipage}{0.40\columnwidth}
$$ J \defeq \left( x=0\right) \land \left(y=0\right)$$ 
 \end{minipage}
\caption{A simple derivation in the invariant-based program logic}
\label{fig:logic}
\end{figure}


%
%We call a thread state $\gts$ valid with respect to $\tup{S_1,S_2,\ldots}$
%if $\gts(i).\lstate \subseteq S_i$ for every thread $i$.
%Similarly, we call a memory $\mem$ valid with respect to a global invariant $J$
%if for every mapping $f$ of location to messages such that $\bigcup_{x\in\Loc}\tup{x,f(x)}\suq\mem$,
%we have $\bigwedge_{x\in\Loc} x=f(x).\lval \implies J$.
%We call a machine state $\tup{\gts,\mem}$ valid with respect to a program proof $\tup{J,S_1,\ldots}$
%if $\mem$ is valid with respect to $J$ and $\gts$ is valid with respect to $\tup{S_1,\ldots}$.
%
%Given a program proof for a program, we can show that
%all the reachable states $\mconf$ from the initial state $\mconf^0$ 
%are valid with respect to the proof.
%
%\ori{make this in a theorem environment?}
%
%\ori{say something about the proof and refer to appendix}
%
%\ori{appendix has to be adapted}
%
%




%%% Local Variables:
%%% mode: latex
%%% TeX-master: "main"
%%% TeX-command-extra-options: "-shell-escape"
%%% End:
