\newcommand{\figrelaxed}{
\begin{figure*}[t]
\small
\begin{mathpar}
%
\vspace*{-1mm}
%
\inferrule[\textsc{(thread: silent)}]
{
\sigma \astep{\silentt} \sigma' 
}
{\tup{\tup{\sigma, \rlxcur, \lprom}, \mem} \astep{} \tup{\tup{\sigma', \rlxcur, \lprom}, \mem}}
\and
\inferrule[\textsc{(thread: read)}]
{
 \sigma \astep{\rlab(x,v)} \sigma' \\
\rlxmsg{x}{v}{t} \in \mem \\\\
\rlxcur(x) \leq t \\
\rlxcur'=\rlxcur[x\mapsto t]
}
{\tup{\tup{\sigma, \rlxcur, \lprom}, \mem} \astep{} \tup{\tup{\sigma', \rlxcur', \lprom}, \mem}}
\and
\inferrule[\textsc{(thread: write)}]
{
 \sigma \astep{\wlab(x,v)} \sigma' \\
\mem'=\mem \insertadd \rlxmsg{x}{v}{t} \\\\
\rlxcur(x) < t \\
\rlxcur'= \rlxcur[x\mapsto t]
}
{\tup{\tup{\sigma, \rlxcur, \lprom},\mem} \astep{} \tup{\tup{\sigma', \rlxcur', \lprom}, \mem'}}
%
\\\vspace*{-1mm}
%
\inferrule[\textsc{(thread: promise)}]
{
\mem' = \mem \insertadd m \\
\lprom' = \lprom \insertadd m 
}
{\tup{\tup{\sigma, \rlxcur, \lprom},\mem} \astep{} \tup{\tup{\sigma, \rlxcur, \lprom'},\mem'}}
\and
\inferrule[\textsc{(thread: fulfill)}]
{
 \sigma \astep{\wlab(x,v)} \sigma' \\
\rlxmsg{x}{v}{t} \in \lprom \\
\lprom'=\lprom\setminus \setofz{\rlxmsg{x}{v}{t}} \\\\
\rlxcur(x) < t \\
\rlxcur' = \rlxcur[x\mapsto t] 
}
{\tup{\tup{\sigma, \rlxcur, \lprom},\mem} \astep{} \tup{\tup{\sigma', \rlxcur', \lprom'}, \mem}}
\and
\inferrule[\textsc{(machine step)}]
{\tup{\gts(i),\mem} \astep{}^+ \tup{\lts',\mem'} \\\\
\tup{\lts',\mem'} \text{ is consistent}}
{\tup{\gts,\mem} \astep{} \tup{\gts[i\mapsto \lts'],\mem'}}
\end{mathpar}
\caption{Operational semantics for the simplified model handling only relaxed read and write accesses.}
\label{fig:relaxed-opsem}
\end{figure*}}

\section{Basic Model for Handling Relaxed Accesses}
\label{sec:relaxed}

In this section, we introduce the key ideas of our memory model, first
by example and then more formally.  At first we will only consider a
semantics for fully ``relaxed'' atomic read and write accesses (in the
sense of C++).  This is a natural starting point, since the OOTA
problem is fundamentally about how to give a reasonable semantics for
these relaxed accesses, and the key elements of our solution are
easiest to understand in this simpler setting.  We will see in
subsequent sections how to extend and generalize this base model to
account for a much richer variety of memory operations.

To illustrate our semantics, we will write small programs such as the
following:
\begin{equation}\label{eq:SB}\tag{SB}
\inarrII{x:=1;\\a:=y;\comment{0}}{y:=1;\\b:=x;\comment{0}}
\end{equation}
As a convention, we write $a$, $b$, $c$ for local variables
(registers) and $x$, $y$, $z$ for (distinct) shared memory locations,
and assume that all variables are initialized to $0$.  
We refer to thread $i$ as $T_i$.  Moreover, in order to
refer to a specific observation of the program, we annotate the
corresponding reads with the values expected to be read (\eg in the
above program, the comment notation indicates the observed result that
$a = b = 0$).

\subsection{Main Ideas}

\paragraph{High-Level Requirements: Reorderings and Coherence}

Relaxed read and write operations are intended to be compiled down
directly to plain loads and stores at the machine level, so one of the
main requirements of our semantics is that it be at least as
permissive as commodity hardware.  Toward this end, our semantics must
justify reordering of independent memory operations (\ie operations
that access distinct locations), since the more weakly consistent
architectures (like ARM) may potentially perform such reorderings.
There are four such classes of reorderings---write-read, write-write,
read-read, and read-write---and in \cref{sec:results} we will prove formally
that our semantics justifies all of them.

On the other hand,
% to support fundamental high-level reasoning
% principles, 
it is also important that our semantics not be
unnecessarily weak.  In particular, all the existing 
implementations of C++, even for weaker architectures like Power and
ARM, guarantee at a bare minimum a property we call \emph{per-location
  coherence} (aka \emph{SC-per-location}).  Per-location coherence
says that, even though threads may observe writes to different
locations in different orders, they must observe writes to the
\emph{same} location in a single total order (called the
``modification order'' in C++ lingo).
%, and it must be consistent with the
%``program order'' (the order in which writes are issued within any
%single thread). \ori{last sentence seems too partial to me: I suggest to remove
 In addition to being supported by hardware,
per-location coherence is preserved by common compiler optimizations
as well.  Hence, we want our semantics of relaxed accesses to
guarantee it. (In \cref{sec:plain} we will present an even weaker mode of accesses that
does not provide full per-location coherence.)

%\jeehoon{we do not mention merges; I think it is reasonable at this stage, but I just want to make a note here.}

% This cut-down language is sufficient to understand
% the core of our solution to the OOTA problem without , and it is enough to 
% We begin with relaxed reads and writes because
% the OOTA problem arises already in this cut-down language, In subsequent sections, we will do some more.
% We will extend this base model to handle
% atomic update operations in the next section, as well as more strongly
% synchronized operations in the following sections.

\paragraph{Operational Semantics with Timestamps}

In contrast to the C++ memory model, which relies on declarative
semantics over event graphs, ours employs a more standard SC-style
operational semantics for concurrency, in which the executions of
different threads are nondeterministically interleaved.  However, in
order to account for weak memory behaviors, we use a more elaborate
memory representation than the standard SC semantics does.  Instead of
being a flat map from addresses to values, our memory records the set
of all writes ever performed.  It may help to think of writes as
messages, and memory as a message pool which grows monotonically.  When
a thread $T$ reads from a location $x$, it need not read ``the
latest'' write to $x$, since there is no shared understanding among
threads of what the latest write is.  The thread $T$ thus retains
flexibility in terms of which message it reads, but we must place some
restrictions on this flexibility in order to guarantee per-location
coherence.

% Unlike SC, which
% executes all memory writes in a total order, our relaxed semantics
% (following C++) only guarantees that writes to any \emph{single}
% variable are executed in some total order (called the ``modification
% order'' in the C++ model), which must be consistent with the ``program
% order'' (the sequential order in which writes are issued within any
% single thread).  This property, which we call \emph{per-location
%   coherence}, is the minimal property guaranteed by all hardware
% implementations, and is preserved by most compiler optimizations.

Specifically, we totally order the writes to each location by
attaching a (unique) \emph{timestamp} to each write message.  Thus,
messages are triples of the form $\rlxmsg{x}{v}{t}$ (where $x$ is a
location, $v$ a value, and $t$ a timestamp).  (The \emph{modification
order} for a location $x$ is thus implicitly derivable from the order
of timestamps on $x$'s messages.)  In addition, for each thread $T$,
we keep track of a map from locations $x$ to the largest timestamp of
a write to $x$ that $T$ has observed or executed.  We refer to this
map as $T$'s \emph{view} of memory, and one can think of it as
recording the set of most recent write messages that $T$ has observed.
Hence, when $T$ reads from a location $x$, it must read from a message with a
timestamp \emph{at least as large} as the one recorded for $x$ in
$T$'s view.  And when $T$ writes to $x$, it must pick a timestamp
\emph{strictly larger} than the one recorded for $x$ in its view.
%This simple mechanism is sufficient to directly obtain per-location coherence.

Let us see now how our semantics, as explained thus far, already suffices to 
justify desirable reorderings while ruling out violations of coherence.
First, recall the write-read reordering exhibited by the ``store buffering'' \ref{eq:SB}
example above, and let us see how the behavior can be justified.
% As explained thus far, our semantics with timestamps already suffices
% to justify the write-read reordering exhibited by the ``store buffering'' \ref{eq:SB} example above.  
%Initially, both threads maintain a view of $x$ and $y$
%that maps them to whatever timestamps $t_x^0$ and $t_y^0$ were
%associated with the $0$-initializations.  When $T_1$ writes $x:=1$, it
%will update its view of $x$ to some $t_x^1 > t_x^0$, but this will
Initially, assume the memory contains the initialization messages 
$\rlxmsg{x}{0}{\ts{0}}$ and $\rlxmsg{y}{0}{\ts{0}}$,
and both threads maintain a view of $x$ and $y$ that maps them to $\ts{0}$.
When $T_1$ performs the assignment $x:=1$, it
will choose some timestamp $t > \ts{0}$,
add the message $\rlxmsg{x}{1}{t}$ to the memory, 
and update its view of $x$ to $t$. But this will
have no effect on its view of $y$ or $T_2$'s view of $x$, which remain at $\ts{0}$.  
Thus, when $T_1$ subsequently reads $y$, it is free to read $0$.
(And analogously for $T_2$.)

On the flip side, per-location timestamps also explain why the following coherence
violation is forbidden.
\begin{equation}\label{eq:COH}\tag{COH}
\inarrII{x:=1;\\a:=x;\comment{2}}{x:=2;\\b:=x;\comment{1}}
\end{equation}
Here, the two writes to $x$ must be totally ordered.  Suppose, without
loss of generality, that the $x:=1$ was written at timestamp $\ts{1}$ and
$x:=2$ at timestamp $\ts{2}$.  Then, although $T_1$ may read value $2$,
$T_2$ cannot read $1$, because $1$ was written at a smaller timestamp
than the one that $T_2$ already recorded in its view when it wrote $x:=2$.

% \ori{the order seems a bit strange to me, I think the last paragraph
% should appear before the paragraph about SB}
% \derek{I think it's OK.}

One subtle point is that, when writing to a location $x$, a
thread $T$ may select any unused timestamp $t$ larger than the one
recorded in its view of $x$, but $t$ need not be globally maximal.
That is, $t$ may be smaller than the timestamp that another thread has
already used for a write to $x$.  %Though this freedom may seem odd, it
%is in fact crucial in order to permit write-write reorderings, as
This freedom is in fact crucial in order to permit write-write reorderings, as
exemplified by the following test case:
\begin{equation}\label{eq:2+2W}\tag{2+2W}
\inarrII{x:=2;\\y:=1;\\a:=y;\comment{2}}{y:=2;\\x:=1;\\b:=x;\comment{2}}
\end{equation}
To get the desired weak outcome, the writes of $x:=1$ and $y:=1$ must
pick smaller timestamps than the $x:=2$ and $y:=2$ writes,
respectively, but at least one of the $1$-writes must be executed
\emph{after} the $2$-write to the same location.  Thus, it is essential
to be able to write using a timestamp that is not globally maximal.

\paragraph{Promises}

%As we have seen, 
% Timestamps account for write-read and write-write
% reorderings, and indeed (as we prove in \cref{sec:results}) they also account
% for read-read reorderings.  
% \ori{actually, our formal proof of WW-reordering uses promises,
% should we just say here that timestamps dont suffice to explain RW?}
Unfortunately, our timestamp semantics alone does not suffice
to explain \emph{read-write} reorderings, as exemplified by the (LB)
and (LBfd) programs from \cref{sec:oota}.  It is precisely these
reorderings that motivate our introduction of \emph{promises}.

As explained in \cref{sec:promising}, a thread $T$ may at any point
\emph{promise} to write $x:=v$ at some timestamp $t$ (provided that
$t$ is greater than $T$'s current view of $x$).  This promise is
treated to a large extent like an actual write operation.  In
particular, it adds a new message $\rlxmsg{x}{v}{t}$ to memory, which may then be
read by other threads.  However, in order to make such a promise, $T$
must \emph{thread-locally certify} it---that is, $T$ must demonstrate
that it will be able to fulfill this promise (writing $x:=v$ at
timestamp $t$) in a finite number of thread-local steps.  Certification
is needed to guarantee plausibility of the promise, but crucially,
there is no requirement that the specific steps of execution taken
during certification must match the subsequent steps of actual execution.
Indeed, we already witnessed this with the (LB) and (LBfd) executions,
where $T_1$ read $x=0$ during the initial certification of its promised
write to $y$, but read $x=1$ during the actual execution.

% We have seen in \cref{sec:promising} how promises justify
% the read-write reordering of (LB) and (LBfd) while prohibiting the
% OOTA behavior of (LBd). 
Let us now briefly touch on a few technical points concerning the
interaction of promises and timestamps.

% First of all, a key technical observation: just because a thread reads
% a particular write message during promise validation does not mean it
% must read that message during the subsequent actual execution.  We
% already witnessed this with the (LB) execution, where $T_1$ read $x=0$
% during the initial validation of its promised write to $y$, but read
% $x=1$ during the actual execution.

First of all, it is important that $T$ cannot directly read its own
promises, because this would violate per-location coherence: for
example, the single-threaded program $a:=x;~ x:=1$ would be able to
return $a=1$!  Note that we do not need to explicitly enforce this
restriction---it just falls out from our rules concerning timestamps.
In particular, if $T$ were to promise $\rlxmsg{x}{v}{t}$, and then
were to read from its own promise, then $T$'s view of $x$ would be
updated to $t$, and there would be no way for $T$ to subsequently
fulfill the promise because it would have to 
pick a timestamp strictly greater than $t$ when performing 
the assignment $x:=v$.

That said, it is possible for $T$ to read its promised value
indirectly via another thread, as in the \ref{eq:LB} and
\ref{eq:LBfalsedep} programs.
It may even read the promised value
from the same location where it promised to write it, as in the
following example~\cite{fm16}.
\begin{equation}\label{eq:ARMweak}\tag{ARM-weak}
\inarrIII{ a:=x; \comment{1} \\ x:=1; }{ y:=x; }{ x:=y; }
\end{equation}
This outcome can be explained by $T_1$ promising
$\rlxmsg{x}{1}{\ts{2}}$, then $T_2$ reading $x=1$ and storing it to
$y$, and $T_3$ reading $y=1$ and writing $x:=1$ at timestamp $\ts{1}$,
which $T_1$ can read before fulfilling its promise.  Such behavior,
strange though it may seem, is actually allowed (though not yet
observed) by the ARM memory model~\cite{arm8-model}.

% Having explained
% the idea of promises in \cref{sec:promising}, we will now clarify how
% promises interact with timestamps.

% \derek{Editing here.}

% At any point during execution a thread can promise to perform a write
% of a certain value at a certain location and at a certain timestamp.
% Naturally, when making a promise, a thread should be able to fulfill
% it.  More specifically, we require that the thread making the promise
% should be able to execute in isolation and fulfill all its outstanding
% promises by performing the corresponing writes.  Thus, in the
% \ref{eq:LB} and \ref{eq:LBfalsedep} examples, the first thread can
% promise to write $x:=1$, because by reading from the initial state it
% can fulfill its promise.  In contrast, in \ref{eq:LBdep}, the $x:=1$
% promise cannot be made since it is impossible to fulfill
% thread-locally.  Intuitively, our thread-local validation of promises
% ensures that ``out of thin air'' values can never be promised: we will
% formalize this intuition in \cref{sec:relaxed-logic}.
% % Consider the following two
% % variants of the load-buffering program:
% % \begin{center}
% % \begin{minipage}{.47\columnwidth}
% % \begin{equation}\label{eq:LB}\tag{LB}
% % \inarrII{ a:=x; \comment{1} \\ y:=1; }{ x:=y; }
% % \end{equation}
% % \end{minipage}
% % ~\vrule\hfill
% % \begin{minipage}{.49\columnwidth}
% % \begin{equation}\label{eq:LBdep}\tag{LBd}
% % \inarrII{ a:=x; \comment{1} \\ y:=a; }{ x:=y; }
% % \end{equation}
% % \end{minipage}
% % \end{center}
% % In the first case, the annotated behavior can easily arise if the compiler or the hardware reoders the two independent accesses on the first thread.
% % In the second case, however, because of the data dependency between them, no accesses may be reordered.
% % Reading $1$ is moreover totally unjustified, because it never appears in the program.

% % \TODO{can LB be observed on ARM??? or Power???}

% % A naive solution to this problem would be for the semantics to buffer loads and delay their execution to the point when their value is actually used.
% % This approach has been successfully applied in definitions of hardware memory models, 
% % but it fails to validate standard compiler optimizations, as these may remove syntactic dependencies.
% % As a very simple example, consider the following program, 
% % that can be trivially transformed to \ref{eq:LB} by a basic optimizing compiler.
% % \begin{equation}\label{eq:LBfalsedep}\tag{LBfd}
% % \inarrII{ a:=x; \comment{1} \\ y:=a+1-a; }{ x:=y; }
% % \end{equation}

% % \paragraph{Promises}

% % We propose to solve this problem by a different mechanism, which we call \emph{promises}.
% Basically, at any point during execution a thread can promise to perform a write of a certain value at a certain location and at a certain timestamp.
% Naturally, when making a promise, a thread should be able to fulfill it.
% More specifically, we require that the thread making the promise should be able to execute in isolation and fulfill all its outstanding promises by performing the corresponing writes.
% Thus, in the \ref{eq:LB} and \ref{eq:LBfalsedep} examples, the first thread can promise to write $x:=1$, because by reading from the initial state it can fulfill its promise. 
% In contrast, in \ref{eq:LBdep}, the $x:=1$ promise cannot be made since it is impossible to fulfill thread-locally.
% Intuitively, our thread-local validation of promises ensures that ``out of thin air'' values can never be promised: we will formalize this intuition in \cref{sec:relaxed-logic}.

% Once a write has been promised, other threads may read it. 
% The thread making the promise cannot directly read its promise, because this would violate per-location coherence:
% for example, the single-threaded program $a:=x; x:=1$ would be able to return $a=1$!
% It may, however, read its promised value indirectly via another thread as in the \ref{eq:LBdep} and \ref{eq:LBfalsedep}.
% It may eventually even get its promised value to the same location as in the following example.
% \begin{equation}\label{eq:ARMweak}\tag{ARM-weak}
% \inarrIII{ a:=x; \comment{1} \\ x:=1; }{ y:=x; }{ x:=y; }
% \end{equation}
% This behavior, strange though it may seem, is actually allowed by the ARM memory model~\cite{armv8}.
% In our semantics, it can be explained by the first thread promising to write $x:=1$ at timestamp $2$,
% then the second thread reading $x=1$ and storing it to $y$ 
% and the third thread reading $y=1$ and writing $x:=1$ at timestamp $1$, 
% which the first thread can read before fulfilling its promise.

\figrelaxed

Last but not least, we wish to ensure that promises do not lead to
impossible situations later down the road, \ie that making a promise
cannot cause the execution of a program to get stuck.  The
thread-local certification that accompanies a promise step goes some way
toward ensuring this progress condition, but it is not enough.  We
 also amend the semantics in the following two ways:
\begin{enumerate}
\item Every step a thread takes, it must \emph{re-certify} all its
  outstanding promises to make sure they can \emph{still} be
  fulfilled.  To see why, consider a possible execution of the
  following program:
$$
\inarrII{a:=x;\\ x:=1; }{ x := 2; }
$$
%Let's assume that $x$ has been initialized to $0$ at timestamp $0$.
Suppose that $T_1$ (for no particularly good reason) promises $\rlxmsg{x}{1}{\ts{1}}$.
 At first, this is easy to certify: $T_1$
can read the initial value of $x$ (the message $\rlxmsg{x}{0}{\ts{0}}$), 
and then perform the assignment $x:=1$ picking timestamp
$\ts{1}$.  Suppose then that $T_2$ picks the timestamp $\ts{2}$ when performing $x:=2$. 
If at this point in the execution $T_1$ were permitted to read the message $\rlxmsg{x}{2}{\ts{2}}$, it
would have the effect of bumping up $T_1$'s view of $x$ to timestamp
$\ts{2}$, which would prevent it from subsequently fulfilling its promise.
It is thus crucial that $T_1$
\emph{not} be allowed to read $x=2$ (in this particular execution), and
indeed our semantics will not allow it to do so because the
re-certification check would fail.  As the example illustrates, promises
can restrict a thread's future nondeterministic choices concerning the
messages it reads.

% \footnote{They do not completely constrain those choices, however:
% just because a thread reads a particular write message
% during promise validation does not, in general, mean it must read
% that message during the subsequent actual execution.  In fact, we
% already witnessed this with the (LB) execution, where $T_1$ read
% $x=0$ during the initial validation of its promised write to $y$,
% but read $x=1$ during the actual execution.}

% $a := x; x := 1$.  Suppose we first promise to write $x := 1$ at
% timestamp $2$.

% We have
% already explained the thread-local validation that accompanies a
% promise step, but a one-time validation is not enough: every step a
% thread takes, it must re-validate that it can still fulfill all its
% outstanding promises.  This re-validation check is non-trivial.
% For example, suppose $T$ had promised to write $x$ at timestamp $2$,
% and could validate that such a write was possible.  In the absence
% of re-validation, the thread could 
% but then it read from $x$ at timestamp $3$.  It would 
% it prevents
% threads from reading a write to $x$ with a greater timestamp than any
% promises it has made concerning $x$.
% promises, because then the promises cannot be fulfilled at the
% appropriate timestamps.

% we we check at every step of
% a thread's execution that its outstanding promises can \emph{still} be
% fulfilled.  This is a non-trivial check: juThis check prevents threads
% from observing writes with greater timestamps than their promises,
% because then the promises cannot be fulfilled at the appropriate
% timestamps.  
\item We require the total order on timestamps to be \emph{dense} (\eg
  choosing timestamps to be rational numbers), so that there is always
  a place to put intermediate writes before a promise.  Consider, for
  example, the following program:
$$
\inarrII{x:=1;\\ x:=2; }{ x := 3; }
$$
%Again, let's assume that $x$ is initially set to $0$ with timestamp $0$. 
Here, $T_1$ may promise $\rlxmsg{x}{2}{\ts{2}}$---when
validating this promise, $T_1$ might write $\rlxmsg{x}{1}{\ts{1}}$
before writing $\rlxmsg{x}{2}{\ts{2}}$. If, however, $T_2$
subsequently writes $\rlxmsg{x}{3}{\ts{1}}$ before $T_1$ has actually
written $x:=1$, then $T_1$ can no longer pick $\ts{1}$ as a timestamp for
$x:=1$.  To make progress here, $T_1$ needs a timestamp for $x:=1$
strictly between $\ts{0}$ and $\ts{2}$, and $\ts{1}$ is already taken.
By requiring the timestamp order to be dense, we ensure that there is
always some free timestamp (\eg $\ts{1.5}$) that $T_1$ can use.
\end{enumerate}

\subsection{Formal Definition}
\label{sec:relaxed-formal}

We now define our model for relaxed accesses formally.  Let
$\Loc$ be the set of memory locations, 
$\Val$ be the set of values, 
and $\Time$ be an infinite
set of \emph{timestamps}, densely totally ordered by $\leq$,
with $\ts{0}$ being the minimum element.
 A \emph{timemap} is a function $T:\Loc \to \Time$.  The order $\leq$ is
extended pointwise to timemaps.
%We write $\Timemap$ for $\Loc \to \Time$. %% NOT USED

\paragraph{Programming Language}
To keep the presentation abstract, we do not fix a particular programming language;
we simply consider each thread $i$ as a transition system with a set of states $\textsf{State}_i$,
initial state $\sigma_i^0\in\textsf{State}_i$ and final state $\sigma^\fin_i\in\textsf{State}_i$.
Intuitively, these states store the values of the local registers and the program counter.
Transitions are labeled:
the label $\rlab(x,v)$ corresponds to a transition that reads the value $v$ from location $x$, 
and  $\wlab(x,v)$ denotes a write of the value $v$ to $x$, 
 while local transitions that do not access the memory are labeled with ``$\silentt$''.
We assume \emph{receptiveness} of the transition systems---whenever an $\rlab(x,v)$-transition is possible from a state $\sigma_i$,
so is an $\rlab(x,v')$-transition for every value $v'$---and
% some value can be read then all values can be read),
that they only get stuck in the $\sigma^\fin_i$ states.

%\jeehoon{we didn't suppose the receptiveness in Coq; is it okay?}
 
% (In the next sections we consider extensions of the programming language, that naturally employ
% a richer set of labels).
 
\paragraph{Messages}

A \emph{message} $m$ is a tuple $\rlxmsg{x}{v}{t}$,
where $x\in\Loc$, $v\in\Val$ and $t\in\Time$.
We denote by $m.\lloc$, $m.\lval$, and $m.\ltime$ the components of a message $m$.
%$\Msg$ denotes the set of all messages.
Two messages $m$ and $m'$ are called \emph{disjoint},
denoted $m \disj m'$, if $m.\lloc\neq m'.\lloc$ or $m.\ltime\neq m'.\ltime$.
Two sets $M$ and $M'$ of messages are called \emph{disjoint},
denoted $M \disj M'$, if $m \disj m'$ for every $m\in M$ and $m'\in M'$.
%A set $M$ of messages is called \emph{pairwise disjoint} 
%if each two distinct messages in $M$ are disjoint.

\paragraph{Memory}

A \emph{memory} is a pairwise disjoint finite set of messages.
%$\mem(x)$ is pairwise disjoint for every $x\in\Loc$.  \derek{Is $\mem$
%  a set of messages or a function?  We seem to want to use it both
%  ways.}  
A message $m$ may be \emph{(additively) inserted} into memory $\mem$ if $m$ is
disjoint from every message in $\mem$.
% When defined, it just adds the message $m$ to $\mem$.
Formally, the additive insertion $\mem \insertadd m$ is given by
$\mem \cup \setofz{m}$ and is only defined if $\mem \disj \setofz{m}$.
%
%\[\small
%\mem \insertadd m \defeq \begin{cases} 
% \mem \cup \setofz{m}  & \text{if } \mem \disj \setofz{m}  \\
%\mathit{undefined} & \text{otherwise} \end{cases}
%\]

\paragraph{Thread States and Configurations}

A \emph{thread state} is a triple $\lts=\tup{\sigma,\rlxcur,\lprom}$,
where $\sigma$ is the thread's local state, $\rlxcur$ is a timemap representing
the thread's \emph{view} of memory, %(see \cref{??}), 
and $\lprom$ is a memory that keeps track of the thread's outstanding
promises.  We denote by $\lts.\lstate$, $\lts.\lview$, and
$\lts.\lprmem$ the components of a thread state $\lts$.
In turn, a \emph{thread configuration} is a pair $\tconf=\tup{\lts,\mem}$,
where $\lts$ is a thread state and $\mem$ is a memory, called the
\emph{global memory}. Note that we will always have $\lts.\lprmem \suq \mem$.

\cref{fig:relaxed-opsem} shows the five reduction rules for thread
configurations.  The \textsc{silent} rule handles the case when the
program performs some local computation that does not affect memory.
The \textsc{read} rule handles the case when the program reads from a
location $x$.  The rule nondeterministically selects some message $m$
in the memory, whose timestamp is greater than or equal to the timestamp recorded
for $x$ in the thread's view, and returns its value; it also updates
the thread's view of $x$ to the timestamp of $m$.  The \textsc{write}
rule handles the case when the program writes to location $x$.  It
extends the memory with a new message for $x$, whose timestamp
$t$ is greater than the one recorded for $x$ in the thread's view, and
it updates the thread's view of $x$ to match $t$.  The
\textsc{promise} rule extends the memory and the thread's promise set
with an arbitrary new message $m$, whose timestamp is not already
present in the memory.  (The promise certification is handled
separately, as described below.)  Finally, the \textsc{fulfill} rule
is similar to the \textsc{write} rule, except that instead of
adding a message to the memory, it removes an appropriate message from
the thread's promise set $\lprom$.

We note that the \textsc{write} rule is redundant; we merely included it
to improve readability.  Any application
of \textsc{write} can be simulated by first promising the appropriate
message with the \textsc{promise} rule and then immediately fulfilling
the promise with the \textsc{fulfill} rule.

%Note that the \textsc{write} rule is, in fact, redundant, as any application
%Note that for conciseness, we have made the \textsc{write} rule always fulfill some outstanding promise.
%To simulate a non-promising write, a thread can first promise the appropriate message with the \textsc{promise} rule 
%and then immediately fulfill its promise with the \textsc{write} rule.

As we have already mentioned, we have to restrict thread executions so
that all promises a thread makes are fulfillable.  Thread
configurations satisfying this property are called \emph{consistent}.
Formally, a thread configuration $\tup{\lts,\mem}$ is \emph{consistent} 
if $\tup{\lts,\mem} \astep{}^* \tup{\lts',\mem'}$
for some $\lts'$ and $\mem'$ such that $\lts'.\lprmem = \emptyset$.
Notice that in the certification of a promise, it is formally possible
to make further promises.  Since, however, in the end all such
promises must be fulfilled, it is useless to make such promises.
(A proof of this property is included in our formal development.)
%, it is useless to make
%such promises except at the step right before their fulfilling \textsc{write}.

\paragraph{Machine States}

A \emph{machine state} $\mconf=\tup{\gts,\mem}$ consists of a
function $\gts$ assigning a thread state to every thread, and a (global) memory $\mem$.
The initial state $\mconf^0$ (for a given program) consists of 
the function $\gts^0$ mapping each thread $i$ to its initial state $\sigma_i^0$,
a current timestamp of $\ts{0}$ for every location, and an empty set of promises;
and the initial memory $\mem^0$ that has one initial message $\rlxmsg{x}{0}{\ts{0}}$ for each location $x$.
A machine takes a step (see the last rule in \cref{fig:relaxed-opsem})
whenever a thread can take several steps to some consistent configuration.
Note that we allow multiple thread steps in one machine step.
This is convenient in our proofs,
and can reduce the number of certifications during an execution of a program.

Finally, we can easily show that a machine never gets stuck unless
each thread $i$ has reached $\tup{\sigma_i^\fin,\rlxcur,\emptyset}$
for some view $\rlxcur$.  For non-final states, progress follows from
the receptiveness and progress assumptions about the programming
language, together with the invariant that no thread has a higher view
of any $x$ than the highest timestamp for $x$ in memory.  Another
crucial invariant is consistency: the \textsc{machine step} rule
demands that each machine step taken by a thread must preserve
consistency of the thread's own configuration, and it implicitly
preserves the consistency of other threads' configurations as well,
since they are free to ignore any new messages the thread has added.
When all threads reach their final states, consistency implies they
must have no promises left to fulfill.

% For final states, we know they must have no promises left to fulfill,
% thanks to consistency.  

% Finally, thanks to our receptiveness and progress assumptions about
% the programming language, we can easily show that a machine can never
% get stuck except when all threads have terminated (\ie when each
% thread $i$ has reached $\tup{\sigma_i^\fin,\rlxcur,\emptyset}$ for
% some view $\rlxcur$).  Consistency plays a crucial role in the
% argument: the \textsc{machine step} rule explicitly demands that each
% machine step taken by a thread must preserve consistency of the
% thread's own configuration, and it implicitly preserves the
% consistency of other threads' configurations as well, since they are
% free to ignore any new messages the thread has added.  When all
% threads reach their final states, consistency implies they must have
% no promises left to fulfill.

% We can easily show that a machine can never get stuck except when all
% threads have terminated (\ie when each thread has reached
% $\tup{\sigma_i^\fin,\rlxcur,\emptyset}$ for some view $\rlxcur$).
% \emph{Preservation:} By definition, each machine step taken by a
% thread must preserve consistency of the thread's own configuration,
% and it is guaranteed to preserve the consistency of other threads'
% configurations as well, since they can simply avoid observing any new
% messages the thread has added.  \emph{Progress:} Thanks to our
% assumptions about the programming language, the only way a thread
% could get stuck in a non-final state is if it tried to read some $x$
% while having a view of $x$ greater than any timestamp for $x$ in
% memory.  However, it is an invariant of the semantics that this
% situation is impossible.  Finally, when a thread is in a final state,
% consistency implies it must have no promises left.


% We can easily show that a machine can never get stuck except 
% when all threads have terminated 
% (\ie when each thread has reached $\tup{\sigma_i^\fin,\rlxcur,\emptyset}$ for some view $\rlxcur$).
% By construction, each thread step preserves consistency of its configuration.
% It is important to note that the consistency of the configurations of other threads is also preserved, 
% since they can always avoid observing any new messages added to memory.
% Finally, because of consistency, when a thread has terminated, it must have no promises left.


%validation at every step is important
%
%$$\inarrIII{
%a:=z \comment{1} \\
% \itne{x=0}{y:=1}
% }{
%z:=y 
%}{
%x:=1
%}$$





%%% Local Variables:
%%% mode: latex
%%% TeX-master: "main"
%%% TeX-command-extra-options: "-shell-escape"
%%% End:
