\newcommand{\figupdates}{
\begin{figure}[t]
\small
\begin{mathpar}
%
\vspace*{-1mm}
%
\inferrule[\textsc{(thread: fulfill update)}]
{
\begin{array}{c}
\sigma \astep{\ulab(x,v_\lr,v_\lw)} \sigma' \quad \updmsg{x}{v_\lr}{f_\lr}{t_\lr} \in \mem \\
m_\lw=\updmsg{x}{v_\lw}{t_\lr}{t_\lw} \quad m_\lw \in \lprom \quad \lprom'=\lprom \setminus \setofz{m_\lw} \\
\rlxcur(x) \leq t_\lr \quad \rlxcur'  = \rlxcur[x\mapsto t_\lw]
\end{array}
}
{\tup{\tup{\sigma, \rlxcur, \lprom}, \mem} \astep{} \tup{\tup{\sigma', \rlxcur', \lprom'}, \mem}}
\end{mathpar}
\caption{Additional rule for updates (all other rules are as before except all
messages $\rlxmsg{x}{v}{t}$ are replaced by $\updmsg{x}{v}{f}{t}$).}
\label{fig:updates}
\end{figure}}


\section{Supporting Atomic Updates}
\label{sec:updates}

In this section, we extend our basic model for relaxed accesses to
also handle (relaxed) \emph{atomic update}---aka \emph{read-modify-write}
(RMW)---instructions, such as fetch-and-add and compare-and-swap.
Updates are essential as a means to implement synchronization (\eg
mutual exclusion) between threads, but this also makes them tricky to
model semantically.  In particular, a successful update operation
performed by one thread will often have the effect of ``winning a
race'' and hence blocking (previously possible) update operations
performed by other ``losing'' threads.  This stands in contrast to the
updates-free fragment in \Cref{sec:relaxed}, in which threads are free
to ignore the messages of other threads.  Thus, to extend our model to
support updates, we must ensure that threads performing
updates cannot invalidate the already-certified promises of other
threads.

%  and avoid the possibility of threads getting stuck, we
% must modify our promises mechanism so as to ensure that threads
% performing updates cannot invalidate the already-certified promises of
% other threads.

An update is an atomic composition of a read and a write to the same
location $x$. However, unlike under SC, atomicity requires more than
just avoiding interference of other threads between the two operations.
Consider the following example
(taking $\fai{x,1}$ to be an atomic \emph{fetch-and-add} of $1$ to $x$,
which returns the value of $x$ before the increment):
% \jeehoon{We use FAA for the abbreviation for fetch-and-increment.  I think we should be consistent and say fetch-and-add here.}
\begin{equation}\label{eq:par-inc}\tag{Par-Inc}
\inarrII{ a:= \fai{x,1};}{b:=\fai{x,1};}
\end{equation}
Atomicity ensures that it is not possible for both threads to
increment $x$ from $0$ to $1$ (we must either get $a=1$ or $b=1$). To
obtain this, we require that the \emph{read timestamp} of the update
(\ie the timestamp of the write message that the update reads from)
immediately precede its \emph{write timestamp} (\ie the timestamp of
the write message that the update generates) in $x$'s modification
order, and that future writes to $x$ may not be assigned timestamps in
between them.  In the example above, if both of the updates were to
increment $x$ from $0$ to $1$, the write timestamp for one of the
updates would have to come between the read and write timestamps for
the other update.

To enforce this restriction, we extend messages to store a
continuous range of timestamps rather than a single timestamp.  Thus,
messages are now tuples of the form $\updmsg{x}{v}{f}{t}$ where
$x\in\Loc$, $v\in\Val$, and $f,t\in \Time$ satisfying
${f < t}$ or $f=t=\ts{0}$.
%\ori{to add: (unless $f=t=\ts{0}$ for messages appearing in the initial memory)}
We write $m.\lfrom$ and $m.\lto$ to denote the $f$
and $t$ components of a message $m$.  Intuitively, $m$ can be
thought of as \emph{reserving} the timestamps in the range
$(m.\lfrom,m.\lto]$; among these, $m.\lto$ is the ``real'' timestamp
of $m$, but the remaining timestamps in the range are
reserved so that other messages cannot use them.  Timestamp
reservation is reflected in the following revised definition of
message disjointness, which enforces that disjoint messages for the
same location must have disjoint ranges:
%\begin{multline*}
%\hspace{-9pt} m\disj m' ~\triangleq ~ m.\lloc \neq m'.\lloc ~\lor \\
%~ (m.\lto \neq m'.\lto ~ \land ~  (m.\lfrom,m.\lto] \cap (m'.\lfrom,m'.\lto] =\emptyset)
%\end{multline*}
\begin{multline*}
\updmsg{x}{v}{f}{t}\disj \updmsg{x'}{v'}{f'}{t'} ~\triangleq ~  x \neq x' ~~\lor~~
%% (t \neq t' ~ \land ~  
t\leq f' < t' ~~\lor~~ t'\leq f < t
%% (f,t] \cap (f',t'] =\emptyset
%% )
\end{multline*}
With timestamp reservation, we can easily ensure that the write timestamp of
an update is adjacent to its read timestamp in the modification order.
% $m.\lto$ can be thought of
% as the ``real'' timestamp marking when the message was sent; 
% in the timestamp range $(m.\lfrom,m.\lto]$, where $m.\lto$ can be thought
% as the real timestamp when the message is sent, and the range
% $(m.\lfrom,m.\lto)$ is the set of reserved timestamps, which can
% ensure adjacency with some previous message.  
Formally, we will say two messages $m$ and $m'$ are \emph{adjacent},
denoted $\mathsf{Adj}(m,m')$, if $m.\lloc=m'.\lloc$ and
$m.\lto = m'.\lfrom$.  In defining the semantics of updates, we will
then insist that the message that the update inserts into memory must
appear adjacently after the message that it reads from.
This suffices
to guarantee the correct outcome in the \ref{eq:par-inc} program
above. 

% The ability to reserve a range of timestamps is a mixed blessing. 
% On the one hand, it is needed to implement the atomicity of updates.
% In the \ref{eq:par-inc} program above, it is used to ensure the two updates happen one after another, and thus the final value of $x$ is $2$.
Although the introduction of timestamp reservation enables us to
easily model updates, it creates a complication for promises, namely
that timestamp reservations may invalidate the promise certifications
already performed by other threads.  Consider, for example, the
following program:
\begin{equation}\label{eq:upd-stuck}\tag{Upd-Stuck}
\inarrIII{ a:=x; \comment{1} \\ b:=\fai{z,1}; \comment{0} \\ y:=b+1; }{ x:=y; }{ \fai{z,1};}
\end{equation}
This behavior ought to be allowed, since hardware could reorder the
read of $x$ after the independent accesses to $z$ and $y$.  To produce
this behavior, following our semantics from the previous section,
$T_1$ could promise to write $y:=1$ because it can
thread-locally certify that the promise can be fulfilled (the
certification will involve updating $z$ from $0$ to $1$).  If,
however, $T_3$ then updates $z$ from $0$ to $1$, that will mean that
$T_1$ can no longer perform the update it needs to fulfill its
promise, and its execution will eventually get stuck.
%   The difference from the simplified model of \Cref{sec:relaxed}
% is that, with relaxed the ability to perform a specific update (\eg from $0$ to $1$)
% is tonow threads may no longer ignore messages of other threads,
% because updates must be atomic.

To avoid such stuck executions, we strengthen the check performed by
promise certification, \ie the consistency requirement on thread
configurations.  We require that each thread's promises are locally
fulfillable not only in the current memory, but also \emph{in any
future memory}, \ie any extension of the memory with additional
messages.  This quantification over future memories ensures that
thread configurations remain consistent whenever another thread
performs an execution step, and thus the machine cannot get stuck.

Returning to the above example, $T_1$ will not be permitted to promise
to write $y:=1$ in the initial state, precisely because that promise
could not be fulfilled under an arbitrary future memory (\eg one
containing the update of $T_3$, as we showed).  $T_1$ may, however,
first promise $\updmsg{z}{1}{\ts{0}}{\ts{1}}$, reserving the time
range from the initialization of $z$ up to its increment.  $T_1$
can fulfill that promise, because no future extension of the memory
will be able to add any messages in between.  After making that
promise, $T_1$ may then promise, \eg $\updmsg{y}{1}{\ts{3}}{\ts{4}}$, which it can
now fulfill under any extension of the memory.  With these promises in
place, $T_3$ will be prevented from updating $z$ from $0$ to $1$; it
will be forced to update $z$ from $1$ to $2$, which will not block the
future execution of $T_1$.

% With these two promises,
% the second thread may read $y=1$, write $x:=1$, and thus the first
% thread may read $x=1$ and fulfil its promises.

\figupdates

Our quantification here over \emph{all} future memories may seem
rather restrictive in that it completely ignores what can or cannot
happen in a particular program.  That said, we find it a simple and
natural way of ensuring ``thread locality''.  The latter is a guiding
principle in our semantics, according to which the set of actions a
thread can take is determined only by the current memory and its own
state.

Formally, we say that $\omem$ is a \emph{future memory} of $\mem$ if
${\omem = \mem \insertadd m_1 \insertadd \ldots \insertadd m_n}$
for some $n\geq 0$ and messages $m_1,\ldots,m_n$. And
we now say a thread configuration $\tup{\lts,\mem}$ is
\emph{consistent} if, for every future memory $\omem$ of $\mem$, there
exist $\lts'$ and $\mem'$ such that $\tup{\lts,\omem} \astep{}^* \tup{\lts',\mem'}$
and $\lts'.\lprmem = \emptyset$.

Finally, we extend the operational semantics for thread configurations
with one additional rule for update fulfillment shown in
\Cref{fig:updates}.  This rule forces its write to be adjacent in
modification order to its read.  As with ordinary writes, a normal
(non-promised) update step can be simulated by a promise step
immediately followed by fulfillment.  Note that the other rules remain
exactly the same; they simply ignore the $m.\lfrom$ component of
messages $m$.
%\derek{This is confusing.  We should use consistent notation between both
%figures if possible.}
% it reads a message $m_\lr$ from memory and fulfils a promised message $m_\lw$ 
% combining the conditions of the earlier \textsc{read} and \textsc{write} rules.
% In addition, it requires the two messages to be adjacent, thereby ensuring the atomicity of the update.

% \jeehoon{We did not discuss in the previous section that plain write equals promise plus fulfill.}
% \ori{we did, when explaining the rules}

% As we have seen, our model treats an update as a write that is
% guaranteed to generate a write message adjacently after an existing
% one.  But 
% Another way in which a thread may get stuck is if another thread
% depletes the range of free timestamps necessary.  Consider, for example, the
% following program:
% $$
% \inarrIII{ a:=y; \comment{2}\\ \itne{a=2}{x:=1};\\x:=2; }{ y := x }{ x := 1 }
% $$
% Suppose $T_1$ promises $\rlxmsg{x}{2}{1}{2}$, after which $T_2$ reads
% this message and writes $y:=2$.  At this point, $T_1$ is free to
% read 


% The first thread may promise to write $x:=2$ at a future timestamp range.
% The second thread may then read $x$ and write $y:=2$, and so if the first thread at this point it may return $a=2$.
% What could, however, happen instead is that the third thread could intervene and write $x:=1$ reserving the entire range from the initialization of $x$ to the promised write of $x:=2$.
% Such a write would then leave no space for the first thread to put its $x:=1$ write, and it would thus be blocked from returning $a=2$.
% What went wrong in this execution was that the third thread was able to deplete the range of free timestamps between the initialization of $x$ and the $x:=2$ write.
% To forbid this case, we require that messages added to the memory cannot be adjacently before an already existing message in the memory.
% Formally, the \emph{additive insertion} of a message $m$ into a memory $\mem$, 
% denoted by $\mem \insertadd m$, is given by:
% \[
% \mem \insertadd m \defeq \begin{cases} 
% \setofz{m} \cup \mem & \text{if } 
% %\setofz{m}\disj \mem \land 
% %{}\\&\phantom{\text{if }} 
% \forall m' {\in}\mem.~ m\disj m' \land \lnot \mathsf{Adj}(m,m') \\
% \mathit{undefined} & \text{otherwise} \end{cases}
% \]


% \TODO{Should we discuss the multistep machine step rule?}

% \[
% \inferrule*[Right=\textsc{(machine step)}]
% {\tup{\gts(i),\mem} \astep{}^* \tup{\lts',\mem'} \\\\
% \tup{\lts',\mem'} \text{ consistent}}
% {\tup{\gts,\mem} \astep{} \tup{\gts[i\mapsto \lts'],\mem'}}
% \] 







% \subsection{OLD STUFF}

% \begin{mathpar}

% \inferrule*[Right=\textsc{(split-promise)}]
% {
% \lprom'= \lprom \insertsplit m  \\
% \mem'= \mem \insertsplit m 
% }
% {\tup{\tup{\sigma, \cur, \lprom},\mem} \astep{} \tup{\tup{\sigma, \cur, \lprom'},\mem'}}

% \end{mathpar}


% \paragraph{Message.}

% The \emph{splitting insertion} of a message $m$ into a memory $\mem$, 
% denoted by $\mem \insertsplit m$, is given by:
% \[
% \mem \insertsplit m \defeq \!\!\begin{cases} 
% \setofz{m, m'[\lfrom\mapsto m.\lto]} \cup (\mem{\setminus}\setofz{m'}) & \text{if }
% m' \!\in\! \mem 
% \\
% \multicolumn{2}{@{}r@{}}{
% {}\land{}
% m.\lloc = m'.\lloc 
% }\\
% \multicolumn{2}{@{}r@{}}{
% {}\land{} 
% m.\lfrom = m'.\lfrom \land m.\lto < m'.\lto
% }\\
% \mathit{undefined} & \text{otherwise} \end{cases}
% \]
% where $m[\lfrom\mapsto t]$ denotes $\tup{m.\lloc,m.\lval,t,m.\lto}$.

% We denote by ${M \astep{m} M'}$ that the following holds:
% \[
% (M' = M\insertadd m) \lor (M' = M \insertsplit m)~.
% \]
% We write ${M\astep{} M'}$ if $M\astep{m} M'$ holds for some message $m$.
% %% We call $M'$ a \emph{future} of $M$ if $M \astep{}^* M'$ for 
% %% $\astep{}^*$ the reflexive transitive closure of $\astep{}$.

% We write $M \subseteq M'$ and $M \setminus M'$ for the usual set inclusion
% and minus:
% \[
% \begin{array}{rcl}
% M\subseteq M' &\iff& \forall m\in M.~ m \in M'~. \\
% M\setminus M' &\defeq&
% \setof{ m \suchthat m \in M \land m \notin M' } ~. \\
% \end{array}
% \]



% \paragraph{Memory.}
% A \emph{memory} is a function $\mem:\Loc \to \finpowerset{\Msg}$, such
% that $\mem(x)$ is valid for every $x\in\Loc$.  All (partial)
% operations and orders on sets of messages ($\insertadd$,
% $\insertsplit$, $\astep{}$, $\subseteq$, $\setminus$) are extended pointwise to
% memories (\eg the additive insertion $\mem \insertadd \tup{x,m}$ is
% defined as ($\lambda y.\; \texttt{if } y = x \texttt{ then }
% \mem(x)\insertadd m \texttt{ else }\mem(y)$) if $\mem(x)\insertadd m$ is
% defined; otherwise it is undefined).

% \paragraph{Closed Memory.}
% Given a timemap $A$ and a memory $\mem$, we write $A\tmin\mem$ 
% %% (resp. $A \tmle \mem$)
% if the following condition holds for every $x\in\Loc$:
% \[ A(x) = m.\lto %% \text{ (resp. $A(x) \leq m.\lto$)}
% \text{ for some } m\in\mem(x)~. \]
% For a capability $C$, we write $C \tmin \mem$ %% (resp. $C \tmle \mem$)
% if $A \tmin \mem$ %% (resp. $A \tmle \mem$)
% for each component timemap $A$ of $C$.


% \paragraph{Future Memory.}
% We say $\omem$ is a \emph{future memory} of $\mem$ w.r.t. $\lprom$
% if $(i)$ $\mem \astep{}^* \omem$; 
%    $(ii)$ $\lprom \subseteq \omem$.

% \paragraph{Thread Consistency.}
% A thread configuration $\tup{\lts,\mem}$ is called \emph{consistent} if
% $\exists \lts',\mem'.$
% \[
% \tup{\lts,\omem} \astep{\tau}^* \tup{\lts',\mem'}
% ~~\land~~ \lts'.\lprmem = \emptyset
% \]
% for every future memory $\omem$ of $\mem$ w.r.t. $\lts.\lprom$.

% A machine takes a step whenever a thread can take a step to a consistent configuration.
% \[
% \inferrule*[Right=\textsc{(machine step)}]
% {\tup{\gts(i),\mem} \astep{}^* \tup{\lts',\mem'} \\\\
% \tup{\lts',\mem'} \text{ consistent}}
% {\tup{\gts,\mem} \astep{} \tup{\gts[i\mapsto \lts'],\mem'}}
% \] 


% \begin{itemize}
% \item read a message with timestamp immediately before, and write
% \item  make sure thar no future msg gets in between: block an interval.
% \end{itemize}

% To explain the basic semantic:
% $$\inarr{\inarrII{ a:= x++;}{b:=x++;} \\a=1 \lor b=1}$$

% To explain that updates should be promised:

% $$\inarrII{
% a:=x; \comment{1} \\ 
%  y++;
% }{  
%   x:=y;}$$


% To explain quantification over future memories:
 
% $$\inarrIII{
% a:=x; \comment{1} \\ 
%  z++; \\
%  \itne{z=1}{y:=1};
% }{ 
% x:=y ;
% }{
%  z++;}
% $$
 
% To explain splitting and multiple thread steps in one machine step:

% $$\inarrII{
% a:=x; \comment{2} \\ 
% \ite{a=1}{y:=1; y++}{y:=2}
% }{  
%   x:=y;
% }$$
 
% 
%  
%  x++
%  x:=2
%  
%  








%%% Local Variables:
%%% mode: latex
%%% TeX-master: "main"
%%% TeX-command-extra-options: "-shell-escape"
%%% End:
