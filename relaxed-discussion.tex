\section{Future and Follow-up Work}\label{sec:discussion}

% there are a number of interesting issues remaining for future work

\paragraph{SC Accesses}

Lahav~\etal{} proposed a fix to SC accesses in C/C++11~\cite{rc11}.  Extending our model with SC
accesses is left for future work (see \Cref{sec:promising}).

% we discovered that our semantics of SC accesses was flawed, and this led us to discover a flaw in
% the C/C++11 standard as well!  (See \cite{rc11} for further details.)


\paragraph{Compilation Correctness}

Podkopaev~\etal{} established the correctness of compilation of our model to Power, ARMv7, and
RISC-V, as well as to ARMv8 using a sub-optimal compilation scheme for read-modify-update
instructions~\cite{imm}.  In fact, the optimal one used by mainstream compilers is unsound for our
semantics due to the fact that ARMv8's read-modify-update instructions perform ``global''
optimizations, which is beyond the reach of our semantics as we will explain shortly.

% Establishing the correctness of the optimal compilation scheme to ARMv8 is left for future work.

% as well as to Power using the more efficient compilation scheme for acquire reads and updates (see
% \Cref{sec:compilation_POWER}), is an important future goal.  To the best of our knowledge, we know
% of no counterexamples for correctness of compilation to these architectures.


%\paragraph{SC accesses}
% it seemed stupid to repeat what was already said in the intro.

\paragraph{Global Optimizations}
In our model, we insist that promises can always be certified thread-locally. 
This decision enables thread-local reasoning about our semantics and suffices
to justify all the known thread-local program transformations that a compiler
or the hardware may perform.  It does, however, render unsound some transformations of a global nature,
such as sequentialization (aka ``thread inlining''), which merges threads together.
To see this, consider the following:
$$\inarrIIId{
a:=x; \comment{1} \\
\itne{a=0}{}\\
\quad x:=1;
}{
y:=x;
}{
x:=y;
}\quad
\leadsto
\quad
\inarrIId{
a:=x; \comment{1} \\
\itne{a=0}{}\\
\quad x:=1; \\
y:=x;
}{
x:=y;
}
$$
This source program disallows the specified behavior because if $T_1$ reads $1$ for $x$ after promising $x:=1$, then
it will not be able to fulfill its promise.
Nevertheless, the result $a=1$ \emph{is} allowed in the target program 
(obtained by sequentializing $T_1$ before $T_2$).
Here, $T_1$ can safely promise $y:=1$, and later read $x=1$ from $T_2$'s 
write.\footnote{Though sequentialization is a very intuitive property that one might expect a memory
model to validate,
we observe that TSO~\cite{x86-tso}, Power~\cite{herding-cats}, ARMv8~\cite{arm8-model}, Java~\cite{jmm}, 
and C18/C++17~\cite{Batty:2011} (without the corrections proposed in \cite{c11comp})
all do not allow sequentialization.
%\ori{add references to the papers showing that sequentialization is invalid in each of them}
%\jeehoon{Maybe we can challenge the forklore knowledge that sequentialization is very intuitive, by arguing with the JMM-05 \& 10 examples (https://github.com/jeehoonkang/memory-model-explorer/issues/28)}
}

Furthermore, thrad-local certification also renders unsound the optimal compilation scheme to ARMv8.
Consider the following example~\cite[Example 3.10]{imm}:

$$\inarrIId{
a:=x; \comment{1} \\
z:=a
}{
b:=z; \comment{1} \\
c:=FAA(x,1,\relw); \comment{0} \\
y:=c+1
}
$$

\noindent This behavior is disallowed in our semantics, because a promise of $y = 1$---which is
required for the behavior---cannot be certified in the presence of release fetch-and-add.  However,
this behavior is allowed in ARMv8, essentially because it allows the global optimization which $(1)$
analyzes that \code{x} is modified only by the fetch-and-add instruction and \code{c} be the initial
value $0$, and $(2)$ transforms \code{y:=c+1} with \code{y:=1}.  After the optimization, $y=1$ can
be promised and the behavior is allowed.

Generalizing our semantics to support global optimizations and verifying the optimal compilation
scheme to ARMv8 is left for future work.

% While sequentialization seems like a transformation that no compiler would perform, there might be
% other more useful global optimizations.  Investigating what global optimizations are supported in
% our model is left for future work.

% \jeehoon{I think it would be nice to mention register promotion, which we currently don't support.}

%{Transformations based on invariant-based reasoning?}

\paragraph{Liveness}

It is natural to extend our operational model with liveness guarantees,
and it is useful and interesting to study their interaction with
 program transformations and DRF theorems.
Liveness properties are currently mostly ignored in weak memory research.

% \jeehoon{DRF is irrelevant here, I think.  Liveness reduces the set of possible behaviors, making the DRF theorem easier to prove.}

\ori{I think that with Liveness one would like to have stronger DRF, that provides also SC's 
liveness guarantees, whatever these are. No?}


\paragraph{Reasoning Principle}

The program logic presented in \Cref{sec:invariant} only establishes the very basic sanity of our
memory model.  Svendsen~\etal{promising-separation-logic} propopsed a separation logic for our
semantics that is capable of proving the absence of out-of-thin-air behaviors for various litmus
tests.  Developing reasoning principles for our semantics that is capable of verifying concurrent
data structures and algorithms, is a direction for future work.


\paragraph{Promising Semantics for Hardware Concurrency}

Using the main idea of our semantics---namely views and promises, Pulte~\etal{} proposed the
promising semantics for ARMv8/RISC-V which is a simpler and faster operational concurrency semantics
than the existing models for those architectures~\cite{promising-arm}.  Developing the promising
semantics for other architectures is left for future work.


\paragraph{Simulation and Model Checking}

\cite{promising-arm}.

The high degree of nondeterminism in our model makes it hard to exhaustively explore all possible
behaviors of a given program.  Further work is required to develop efficient methods and tools for
this purpose.

\jeehoon{Cite Promising ARM}


%% Based on our
%% understanding of ARMv8 architecture~\cite{}, we believe that our model is
%% weaker than ARMv8.


% \paragraph{Promise analysis}

% \jeehoon{promise anaysis?}




%%% Local Variables:
%%% mode: latex
%%% TeX-master: "main"
%%% TeX-command-extra-options: "-shell-escape"
%%% End:
