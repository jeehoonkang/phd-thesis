\section{Future Work}\label{sec:discussion}

Besides extending our model with SC accesses
(see \cref{sec:promising}), 
there are a number of interesting issues remaining for future work.

\paragraph{Compilation Correctness}
Establishing the correctness of compilation of our model to ARM, as well as
to Power using the more efficient compilation scheme for acquire reads and updates 
(see \cref{sec:compilation_POWER}), is an important future goal. 
To the best of our knowledge, we know of no counterexamples
for correctness of compilation to these architectures. 

%\paragraph{SC accesses}
% it seemed stupid to repeat what was already said in the intro.

\paragraph{Global Optimizations}
In our model, we insist that promises can always be certified thread-locally. 
This decision enables thread-local reasoning about our semantics and suffices
to justify all the known thread-local program transformations that a compiler
or the hardware may perform.  It does, however, render unsound some transformations of a global nature,
such as sequentialization (aka ``thread inlining''), which merges threads together.
To see this, consider the following:
$$\inarrIIId{
a:=x; \comment{1} \\
\itne{a=0}{}\\
\quad x:=1;
}{
y:=x;
}{
x:=y;
}\quad
\leadsto
\quad
\inarrIId{
a:=x; \comment{1} \\
\itne{a=0}{}\\
\quad x:=1; \\
y:=x;
}{
x:=y;
}
$$
This source program disallows the specified behavior because if $T_1$ reads $1$ for $x$ after promising $x:=1$, then
it will not be able to fulfill its promise.
Nevertheless, the result $a=1$ \emph{is} allowed in the target program 
(obtained by sequentializing $T_1$ before $T_2$).
Here, $T_1$ can safely promise $y:=1$, and later read $x=1$ from $T_2$'s 
write.\footnote{Though sequentialization is a very intuitive property that one might expect a memory
model to validate,
we observe that TSO~\cite{x86-tso}, Power~\cite{herding-cats}, ARMv8~\cite{arm8-model}, Java~\cite{jmm}, 
and C/C++11~\cite{Batty:2011} (without the corrections proposed in \cite{c11comp})
all do not allow sequentialization.
%\ori{add references to the papers showing that sequentialization is invalid in each of them}
%\jeehoon{Maybe we can challenge the forklore knowledge that sequentialization is very intuitive, by arguing with the JMM-05 \& 10 examples (https://github.com/jeehoonkang/memory-model-explorer/issues/28)}
}
While sequentialization seems like a transformation that no compiler would perform, there might be other more useful global optimizations.
Investigating what global optimizations are supported in our model is left for future work.

\jeehoon{I think it would be nice to mention register promotion, which we currently don't support.}

%{Transformations based on invariant-based reasoning?}

\paragraph{Liveness}

It is natural to extend our operational model with liveness guarantees,
and it is useful and interesting to study their interaction with
 program transformations and DRF theorems.
Liveness properties are currently mostly ignored in weak memory research.

\jeehoon{DRF is irrelevant here, I think.  Liveness reduces the set of possible behaviors, making the DRF theorem easier to prove.}

\ori{I think that with Liveness one would like to have stronger DRF, that provides also SC's 
liveness guarantees, whatever these are. No?}

\paragraph{Program Logic}

The program logic presented in \cref{sec:invariant} only establishes the very basic sanity of our memory model.
Developing a useful program logic for this model is a direction for future work.

\paragraph{Simulation and Model Checking}

The high degree of nondeterminism in our model makes it hard to 
exhaustively explore all possible behaviors of a given program.
Further work is required to develop efficient methods and tools for this purpose.

%% Based on our
%% understanding of ARMv8 architecture~\cite{}, we believe that our model is
%% weaker than ARMv8.





%%% Local Variables:
%%% mode: latex
%%% TeX-master: "main"
%%% TeX-command-extra-options: "-shell-escape"
%%% End:
