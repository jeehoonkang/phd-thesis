\section{Discussion}
\label{sec:intptrcast:discussion}

\subsection{Implementation and Experiment Details}

\paragraph{Coq Formalization}
All the proofs reported in this paper have been fully formalized in
Coq and can be found in the project webpage.  Our Coq formalization is about 10,000 lines of code,
excluding empty lines and library code.  The formalization took about 2
person-months to complete.

\paragraph{Optimization Examples}
All optimization examples presented in the paper are performed by Clang
3.4.2 and/or GCC 4.8.3.  Examples in C and their compilation
results can be found in the project webpage.


\subsection{Related Work on Formal Semantics}

\paragraph{Formal Memory Models}
There have been numerous efforts to formalize C semantics, both from
the perspective of clarifying the specification and defining
implementations with formal semantics~\cite{norrish1998c,leroy:compcert,ellison2012executable,krebbers2011formalization,Greenaway:2014:DSS:2594291.2594296}. These invariably use
variations of the logical memory model, where each allocation is
associated with some abstract identifier and pointers consist of
an identifier and some path representing an offset into the memory
block, except for the work of Norrish~\cite{norrish1998c} which uses the concrete model.

\paragraph{Comparison with CompCert}
CompCert~\cite{leroy:compcert,Leroy-Appel-Blazy-Stewart-memory-v2} and its various extensions
currently allow casting pointers to and from integers, but the
semantics preserves the logical representation of pointers after the cast. 
As a result, integer variables can contain not only normal 32-bit integers,
but also logical pointer representations.
In the higher-level languages (CompCert C and Clight), performing arithmetic
on cast pointer is treated as a program error, whereas in the low-level languages
(from Cminor down to assembly), adding and subtracting integer values 
from converted pointers is defined and affects only the offset into the pointer's 
logical block. There
has also been work on extending the semantics to support pointer
fragments to allow, for example, \code{memcpy} to work on memory
containing pointers~\cite{krebbers2014formal}, but these extensions still cannot fully 
support arithmetic operations on pointer values that have been cast
to an integer type.

\paragraph{Comparison with CompCertTSO}
The CompCertTSO compiler~\cite{vsevvcik2013compcerttso} extends CompCert's
Clight language with threading and atomic memory primitives following the
x86-TSO relaxed memory model. Similar to us, CompCertTSO's memory model 
also supports finite memory, but uses a different mechanism to do so.
It has a distinguished logical block, where the offset serves as essentially a
concrete memory address. 
%In addition to the operations on pointers defined
%under CompCert's Clight semantics, in CompCertTSO pointer equality is always
%defined within the same block. 
During compilation, all memory operations are
lowered to act only on a single finite logical block. This allows the source
and target languages, with infinite and finite memory respectively, to share a
single memory model, and simplifies the correctness statements by removing
the need for CompCert's memory injections. 
%We can replicate this behavior by having a fully concrete semantics, where
%all pointers become realized immediately upon allocation.
CompCertTSO handles pointer-integer casts in the same way as CompCert, with the same limitations.

\paragraph{Comparison with the Symbolic Value Approach}
Most recently, Besson et al. have proposed an extension to CompCert's
memory model that gives semantics to bit-masking operations on
pointers and uninitialized values~\cite{besson2014precise}. Their
approach involves adding lazily-evaluated symbolic expressions,
including arbitrary operations on the representation of pointers, to
the class of semantic values. Symbolic values are forced whenever a
concrete value is needed to take a step, for example to access memory
through a pointer or in the guard of a conditional. The mapping is
performed by a normalization function given as a parameter of the
semantics. The normalization function is partial, and is only defined
precisely when the symbolic value evaluates to a unique result under
every assignment from logical block identifiers to concrete addresses
(subject to some validity conditions).
% This is done using an auxiliary
% low-level semantics for expressions in which pointers and integers
% have the same representation.

The semantics of Besson et al. is necessarily deterministic: non-determinism is
interpreted as undefined behavior, while our model
captures the non-deterministic allocation of concrete addresses.
Furthermore, their semantics is complex and indeed intractable: their
normalization is implemented with an SMT solver, and the semantics in general is too complex to serve as a mental model for ordinary C programmers. Normalization in our semantics,
on the other hand, is a straightforward translation from
pointers to concrete blocks and integers.

Most importantly, while their approach gives semantics to non-strictly-conforming C programs
involving bit-masking of pointers and uninitialized values, it fails to define useful programs that
use integer-pointer casts.  Consider the \code{hash\_put} example discussed in
\Cref{sec:intptrcast:formal-semantics:ownership}, where a pointer is hashed and then presumably used
to index into an array.  Since the resulting memory location will depend on the concrete layout of
memory, the resulting program will have undefined behavior in their semantics.  In general, any
program that displays non-determinism due to the concretization of pointers in our model is necessarily
undefined in Besson et al.'s model.

% the model of Besson et al. Furthermore, it
% is not possible to extend their model to support these cases since
% determinism is required for the correctness of their semantics: memory
% only contains logical pointers and concrete realizations are
% enumerated independently at each normalization step. Allowing
% non-determinism would allow contradictory observations about the
% physical layout of memory. Supporting concrete pointers in memory is
% essential to ``remember'' previous decisions about the realization of
% logical pointers. Finally, we contend that their model fails to allow
% simple reasoning by the user.


\subsection{Compatibility}

\paragraph{Compatibility with Other C Language Features}

There are numerous other C language features that have some
interaction with the memory model.  Some of them, such as
\textit{indeterminate values} \cite[\S3.19.2p1]{iso2011iec}, \textit{dangling pointers}
\cite[\S6.2.4p2]{iso2011iec}, and \textit{infinite loops with no side-effects}
\cite[\S6.8.5p6]{iso2011iec}, have semantics that are largely orthogonal to
the pointer concretization used in our hybrid model.
Similarly, our model explicitly allows \textit{unsafely derived pointers},
which are permitted in C11 and implementation-defined in C++11 \cite[\S3.7.4p4]{iso2011iec}. \jeehoon{C++?}
We allow them in order to support low-level programming idioms such as XOR linked lists and
compressed oops in HotSpot JVM.

Our paper does not directly address threads, so we cannot claim with
certainty that the model extends to handle them.  However, we see no
obstacles in this direction, and the hybrid model is similar
to CompCertTSO, which does support a weak memory model and threads, so
we are optimistic that this extension to the semantics should follow similarly.

A few language features require some adaptation of our memory model.
For instance, we can adapt the hybrid model to support
\textit{union types} and \textit{strict aliasing}, following Krebbers'
technique~\cite{krebbers2013aliasing}, which works regardless of
whether the model is concrete, logical or hybrid.

As another example, in C, \code{char*} is a ``universal'' pointer type, which allows
efficient bulk data moves via \code{memcpy}.  Krebber's variant of
CompCert~\cite{krebbers2014formal} already supports this semantics using a logical
memory, and the hybrid model is compatible with that solution.
Briefly: we let \code{char} types store byte-indexed logical values
(such as $(l,10)\!:\!2$, which denotes the second byte of the logical
address $(l,10)$). This strategy works because a \code{char} is
implicitly cast to an integer when used in arithmetic operations,
and thus we
can simply treat these casts as side-effecting (\textit{i.e.},
concretizing the logical addresses).  This approach lose (almost) no
optimization opportunities because byte-indexed logical addresses
are typically loaded from the memory and 
thus (mostly) already treated as public by the compiler.


% *** What is the role of our model?

% We intend to use our model to build a verified translation validation
% framework for LLVM (eventually) supporting all C features used in
% practice.  Cleanly modeling pointer-to-integer casts is a necessary
% step toward this goal.  With this application in mind, we strived to
% make the model simple (necessary for proving correctness of
% optimizations) yet permissive (allowing us to justify many
% optimizations).

% Our model is not intended to be a replacement for the C standard;
% rather it is the specification of a particular compiler implementation
% that refines the standard.

% *** Can our model evolve to an actual language spec?

% We believe that our model can be extended to support all practically
% used features of C, and that all these extensions are orthogonal to
% the realization of pointers we propose.  Here we elaborate on them.

% 1. Unsafely-derived pointers. 
% They are allowed in C11 and implementation-defined in C++11 [C++11
% 3.7.4p4]. We chose to allow them in order to support low-level
% programming idioms such as XOR linked lists and compressed oops in
% HotSpot JVM.  (Disallowing them would enable garbage collection in
% C/C++, which is rare.)

% 2. Threads \& Concurrency.
% Our paper does not directly address threads, so we cannot claim with
% certainty that the model extends to handle them.  However, we see no
% obstacles in this direction, and our model is similar to CompCerTSO,
% which does support a weak memory model and threads, so we are
% optimistic that this should be straight forward.

% 3. Char*/memcpy
% A variant of CompCert [with Krebber's patch, in github] already
% supports char*/memcpy in logical memory, and our model is compatible
% with that solution.  Briefly: we let char types store logical values
% (eg, such as (l,10):2 denoting the second byte of the logical address
% (l,10)). This is OK because chars are implicitly cast to integers by
% arithmetic operations. We can simply treat these casts as
% side-effecting (ie, realizing any logical addresses). Here, we lose
% (almost) no optimization opportunities by realizing such logical
% addresses because they are stored in the memory and thus (mostly)
% already treated as public by the compiler.

% 5. 
% *** Applicability to existing verified compilers?

% We believe that our ideas are readily applicable to CompCert(TSO) and
% related projects like Vellvm because our memory model and notion of
% memory invariant are technically very close to CompCert's. (Vellvm
% also uses CompCert's memory model.) Essentially, all that would have
% to change in the proofs are the cases handling pointer to integer
% casts.

\paragraph{Compatibility with Alias Analyses} The hybrid model is largely
compatible with common alias analyses.  For instance, it can be used
to justify \textit{size-based alias analysis}, which considers
pointers to differently-sized objects as distinct. For example, in
the code below, there is no alias between \code{p} and \code{q}:
even if \code{q} points to the block
pointed to by~\code{p}, loading or storing a double value in the
block will fail since the block is not big enough to contain double
values.
\begin{center}
  \begin{minted}{c}
    int *p = malloc(sizeof(int));
    double *q = foo(p);    // no alias between p and q
  \end{minted}
\end{center}

It also justifies %(modulo the effects of cast)
\textit{freshness-based alias analysis}, which assumes that the result of
\code{malloc} is distinct from all other pointers.  The 
following example of constant propagation is valid in the
hybrid model since \code{q} points to a fresh block that
is different from the block pointed to by \code{p}.
It is important to note that there is no alias
between \code{p} and \code{q} even after
the fresh block is concretized. The reason is because
even if \code{p} and \code{q} may be cast to the same integer,
they still point to different blocks as pointer values.
%% However, uncommenting the cast
%% invalidates the transformation because  \code{p} might happen to point
%% to the realized physical address of \code{q}. In practice, such cast
%% pointer values are usually global and passed to other functions, such
%% as \code{hash\_put} (as shown in the second comment). In
%% such cases, neither GCC nor Clang performs the constant propagation.
\begin{center}
\begin{tabular}{@{}l@{}l@{~~}l}
\small
\begin{minipage}{0.45\textwidth}
\begin{minted}{c}
void foo(int *p) {
  auto q = (int *) malloc(4);
  auto a = (uintptr_t) q;
  auto b = *p;
  *q = 123;
  auto r = *p;
}
\end{minted}
\end{minipage}
&
$~\rightarrow$
&
\small
\begin{minipage}{0.45\textwidth}
\begin{minted}{c}
void foo(int *p) {
  auto q = (int *) malloc(4);
  auto a = (uintptr_t) q;
  auto b = *p;
  *q = 123;
  auto r = b;         // CP
}
\end{minted}
\end{minipage}
\end{tabular}
\end{center}




\subsection{Impact}

In the course of doing this research, we discovered the GCC bug presented in
\Cref{fig:introduction:bug}.  With this bug, we persuaded compiler writers that $(1)$ it is subtle
to define the semantics of casts between pointers and integers, and $(2)$ it is safe to turn off
some alias analyses that are too aggressive to peacefully coexist with other optimizations.

Our idea to give semantics to more programs involving pointer operations---namely, using concrete
and logical blocks at the same time---has subsequently been refined by follow-up papers by other
researchers~\cite{intptrcast-oopsla,intptrcast-popl}, which are accompanied with promising revision
proposals to the LLVM compiler and ISO C Standard.

% The hybrid model refines the C standard by giving semantics to more programs involving pointer
% operations.

We intend to use this model for compiler verification tasks, extending the range of common
optimizations that can be verified.  Ultimately, we would like to generalize
Crellvm~\cite{crellvm}---the credible compilation framework for LLVM---to support casts between
pointers and integers.  We would also like to integrate our model with CompCert and use it to
justify new CompCert optimizations.  We believe that our ideas are readily applicable to
CompCert(TSO) and related projects like Vellvm~\cite{vellvm:popl12,vellvm:pldi13} because our memory
model and notion of memory invariant are technically very close to CompCert's.  (Vellvm also uses
CompCert's memory model.)  Essentially, all that would have to change in the proofs are the cases
handling pointer to integer casts.

% all commonly-used features of C.


%% 

% \paragraph{Conclusion and Future Work}
% Future work 1: Revising CompCert in our way requires a lot of overhead? \cite{leroy:compcert}

% Conclusion: our model applies to C, LLVM IR and CompCert(TSO).


%%% Local Variables:
%%% mode: latex
%%% TeX-master: "main"
%%% TeX-command-extra-options: "-shell-escape"
%%% End:
