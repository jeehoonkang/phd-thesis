\section{Running Example: Constant Propagation}
\label{sec:sepcomp:constprop}

In this section, we explain how to adapt the CompCert proof of
constant propagation to support separate compilation.  Before doing
so, let us briefly explain syntactic linking in some more detail since
it is central to the compositional correctness results we prove.

Syntactic linking merges global declarations for each identifier.
The linker must check if the declarations meet the following conditions:
\begin{itemize}
\item The declarations have the same type.  They should be either
  $(i)$ function declarations or definitions of the same signature, or
  $(ii)$ variable declarations or definitions of the same type.
\item At most one of the declarations is a definition.  If there is a
  single definition, then that is the result of the linking;
  otherwise, everything is necessarily the same declaration, and that
  is the result of the linking.
\end{itemize}
\noindent We have generically defined syntactic linking for all the
languages used in CompCert, as it does not depend on specific language
features.

% Next, we describe how we can adapt the CompCert proof of constant propagation to achieve separate compilation.

\subsection{Verifying Compositional Correctness Level A}

\paragraph{Adapting the Simulation Relation Definition}

To verify compositional correctness Level A, we will---as in the
original CompCert proof---construct a simulation relation $R$ that
relates the initial states of the source and target programs.  The
difference is that the source program consists of multiple files, each
of which is separately compiled.
\[
\frac{
\begin{array}{@{}c@{}}
\vprg = \mathtt{s}_1\circ \ldots \circ \mathtt{s}_n \qquad
\vprg' = \mathcal{T}_\mathrm{cp}(\mathtt{s}_n)\circ \ldots \circ \mathcal{T}_\mathrm{cp}(\mathtt{s}_n)
\end{array}
}
{\exists R.~ \mathrm{simulation}~R \land (\rtlload(\vprg),\rtlload(\vprg'))\in R
}
\]
Therefore, for each function definition ($\vfd$) in the source program
($\vprg$), the corresponding function definition ($\vfd'$) in the
target program ($\vprg'$) is no longer obtained by
$\mathrm{transfun}(\vprg,\vfd)$, but rather by
$\mathrm{transfun}(\mathtt{s}_i,\vfd)$ for some subprogram
$\mathtt{s}_i$ of $\vprg$.  Moreover, the value analysis run as part
of $\mathrm{transfun}$ also gets a subprogram of $\vprg$ as its first
argument.

Consequently, the simulation relation we use for the proof of soundness has to change.
The main change is, naturally, in the definition of $\simFun$.
Two function definitions are now related if the second can be obtained by transforming the first in the context of a subprogram of $\vprg$.
\[ 
\vprg \vdash \vfd \simFun \vfd' ~\defeq~ \exists \vsprg \subseteq \vprg.~ \vfd' = \mathrm{transfun}(\vsprg,\vfd)
\]
where $\vsprg \subseteq \vprg$ iff $\exists \vsprg'.~\vsprg \circ
\vsprg' = \vprg$.\footnote{In the Coq development, we define $\vsprg
  \subseteq \vprg$ in an equivalent but more direct, semantic way,
  rather than relying on syntactic linking ($\circ$).  This enables us
  to avoid having to prove associativity and commutativity of
  linking.}

The second change is to decouple the two uses of $\vprg$ in $\soundstate$,
changing its signature so that it takes three arguments: two programs $\vprg$ and $\vsprg$, and a state $s$.
The first program, $\vprg$, corresponds to the full program and is used to calculate the global environment,
whereas the second program, $\vsprg$, is a subprogram of the first one and is used to perform the global analysis
(\ie to detect which variables are constant). 

We then define a wrapper predicate $\soundstate'$ as follows:
\[
\soundstate'(\vprg,s) ~\defeq~ 
  \forall \vsprg \subseteq \vprg.~ \soundstate(\vprg,\vsprg,s)
\]
and change $R$ to use $\soundstate'$ instead of $\soundstate$.

%A final difference is that also have to parametrize the simulation relation by the target program, $\vprg'$
%is preserved by constant propagation.

\paragraph{Adapting the Proof of Value Analysis}

There are multiple proofs that require adaptation.
First, we have to prove that value analysis is correct with respect to our stronger invariant.
That is, we have to show that $\soundstate'$ holds of the initial state of a loaded program and that it is preserved by execution steps.

The latter requirement is actually trivial to show and requires only
very minor changes to the CompCert proof script.  The reason for this,
informally, is that the uses of the $\vprg$ and $\vsprg$ parameters in
the revised $\soundstate$ invariant are really decoupled and that the
preservation proof never depends on them being the same.

% \input{constprop-bug} TODO

The former requirement, however, requires some more work:
not only should $\soundstate(\vprg,\vprg,\load(\vprg))$ hold for all programs $\vprg$,
but rather $\soundstate(\vprg,\vsprg,\load(\vprg))$ should hold for all programs $\vprg$ and all subprograms $\vsprg\subseteq\vprg$.
In essence, to satisfy this stronger statement, the additional requirement that we have to show is that value analysis is monotone with respect to linking.
That is, for each global variable \texttt{x}, we prove
\[
\vsprg \subseteq \vprg \implies  \mathrm{abs\text{-}val}(\vsprg,\texttt{x}) \sqsupseteq \mathrm{abs\text{-}val}(\vprg,\texttt{x})
\]
where $\mathrm{abs\text{-}val}(\vprg,\texttt{x})$ is the abstract value of \texttt{x} computed by the global analysis on $\vprg$.
In other words, running the analysis on a larger program may only give results that are at least as precise.

Somewhat surprisingly, (the global part of) the value analysis in
CompCert 2.4 does not satisfy this monotonicity requirement because of
its treatment of variables declared as both \cd{extern} and
\cd{const}.  Figure~\ref{fig:constprop-bug} presents two C files that,
when compiled separately and linked together, expose the bug.  Since
the global variable \cd{xptr} is declared using the \cd{const}
qualifier, the global part of CompCert 2.4's value analysis assumes
that it is uninitialized and therefore assigns it the abstract value
$\bot$.  As a result, the value analysis deems that \cd{xptr} cannot
possibly alias with \texttt{x}.  At the \cd{printf} statement, it
hence deduces that $\texttt{x}=1$, and constant propagation
``optimizes'' away the summation \cd{x+x} to the constant $2$, which
gets printed.  This analysis, however, is unsound.  In particular, the
assumption that \cd{xptr} is uninitialized is invalid in the context
of multiple separately compiled files: since \cd{xptr} is also
declared as \cd{extern}, another file (\cd{b.c}) can provide a
definition that initializes it.  And indeed, since the definition of
\cd{xptr} in \cd{b.c} causes \cd{x} and \cd{xptr} to alias, the
correct result is $0$, not $2$.

Restoring soundness of the value analysis is straightforward: one
simple if rather crude fix (which has been adopted by CompCert 2.5
since we reported the bug) is just to ignore the \cd{const} modifier
on \cd{extern} declarations.  Having done that, it is easy to show
that the analysis is monotone with respect to program linking and that
therefore the initial state of loading a program satisfies the
stronger invariant $\soundstate'(\vprg,\load(\vprg))$.


\paragraph{Adapting the Proof of Constant Propagation}

Showing that constant propagation is sound requires only very little additional work.

The only important difference is in the treatment of global environments that index the $\estep{}$ relation.
Generally, these environments are obtained by the respective programs ($ge=\rtlgenv(\vprg)$ and $ge'=\rtlgenv(\vprg')$). 

The original CompCert 2.4 proof used the fact that
$\vprg'=\mathcal{T}_{\rm cp}(\vprg)$ to establish a relationship
between $ge$ and $ge'$.  It proved two basic properties relating the
two global environments: (1) that they map each global variable name
to the same block identifier, and (2) that if $ge$ maps a block
identifier to a function signature or definition $\vfds$, then $ge'$
maps it to $\mathrm{transfun}(\vprg,\vfds)$, where $\mathrm{transfun}$
applied to a function signature returns the same signature.  These
lemmas were then used in the proof that $R$ is a simulation relation.

Since, now, the relationship between $\vprg$ and $\vprg'$ is more involved, 
we have to update the proof of the first lemma, which is rather straightforward,
as well as the statement and proofs of the second lemma.
For the second lemma, we now assert that 
if $ge$ maps a block identifier to a function signature or definition $\vfds$, 
then $ge'$ maps it to $\mathrm{transfun}(\vsprg,\vfds)$ for \emph{some} subprogram $\vsprg\subseteq\vprg$.
Besides this change, 
the proof that now (the new definition of) $R$ is a simulation relation is basically unchanged.
The few lines of the proof script that required editing were those invoking the lemmas about the relationship between the global environments.


\subsection{Verifying Compositional Correctness Level B}

\paragraph{Adapting the Simulation Relation Definition}

We move on to verifying the second level of compositional correctness, that of composition with the same compiler modulo optimization flags.
As we have explained in \S\ref{sec:overview:LevelB}, for every optional optimization pass, we need to show that linking it against the identity compiler is sound.
Since constant propagation is one of the optional optimization passes, we have to prove the following:
\[
\frac{
\begin{array}{@{}r@{~}l@{}}
\vprg &= \mathtt{u}_1\circ \ldots \circ \mathtt{u}_m\circ\mathtt{s} \circ \mathtt{v}_1\circ \ldots \circ \mathtt{v}_n\\
\vprg' &= \mathtt{u}_1\circ \ldots \circ \mathtt{u}_m\circ\mathcal{T}_{\rm cp}(\mathtt{s}) \circ \mathtt{v}_1\circ \ldots \circ \mathtt{v}_n
\end{array}
}
{\exists R.~ \mathrm{simulation}~R \land 
(\rtlload(\vprg),\rtlload(\vprg'))\in R
}
\]
In the scenario above, the corresponding target function definition $\vfd'$ of a source function definition $\vfd$
is either syntactically identical to $\vfd$ if it belongs to one of the $\mathtt{u}_i$/$\mathtt{v}_j$ files, 
or has been obtained by optimizing $\vfd$ if it belongs to the $\mathtt{s}$ file.
To account for the change, we should therefore redefine the $\simFun$ relation as follows:
\[ 
\begin{array}{l}
\vprg \vdash \vfd \simFun \vfd' ~\defeq~ 
\\\qquad\qquad
\vfd' = \vfd \lor (\exists \vsprg \subseteq \vprg.~ \vfd' = \mathrm{transfun}(\vsprg,\vfd))
\end{array}
\]
This is the only change needed in the simulation relation.

%% \[
%%  \frac{
%% \begin{array}{c}
%% \vfd = \vfd' \qquad
%%  rs \le_\rtldef rs'
%% \end{array}
%%  }
%%  {
%% \vprg \vdash (r,\vfd,sp,pc,rs) \simFrame (r,\vfd',sp,pc,rs')
%%  }
%% \]

%% \[
%%  \frac{
%% \begin{array}{c}
%%  m \sqsubseteq_\mathrm{ext}  m' \quad
%%  s \simStack s' \quad
%% \vfd = \vfd' \quad
%%  rs \le_\rtldef rs'
%% \end{array}
%%  }
%%  {
%% \vprg \vdash
%% \rtlist~m~s~\vfd~sp~pc~rs \simState
%% \rtlist~m'~s'~\vfd'~sp~pc~rs'
%%  }
%% \]


%% \[
%%  \frac{
%% \begin{array}{@{}c@{}}
%% m \sqsubseteq_\mathrm{ext}  m' \quad
%% s \simStack s' \quad
%% \vfd = \vfd' \quad
%% \vargs \le_\rtldef \vargs' 
%% \end{array}
%%  }
%%  {
%% \vprg \vdash
%% \rtlcst~m~s~\vfd~\vargs \simState 
%% \rtlcst~m'~s'~\vfd'~\vargs'
%%  }
%% \]

\paragraph{Adapting the Proof of Constant Propagation}

To avoid changing the proof script of the main lemma showing that $R$
is in fact a simulation, we prove a helper lemma
(\textit{transf\_step\_correct\_identical}) saying that, given two
matching instruction states with $\vfd'=\vfd$, when the source state
takes a step, the target state can also take a step and reach a
matching state.  As explained in \S\ref{sec:overview:LevelB}, the
content of the proof of this helper lemma is already present in the
existing simulation proof, just not in one place, so it simply needs
to be extracted and consolidated.  We then adapt the proof of the main
lemma (showing $R$ is a simulation) so that it performs a case split on
whether $\vfd=\vfd'$ or not, and correspondingly either invokes our
helper lemma or uses the same proof script as for Level A.


%%% Local Variables:
%%% mode: latex
%%% TeX-master: "main"
%%% TeX-command-extra-options: "-shell-escape"
%%% End:
