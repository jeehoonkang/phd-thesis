\section{Results}

\newcommand{\specialcell}[2][c]{%
  \begin{tabular}[#1]{@{}c@{}}#2\end{tabular}}
\definecolor{light-gray}{gray}{0.8}

We applied our Level A and B techniques to CompCert 2.4 with three
patches applied (two bug fixes and RTL-level optimization flags),
resulting in \emph{\sepcomp-LevelA} and \emph{\sepcomp-LevelB}.
In the former, we prove the behavioral refinement result between the
source C program obtained by syntactically linking several C files and
the target assembly program obtained by syntactically linking the
results of compiling source files with the \emph{same} optimization
flag.  In the latter, we prove the same behavioral refinement result
even when each source file is compiled with a different optimization
flag.

\begin{landscape}
\begin{figure*}
\begin{center}
{\scriptsize \input{sepcomp-results-table.tex}}
\end{center}
{\vskip -1.2em
  \[\begin{array}{lll}
      \text{\textbf{Rm}: LOC removed from CompCert} &
      \text{\textbf{AddD}: LOC added but derived from CompCert} &
      \text{\textbf{AddN}: LOC newly added} \\
      \text{\textbf{\%}: ratio to the LOC of CompCert} &
      \text{\colorbox{light-gray}{shaded cell}: interesting changes}
    \end{array}\]}
\caption{Changes to lines of code when porting CompCert to \sepcomp-LevelA and \sepcomp-LevelB.}
\label{fig:results}
\end{figure*}
\end{landscape}

In the table in \Cref{fig:results}, we summarize the changes we
made in number of lines of Coq.  For these statistics, we first split
the development of CompCert into two categories: $(1)$ Compiler \&
Verification; and $(2)$ Metatheory (\ie everything else). The former
includes all Coq files in the directories \code{cfrontend},
\code{backend}, \code{driver}, \code{arm}, \code{ia32}, and
\code{powerpc}; and the latter includes all other files.

We calculated the statistics fully automatically using the Unix
\code{diff} command. The column \textbf{Rm} shows the number of
lines of code~(LOC) removed from the original CompCert code reported
by the diff command, and the columns \textbf{AddD} and \textbf{AddN}
together show the number of LOC added in \sepcomp.  Here,
\textbf{AddD} counts the LOC that are derived from the original
CompCert code including copy-pasted-and-modified code and \textbf{AddN}
the LOC that we newly proved. We syntactically marked the newly proved
code in \sepcomp, so that we can automatically distinguish
\textbf{AddD} and \textbf{AddN}.

The shaded part in the table denotes interesting changes we made.  In
Level A, the shaded part is mainly due to proving various monotonicity
properties. In Level B, the shaded part is mainly due to
copy-paste-and-modifying the original proof of simulation.  Examples of
(what we consider) uninteresting changes, which are not shaded,
include (a) proving straightforward ``wheel-greasing'' lemmas that
merely serve to streamline the proof effort, and (b) updating
automatically generated hypothesis names appropriately (\eg changing
\code{apply H1} to \code{apply H2}).

\sepcomp{} supports verified separate compilation to all three target
assembly languages of CompCert---PowerPC, ARM, and IA32---with very
few changes made to the original proofs. In the whole development, for
Level A, we modified only 0.2\% of the existing code (\textbf{Rm}) and
introduced an additional 1.6\%
(\textbf{AddN}+\textbf{AddD}-\textbf{Rm}); and for Level B, we
modified only 0.2\% of the existing code (\textbf{Rm}) and introduced
an additional 2.8\% (\textbf{AddN}+\textbf{AddD}-\textbf{Rm}).  We
spent less than two person-months in total for the whole development,
much of which was spent understanding the existing CompCert
development.



%%% Local Variables:
%%% mode: latex
%%% TeX-master: "main"
%%% TeX-command-extra-options: "-shell-escape"
%%% End:
