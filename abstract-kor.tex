\keywordalt{C, 실행의미, 컴파일러 검증, 느슨한 동시성 메모리, 분할 컴파일, 정수-포인터 변환}
\begin{abstractalt}
  주류 C 컴파일러들은 프로그램의 성능을 높이기 위해 공격적인 최적화를 수행하는데, 그런 최적화는
  저수준 기능을 사용하는 프로그램의 행동을 바꾸기도 한다.  불행히도 C 언어를 디자인할 때 저수준
  기능과 컴파일러 최적화를 적절하게 조화시키가 굉장히 어렵다는 것이 학계와 업계의 중론이다.  저수준
  기능을 위해서는, 그러한 기능이 시스템 프로그래밍에 사용되는 패턴을 잘 지원해야 한다.  컴파일러
  최적화를 위해서는, 주류 컴파일러가 수행하는 복잡하고도 효과적인 최적화를 잘 지원해야 한다.  그러나
  저수준 기능과 컴파일러 최적화를 동시에 잘 지원하는 실행의미는 오늘날까지 제안된 바가 없다.

  본 박사학위 논문은 시스템 프로그래밍에서 요긴하게 사용되는 저수준 기능과 주요한 컴파일러 최적화를
  조화시킨다.  구체적으로, 우린 다음 성질을 만족하는 느슨한 동시성, 분할 컴파일, 정수-포인터 변환의
  실행의미를 처음으로 제안한다.  첫째, 기능이 시스템 프로그래밍에서 사용되는 패턴과, 그러한 패턴을
  논증할 수 있는 기법을 지원한다.  둘째, 주요한 컴파일러 최적화들을 지원한다.  우리가 제안한
  실행의미에 자신감을 얻기 위해 우리는 논문의 주요 결과를 대부분 Coq 증명기 위에서 증명하고, 그
  증명을 기계적이고 엄밀하게 확인했다.
\end{abstractalt}

%%% Local Variables:
%%% mode: latex
%%% TeX-master: "main"
%%% TeX-command-extra-options: "-shell-escape"
%%% End:
