%!TEX root = main.tex
\newcommand{\figfull}{
\begin{figure*}[t]
\small
\begin{mathpar}
%\inferrule[\textsc{(memory: promise)}]{
%\hookleftarrow{}\in \setofz{\insertadd, \insertsplit, \insertupdate} 
%}{\tup{\lprom, \mem } \astep{\text{promise}(m)} \tup{\lprom \hookleftarrow m, \mem \hookleftarrow m}}
%\and
%\inferrule[\textsc{(memory: write)}]{
%o \sqsupseteq \ra \implies \forall m'{\in}\lprom.\; m'.\lloc \neq m.\lloc \\\\
%\tup{\lprom, \mem} \astep{\text{promise}(m)} \tup{\lprom', \mem'}
%}{\tup{\lprom, \mem}  \astep{\text{write}(o,m)} \tup{\lprom' \setminus \setofz{m}, \mem'}}
%\and
%
\vspace*{-1mm}
%
\inferrule[\textsc{(memory: new)}]{
\ %o \sqsupseteq \ra \implies \forall m' \in \lprom.\; m'.\lloc \neq m.\lloc 
}{\tup{\lprom, \mem}  \astep{m} \tup{\lprom, \mem \insertadd m}}
\and
\inferrule[\textsc{(memory: fulfill)}]{
\hookleftarrow \in \setofz{\insertsplit, \insertupdate} \\
\lprom'= \lprom \hookleftarrow m\\ \mem'= \mem \hookleftarrow m 
}{\tup{\lprom, \mem}  \astep{m} \tup{\lprom' \setminus \setofz{m}, \mem'}}
%
\\\vspace*{-1mm}
%
\inferrule[\textsc{(read-helper)}]{
 {\begin{array}{@{}c@{}c@{~}c@{~}c@{}}
o =\pln  & \implies & \cur.\lur(x) &\leq t \\
o \in \{\rlx,\ra\}  & \implies & \cur.\lrw(x) &\leq t  
\end{array}}\\\\
 {\begin{array}{@{}c@{~}c@{~}l@{}l@{}}
\cur' & = & \cur  \sqcup \view \sqcup  \iteb{o \sqsupseteq {\makebox[0pt][l]{$\ra$}\phantom{\rlx}}}{\mrel}  \\
\acq' &=& \acq \sqcup  \view \sqcup \iteb{o \sqsupseteq \rlx}{\mrel}   
\end{array}}\\\\
\textit{where\ }\view=[\lur:\iteb{o \sqsupseteq \rlx}{\{\vtup{x}{t}\}}, \lrw: \{\vtup{x}{t}\}]
}
%{\tup{\tup{\cur,\acq,\rel},\gsco}  \astep{\rlab:o,x,t,\mrel} \tup{\tup{\cur',\acq',\rel},\gsco}}
{{\tup{\cur,\acq,\rel}}  \astep{\rlab:o,x,t,\mrel} {\tup{\cur',\acq',\rel}}}
\and
\inferrule[\textsc{(write-helper)}]{
\cur.\lrw(x) < t \\\\
\cur'  = \cur \sqcup \view \\
\acq'   =  \acq \sqcup \cur' \\\\  
\rel'  = \rel[x\mapsto \rel(x) \sqcup \view \sqcup \iteb{o\sqsupseteq\ra}{\cur'}]\\\\ 
\mrel_\lw = \iteb{o \sqsupseteq \rlx}{(\rel'(x) \sqcup \mrel_\lr)} \\\\
\textit{where\ }\view=[\lur:\{\vtup{x}{t}\}, \lrw: \{\vtup{x}{t}\}]
}
%{\tup{\tup{\cur,\acq,\rel},\gsco}  \astep{\wlab:o,x,t,\mrel_\lr,\mrel_\lw} \tup{\tup{\cur',\acq',\rel'},\gsco}}
{{\tup{\cur,\acq,\rel}}  \astep{\wlab:o,x,t,\mrel_\lr,\mrel_\lw} {\tup{\cur',\acq',\rel'}}}
\and
%\inferrule[\textsc{(acq-fence-helper)}]{
%\cur'=\acq
%}{\tup{\tup{\cur,\acq,\rel},\gsco}  \astep{\flab_\acqo} \tup{\tup{\cur',\acq,\rel},\gsco}}
%~
%\inferrule[\textsc{(rel-fence-helper)}]{
%\rel'= \lambda \_.\cur
%}{\tup{\tup{\cur,\acq,\rel},\gsco}   \astep{\flab_\relo} \tup{\tup{\cur,\acq,\rel'},\gsco}}
%~
\raisebox{9pt}{
\inferrule[\textsc{(sc-fence-helper)}]{
\gsco'=\acq.\lrw \sqcup \gsco \\\\
\cur'=\acq'=\tup{\gsco',\gsco'} \\\\
\rel'=\lambda \_.\tup{\gsco',\gsco'}
}{\tup{\tup{\cur,\acq,\rel},\gsco}  \astep{\flab_\sco}\\\\\quad \tup{\tup{\cur',\acq',\rel'},\gsco'}}
}
%
\\\vspace*{-1mm}
%
\inferrule[\textsc{(read)}]{
\sigma \astep{\rlab(o,x,v)} \sigma'  \\\\ \msg{x}{v}{\_}{t}{\mrel}\in\mem\\\\
%\tup{\tcom,\gsco} {\astep{\rlab:o,x,t,\mrel}}  \tup{\tcom',\gsco} 
{\tcom} {\astep{\rlab:o,x,t,\mrel}}  {\tcom'} 
}{\tup{\tup{\sigma, \tcom, \lprom},\gsco, \mem} \astep{} \tup{\tup{\sigma', \tcom', \lprom}, \gsco, \mem}}
\and
\inferrule[\textsc{(write)}]{
\sigma \astep{\wlab(o,x,v)} \sigma' \\\\
o = \ra \implies \forall m' \in \lprom(x).~ m'.\lmrel = \bot  \\\\
m=\msg{x}{v}{\_}{t}{\mrel}\\\\
%\tup{\lprom, \mem}  \astep{\text{write}(o,m)} \tup{\lprom', \mem'} \\\\
\tup{\lprom, \mem}  \astep{m} \tup{\lprom', \mem'} \\\\
%\tup{\tcom,\gsco} {\astep{\wlab:o,x,t,\bot,\mrel}} \tup{\tcom',\gsco'} 
{\tcom} {\astep{\wlab:o,x,t,\bot,\mrel}} {\tcom'} 
}{\tup{\tup{\sigma, \tcom, \lprom},\gsco, \mem} \astep{} \tup{\tup{\sigma', \tcom', \lprom'}, \gsco, \mem'}}
\and
\inferrule[\textsc{(update)}]{
\sigma \astep{\ulab(o_\lr,o_\lw,x,v_\lr,v_\lw)} \sigma'  \\\\
o_\lw = \ra \implies \forall m' \in \lprom(x).~ m'.\lmrel = \bot  \\\\
 {\begin{array}{@{}c@{}l@{}}
& \msg{x}
{\makebox[0pt][l]{$v_\lr$}\phantom{v_\lw}}
{\makebox[0pt][l]{$\_$}\phantom{t_\lr}}
{\makebox[0pt][l]{$t_\lr$}\phantom{t_\lw}}
{\makebox[0pt][l]{$\mrel_\lr$}\phantom{\mrel_\lw}} \in\mem\\
 m_\lw=&\msg{x}{v_\lw}{t_\lr}{t_\lw}{\mrel_\lw}
\end{array}}\\\\
%\tup{\lprom, \mem}  \astep{\text{write}(o_\lw,m_\lw)} \tup{\lprom', \mem'} \\\\
\tup{\lprom, \mem}  \astep{m_\lw} \tup{\lprom', \mem'} \\\\
%\tup{\tcom,\gsco} {\astep{\rlab: o_\lr,x,t_\lr,\mrel_\lr}} %\tup{\tcom_0,\gsco} {\astep{\wlab:o_\lw,x,t_\lw,\mrel_\lr,\mrel_\lw}} \tup{\tcom',\gsco'}
%{\astep{\wlab:o_\lw,x,t_\lw,\mrel_\lr,\mrel_\lw}} \tup{\tcom',\gsco'}
{\tcom} {\astep{\rlab: o_\lr,x,t_\lr,\mrel_\lr}} %\tup{\tcom_0,\gsco} {\astep{\wlab:o_\lw,x,t_\lw,\mrel_\lr,\mrel_\lw}} \tup{\tcom',\gsco'}
{\astep{\wlab:o_\lw,x,t_\lw,\mrel_\lr,\mrel_\lw}} {\tcom'}
}{\tup{\tup{\sigma,\tcom,\lprom}, \gsco,\mem} \astep{} \tup{\tup{\sigma',\tcom',\lprom'}, \gsco,\mem'}}
%\and
%\inferrule[\textsc{(fence)}]{
%\tlab\in\{\acqo,\relo,\sco\}\\ \sigma \astep{\flab_\tlab} \sigma' \\ \tup{\tcom,\gsco}  \astep{\flab_\tlab} \tup{\tcom',\gsco'} \\\\
%\tlab \neq \acqo \implies \forall m' \in \lprom.~ m'.\lmrel = \bot
%}{\tup{\tup{\sigma, \tcom, \lprom}, \gsco,\mem} \astep{} \tup{\tup{\sigma', \tcom', \lprom}, \gsco',\mem}}
%
\\\vspace*{-1mm}
%
\inferrule[\textsc{(acq-fence)}]{
\sigma \astep{\flab_\acqo} \sigma' \\ 
\cur'=\acq
}{\tup{\tup{\sigma, \tup{\cur,\acq,\rel}, \lprom}, \gsco,\mem} \astep{} \\\\\quad \tup{\tup{\sigma', \tup{\tup{\cur',\acq,\rel}, \lprom}, \gsco,\mem}}}
\and
\inferrule[\textsc{(rel-fence)}]{
\sigma \astep{\flab_\relo} \sigma' \\ 
\rel'= \lambda \_.\cur \\\\
\forall m \in \lprom.~ m.\lmrel = \bot
}{\tup{\tup{\sigma, \tup{\cur,\acq,\rel}, \lprom}, \gsco,\mem} \astep{} \\\\\quad \tup{\tup{\sigma', \tup{\tup{\cur,\acq,\rel'}, \lprom}, \gsco,\mem}}}
\and
\inferrule[\textsc{(sc-fence)}]{
\sigma \astep{\flab_\sco} \sigma' \\\\ 
\tup{\tcom,\gsco}  \astep{\flab_\sco} \tup{\tcom',\gsco'} \\\\
\forall m \in \lprom.~ m.\lmrel = \bot
}{\tup{\tup{\sigma, \tcom, \lprom}, \gsco, \mem} \astep{} \\\\\quad\tup{\tup{\sigma', \tcom', \lprom}, \gsco', \mem}}
\and
\inferrule[\textsc{(system call)}]{
\sigma \astep{\syscallt} \sigma' \\\\
\tup{\tcom,\gsco}  \astep{\flab_\sco} \tup{\tcom',\gsco'}\\\\
\forall m \in \lprom.~ m.\lmrel = \bot
}{\tup{\tup{\sigma, \tcom, \lprom}, \gsco, \mem} \astep{\syscallt} \\\\\quad\tup{\tup{\sigma', \tcom', \lprom}, \gsco', \mem}}
%
\\\vspace*{-1mm}
%
\inferrule[\textsc{(silent)}]{
\sigma \astep{\silentt} \sigma'
}{\tup{\tup{\sigma, \tcom, \lprom}, \gsco, \mem} \astep{} \tup{\tup{\sigma', \tcom, \lprom}, \gsco, \mem}}
\and
%\inferrule[\textsc{(promise)}]{
%\tup{\lprom, \mem} \astep{\text{promise}(m)} \tup{\lprom', \mem'} \\\\ 
%m.\lmrel \tmin \mem'
%}{\tup{\tup{\sigma, \tcom, \lprom},\gsco,\mem} \astep{} \tup{\tup{\sigma, \tcom, \lprom'},\gsco,\mem'}}
\inferrule[\textsc{(promise)}]{
\hookleftarrow \in \setofz{\insertadd, \insertsplit, \insertupdate} \\ \lprom'= \lprom \hookleftarrow m\\\\
 \mem'= \mem \hookleftarrow m \\
 m.\lmrel \tmin \mem'
}{\tup{\tup{\sigma, \tcom, \lprom},\gsco,\mem} \astep{} \tup{\tup{\sigma, \tcom, \lprom'},\gsco,\mem'}}
\and
\inferrule[\bf \textsc{(machine step)}]{
\tup{\gts(i),\gsco,\mem} \astep{}^* \tup{\lts',\gsco',\mem'}  \\\\
\tup{\lts',\gsco',\mem'} \astep{e} \tup{\lts'',\gsco'',\mem''}  \\\\
\tup{\lts'',\gsco'',\mem''} \text{ is consistent}
}{\tup{\gts,\gsco,\mem} \astep{e} \tup{\gts[i\mapsto \lts''],\gsco'',\mem''}}
\end{mathpar}
\caption{Full operational semantics.}
\label{fig:full-opsem-a}
\end{figure*}}


\section{Full Model}
\label{sec:full}

In this section, we extend the basic model of
\Cref{sec:relaxed}-\ref{sec:updates} to handle all the
features of the C++ concurrency model except SC accesses and consume reads.
% \footnote{
% Consume reads are widely considered a somewhat
%   premature aspect of the C++ standard and are implemented as acquire reads in mainstream compilers.}


\subsection{Release/Acquire Synchronization}
\label{sec:relacq}

\paragraph{Release/Acquire Fences}

A crucial feature of the C++ model is the ability for threads to synchronize using memory fences or stronger kinds of atomic accesses.
Consider the message-passing test case:
\begin{equation}\label{eq:MPfences}\tag{MP+fences}
\inarrII{ x:=1 ;\\ \fencerel ;\\ y:=1; }{  a:=y; \comment{1} \\ \fenceacq;\\ b:=x; \nocomment{0} }
\end{equation}
The release fence between the writes, together with the acquire fence
between the reads, prevents the weak behavior of the example (\ie that
of returning $a=1$ and $b=0$).  Roughly speaking, the C++ model
forbids this behavior by requiring that whenever a read before an
acquire fence reads from a write after a release fence, the two fences
synchronize, which in turn means that any write that happens-before the release
fence must be visible to any read that happens-after the acquire fence.  So, if $T_2$
reads $y=1$, then after the acquire fence it \emph{must} read $x=1$.

% \jeehoon{Actually, any \emph{read/write} before the release fence is
%   visible to any \emph{read/write} after the acquire fence.  I think
%   specialization to only read (and write) here is a little bit
%   confusing.} \ori{I added `happens-' to make it precise}

To implement this semantics, we extend our model in two ways.

First, we refine each thread's view.  Rather than having a single view
of which messages it has observed, a thread now has three views:
$\tcom=\tup{\cur,\acq,\rel}$.  We denote by $\tcom.\lcur$,
$\tcom.\lacq$ and $\tcom.\lrel$ the components of a thread view
$\tcom$.  A thread's \emph{current view}, $\lcur$, is as before: it records
which messages the thread has observed and restricts which messages a
read may return and a write may create.  Its \emph{release view}, $\lrel$,
records what the thread's $\lcur$ view \emph{was} at the point of its
last release fence.
% which are intuitively the messages that have necessarily happened
% before the current point.
Dually, its \emph{acquire view}, $\lacq$, records what the thread's $\lcur$
view \emph{will become} if it performs an acquire fence.
Consequently, the views are related as follows:
$\lrel \leq \lcur \leq \lacq$.

Second, we extend write messages to record a \emph{message
  view}~$\mrel$, which records the release view of the writing thread
at the time the write occurred (updated to include the write itself).
Thus, a message now takes the form $m=\msg{x}{v}{f}{t}{\mrel}$, where
$\mrel(x) = t$.  We write $m.\lmrel$ for the message view of $m$.
% Second, we extend write messages to additionally record the release
% view of the writing thread at the point when the write occurred.
% Thus, a message now takes the form $m=\msg{x}{v}{f}{t}{\mrel}$, where
% $x,v,f,t$ are as before, and $\mrel$ is the \emph{message view},
% satisfying $\mrel(x)\leq t$.  We write $m.\lmrel$ to mean the message
% view $\mrel$ of $m$.

% The message view, $R$,  in order to maintain the acquire view, joins the thread's $\lcur$ view
% together with the views of the messages that the thread has observed,

During execution of relaxed accesses, a thread's views drift apart.
When a thread reads a message, it incorporates the message's view into
the thread's $\lacq$ view, but not into its $\lcur$ or $\lrel$ views.
When a thread writes a message, it uses the thread's $\lrel$ view as
the basis for the message's view, but only incorporates the message
itself into the thread's $\lcur$ and $\lacq$ views, not its $\lrel$
view.
% Reading a message increases the thread's $\lacq$ to include both the message
% and its view, but increases the thread's $\lcur$ only to include the message,
% not its view.  Writing a message increases the thread's $\lcur$ to include
% the message written, but does not increase 
% but   increase the thread's $\lcur$ to include the message read, and
% increase the thread's $\lacq$ to include both the message and its
% view.  
% Writes increase all the views to include the message's view,
% which itself is derived from the thread's $\lrel$ view.

Fence commands bring these diverging views closer to one another.
Specifically, an acquire fence increases the thread's $\lcur$ view to
match its $\lacq$ view, thereby ensuring that the thread is up to date
with respect to views of all the messages read before the fence.
Symmetrically, a release fence increases the thread's $\lrel$ view to
match its $\lcur$ view, thereby ensuring that the views of all
messages the thread writes after the release fence will contain the
messages observed before the fence.

Returning to the \ref{eq:MPfences} program, suppose that 
$T_1$ emitted messages $\msg{x}{1}{\_}{t_x}{\_}$
and $\msg{y}{1}{\_}{t_y}{\mrel_y}$. 
Then, $T_1$'s $\lcur$ view before the release fence 
maps $x$ to $t_x$. % $[\vtup{x}{t_x},\vtup{y}{\ts{0}}]$. 
The fence then updates $T_1$'s $\lrel$ view to match
its $\lcur$ view, so that the message view accompanying the subsequent write to $y$ will map $x$ to $t_x$ as well.  % $\mrel_y=[\vtup{x}{t_x},\vtup{y}{t_y}]$.  
(Without the release fence, 
this message view would map $x$ to $\ts{0}$.) %$\mrel_y=[\vtup{x}{\ts{0}},\vtup{y}{\ts{0}}]$.)
 On $T_2$'s side, the read of
$y=1$ updates $T_2$'s $\lcur$ view to
$[\vtup{x}{\ts{0}},\vtup{y}{t_y}]$, and its $\lacq$ view to
$[\vtup{x}{t_x},\vtup{y}{t_y}]$.  The acquire fence then updates
$T_2$'s $\lcur$ view to match its $\lacq$ view, and hence the
subsequent read of $x$ must see the $x:=1$ write.  If either the
release or the acquire fence were missing, then $T_2$'s $\lcur$ view
at the read of $x$ would have been $[\vtup{x}{\ts{0}},\vtup{y}{t_y}]$,
allowing it to read $x=0$.

\paragraph{Interaction with Promises}

Promises (like every other message) now carry a view,
and threads reading a promise are subject to the same constraints as if they were reading a normal message.
In particular, after reading a promise and performing an acquire fence, a thread can only read messages
with timestamp greater than or equal to the view carried in the promise message.
In order to avoid cases where execution gets stuck, we must ensure that \emph{some} message can be read
for every location. Thus we require that the view attached to a promise message includes only timestamps of messages that exist in memory at the time the promise is made.
%Formally, we impose a \emph{closedness} condition on the memory at every step of the execution:
%we write $\mrel\in\mem$ if for every $x\in\Loc$, $\mrel(x)=m.\lto$ for some $m\in\mem$ with $m.\lloc=x$,
%and say that a memory $\mem$ is \emph{closed} if $m.\lmrel\in\mem$ for every $m\in\mem$.

Going back to \ref{eq:MPfences}, note that $T_1$ cannot promise $y:=1$
before performing $x:=1$. Indeed, because of the release fence, the view in the $y:=1$ message must include the message
that will be produced for the $x:=1$ assignment, but at the beginning the only message for $x$ in memory is the initial one (at timestamp $\ts{0}$).
Hence, release fences effectively serve also as barriers for promises.
We find it convenient to explicitly require this in our semantics: whenever a release fence is performed,
the set of promises of the executing thread must be empty.
% %
% \ori{is it only convenient?}
% %
% \gil{We explicitly need it because one may be able to satisfy a
%   promise over a release fence by raising undefined behavior. We have
%   an example for this.}
% %
This may seem restrictive, but note that the main reason for introducing promises was to allow read-write reorderings,
as in the \ref{eq:LB} example of \Cref{sec:oota}.
If there is a release fence in between the read and write, then the reordering is no longer possible, and thus our motivation for promising the write is void.

\paragraph{Release/Acquire Accesses}

In addition to release and acquire fences, C++ offers a more fine-grained way
of achieving synchronization, via \emph{acquire reads} and \emph{release writes}.
Intuitively speaking, an acquire read is a relaxed read followed by an acquire fence,
whereas a release write is a release fence followed by a relaxed write,
with the restriction that these fences induce synchronization \emph{only} on the location of the access.
For example, in the following program,\footnote{In this and in following code snippets, we annotate non-relaxed accesses with their access mode; 
all non-annotated accesses are relaxed.} 
only the second thread synchronizes with the first one.
$$
\inarrIII{ x:=1 ;\\ y_{\relw}:=1;\\ z:=1; }
         { a:=y_{\acqr}; \comment{1} \\ b:=x; \nocomment{0} }
         { c:=z_{\acqr}; \comment{1} \\ d:=x; \comment{0} }
$$
Hence, $b$ must get the value $1$, while $d$ may get $0$.




To model these accesses, we treat the $\lrel$ view of each thread not as a single view,
but rather as one separate view per location, 
recording the thread's current view at the latest release fence or release write to that location.
Thus, when a thread performs a release write to location $x$, we update its release view of $x$ to match its $\cur$ view,
while a release fence effects this update on the release views of \emph{all} locations.
Then, a write to $x$ (either release or relaxed) will use the release view of $x$ (newly updated, if it is a release write) to form the view of the write message, and an acquire read
will incorporate the message's view into the reading thread's current view.
% On the other side, an acquire read updates the thread's current view to include the message's view.

In the example above, at the end of $T_1$'s execution,
its thread view has $\lrel(y)=[\vtup{x}{t_x},\vtup{y}{t_y},\vtup{z}{\ts{0}}]$,
whereas $\lrel(z)=[\vtup{x}{\ts{0}},\vtup{y}{\ts{0}},\vtup{z}{t_z}]$.
As a result, the $y_{\acqr}$ read increases $T_2$'s $\lcur$ view to $[\vtup{x}{t_x},\vtup{y}{t_y},\vtup{z}{\ts{0}}]$, 
which forces it to then read $x=1$,
whereas the $z_{\acqr}$ read increases $T_3$'s $\lcur$ view to $[\vtup{x}{\ts{0}},\vtup{y}{\ts{0}},\vtup{z}{t_z}]$, 
which allows it to later read $x=0$.

\paragraph{Release Sequences}

Using the per-location release views, 
we can straightforwardly handle C++-style \emph{release sequences}
(following the  definition of release sequences given in \cite{c11comp}).
%.\footnote{%
%We follow the corrected definition of release sequences in \cite{c11comp}.}
%%because the definition in the C++ standard is flawed. See \citet{c11comp} for details.}
In C++, an acquire read synchronizes with a release write $w$ to $x$ not only if it reads from $w$
but also if it reads from a write in $w$'s release sequence.
The release sequence of $w$ is inductively defined to include all the same-thread writes/updates to $x$ after $w$, 
as well as all updates reading from an event in the release sequence of $w$.
For example, in the following program, the $y_{\acqr}$ synchronizes with the $y_{\relw}:=1$ 
because it reads from the $\fai{y,1}$, which in turn reads from the $y:=2$.
$$
\inarrIII{ x:=1 ;\\ y_{\relw}:=1;\\ y:=2; }
         { \fai{y,1}; }
         { a:=y_{\acqr}; \comment{3} \\ b:=x; \nocomment{0} }
$$
Our operational semantics already handles the case of reading from a later write of the same thread,
because the thread's release view for $y$ is included in the message's view.
To handle the updates that read from elements of the release sequence, 
we insist that the view of the write message of an update must incorporate the view of the read message of the update.
Thus, in this example, the views of all the $y$ messages contain $\vtup{x}{t_x}$,
and hence $T_3$ must read $x=1$.

\paragraph{Promises Over Release/Acquire Accesses}
We finally point out another delicate issue related to the interaction between
promises and release/acquire accesses.
Consider the following variants of the \ref{eq:LB} example:
\begin{center}
\hspace*{-2mm}
\begin{minipage}{.5\columnwidth}
\begin{equation}\label{eq:LBrel}\tag{LBr}
\inarrII{ a:=x; \nocomment{1}\hspace*{-2pt}\\ y_{\relw}:=1; }{\hspace*{-2pt} x:=y; }
\end{equation}
\end{minipage}
\hfill
\begin{minipage}{.5\columnwidth}
\begin{equation}\label{eq:LBacq}\tag{LBa}
\inarrII{ a:=x_{\acqr}; \comment{1}\hspace*{-2pt} \\ y:=1; }{\hspace*{-2pt} x:=y; }
\end{equation}
\end{minipage}
\end{center}
%
In the first variant (\ref{eq:LBrel}), the promise of $y_{\relw}:=1$ should be forbidden
for the same reason that a promise over a release fence is forbidden,
and hence the specified behavior is disallowed.
We note that this behavior is possible under the C++ model, 
but is not possible under the usual compilation of release writes
to Power and ARM (using a \texttt{lwsync}/\texttt{dmb\_sy} fence in the first thread).\footnote{%
Moreover, we observe that even the C++ model forbids this outcome, 
if we additionally make the read of $y$ in the second thread into a consume read 
(which is supposed to be compiled exactly as a relaxed read, but preserving syntactic dependencies).}
More generally, our model forbids promises over release writes to the same location.
%\ori{maybe just add: ``More generally, our model forbids promises over release accesses to the same location, and forbids the specified behavior in \ref{eq:LBrel}''.
%I actually dislike our $\neq$ notation, program comments are needed just to specify a behavior,
%so we can later say ``the specified behavior in ...''. When we put $\neq$ we choose 
%arbitrarily where to put it...}
%(release writes to location $x$ can only be performed by a thread $T$ when 
%the set of promises of $T$ do not include a promise for $x$).
%This restriction simplifies one part in the DRF proof.

In the second variant (\ref{eq:LBacq}), we allow the promise of $y:=1$ and thus the $a=1$ outcome.
The reason is that we want to enable optimizations that 
result in the elimination of an acquire read and thus remove the reordering constraints of the acquire.
Consider, for example, the following program transformation:
\begin{equation}\label{eq:acq_merge}\tag{LBa$'$}
\inarrII{a:=x;\comment{2}\\ y:=1;\\b := y_{\acqr};\\y:=2;}{x:=y;} \quad \leadsto \quad
\inarrII{y:=1;\\b := 1;\\y:=2;\\ a:=x;\comment{2}}{x:=y;}
\end{equation}
which may in effect reorder the $y:=2$ write before the $a:=x$ read even though there is an acquire read in between
(by first replacing $y_{\acqr}$ with $1$ and then reordering $a:=x$ past both writes to $y$).
Thus, our semantics has to allow promises over acquire actions.
Note that there is no need to do so for release writes, because release writes cannot simply be eliminated in this way.
%They may be only merged with adjacent writes to the same location.\ori{and why is this not a problem? to continue here}


%\begin{remark}
%\label{explicit_np}
%Our model explicitly forbids promises over release accesses to the same location
%(release writes to location $x$ can only be performed by a thread $T$ when 
%the set of promises of $T$ do not include a promise for $x$).
%This restriction allows us to simplify one part in the DRF proof.
%Nevertheless, we believe it is redundant as an explicit condition, 
%and we leave its removal to future work.
%\end{remark}

\subsection{Sequentially Consistent (SC) Fences}
\label{sec:sc}

We now extend the model with sequentially consistent (SC) fences, 
whose purpose is to allow the programmer
to enforce strong ordering guarantees among memory accesses. 
In particular, full
sequential consistency is restored if an SC fence is  
placed between every two shared memory accesses of
a program.\footnote{In this regard, our semantics is stronger than the C++ model~\cite{Batty:2011},
which fails to validate this basic property, and follows Lahav~\etal~\cite{sra,repairing-sc} instead.} 
%and is stronger than the original C++
%model~\cite{Batty:2011} as well as that of Batty~\etal~\cite{Batty:2016}, both of which fail to validate
%this basic property. \derek{Cite our PLDI TR?  We need something to back up
%claim of others' failure.}}

To handle SC fences, we extend our machine state with a \emph{global} timemap $\gsco$,
which records the latest messages written by any thread before an SC fence.
When a thread $T$ executes an SC fence, in addition to the effect of both an acquire and a release fence, 
$T$ increases both its $\lcur$ view and the global timemap to the maximum of the two.
Consider the following variant of the \ref{eq:SB} example:
\begin{equation}\label{eq:SBfences}\tag{SB+fences}
\inarrII{ x:=1 ;\\ \fencesc;\\ a:=y; \comment{0} }{  y:=1; \\ \fencesc;\\ b:=x; \nocomment{0} }
\end{equation}
Here, the current views of the two threads
just before their SC fences are $[\vtup{x}{t_x},\vtup{y}{\ts{0}}]$ and 
$[\vtup{x}{\ts{0}},\vtup{y}{t_y}]$, respectively, while the global view is
$[\vtup{x}{\ts{0}},\vtup{y}{\ts{0}}]$.  If the fence of $T_1$ is executed
first, it will update $\gsco$ to $[\vtup{x}{t_x},\vtup{y}{\ts{0}}]$.
So, when the fence of $T_2$ is executed, both its $\lcur$ view and
$\gsco$ become $[\vtup{x}{t_x},\vtup{y}{t_y}]$, from which point
onwards $T_2$ \emph{must} read $x=1$.


%\paragraph{SC Accesses } %Ori: if i remove the space here, my latex goes crazy 
%
%Besides SC fences, C++ also provides the notion of SC accesses, which can be
%thought of roughly as release/acquire accesses bundled with a ``location-specific SC fence''. 
%In the case of the \ref{eq:SB} example, ruling out the weak behavior without
%fences requires one to make \emph{all four} accesses be SC accesses.
%
%To handle these accesses, we need more machinery.
%First, we extend the notion of a view $\view$---both message views $\mrel$ and the
%three component views of a thread $(\lcur,\lacq,\lrel)$---from being a single timemap to a pair of timemaps:
%one ``normal'' one ($\view.\lrw$) as before, and one for SC accesses ($\view.\lsco$), which must be higher in timestamp order than $\view.\lrw$ and which serves to restrict the possible timestamps available to future SC accesses.
%% The normal timemap is interpreted as before: it records the most recent writes that have so
%% far been observed.
%% The SC timemap is generally larger ($\view.\lrw\leq \view.\lsco$), and it serves to record
%% the  and in the case of an SC write, 
%% it additionally records all the SC writes (as well as the normal writes before SC fences) that were executed before it.
%% We likewise extend the three component views of a thread $(\lrel,\lcur,\lacq)$ to consist of such pairs of timemaps.
%When performing an SC read,
%a thread has to respect its $\lcur.\lsco$ timemap,
%and update its views similarly to an acquire read.
%When performing an SC write, again, a thread has to respect its $\lcur.\lsco$ timemap,
%and update its views similarly to a release write;
%but it also has to respect the global SC timemap $\gsco$ (by picking a timestamp greater than 
%the one recorded in $\gsco$), and include $\gsco$ in its new timemap $\lcur.\lsco$.
%% Together, we have that the $\lrw$ timemaps in the thread views are interpreted as before, whereas the $\lsco$ timemaps also record
%% the latest access that was executed (perhaps by another thread) before an SC access by the current point ($\lcur$),
%% or before the last release fence ($\lrel$) or will be seen by the next acquire fence ($\lacq$).
%
%% \ori{maybe a bit better- but still unreadable....}

\subsection{``Plain'' Non-Synchronizing Accesses}
\label{sec:plain}

Both C++ and Java provide some form of \emph{non-synchronizing} accesses, \ie accesses that are meant to be used only for non-racy data accesses
(C++'s non-atomic accesses and Java's normal accesses).
Such accesses can never achieve synchronization, even together with fences.
Consequently, compilers are free to reorder non-synchronizing reads across acquire fences, and to reorder release fences across non-synchronizing writes.
These non-synchronizing accesses, which we refer to as \emph{plain} accesses, are easily supported in our model.
The difference from relaxed accesses is simple:
a plain read from a message $m$ should not incorporate $m.\lmrel$ into the thread's $\lacq$ view;
and a message $m$ produced by a plain write should only carry the $\ts{0}$-view (\ie $\bot$ in the lattice of views).
Moreover, plain writes can be promised even beyond a release fence or a release write to the same location.

Besides the reordering mentioned above, compilers can (and do) utilize further the assumption that some accesses 
are intended to be non-racy.
Indeed, assuming two non-racy reads, a compiler may reorder them even if they are reading \emph{the same} location.
In a broader context, it may pave the way to further optimizations 
(\eg a compiler may prefer to unconditionally optimize $a:=x; b:=*p;$ $c:=x$ to $b:=*p; a:=x; c:=a$, 
without the burden of analyzing whether the pointer $p$ points to $x$ or not).
Since we followed C++'s assumption of full per-location coherence for our relaxed accesses, 
the reordering of two reads from the same location is unsound for them. 
Concretely, consider the following example:
$$
\inarrII{x:=1;\\ x:=2;}{a:=x;\comment{2}\\ b:=x;\comment{1}} \quad \leadsto \quad
\inarrII{x:=1;\\ x:=2;}{b:=x;\comment{1}\\ a:=x;\comment{2}}
$$
The target program obviously allows the specified behavior, while the source does not.
Fortunately, it is not hard to adapt our plain accesses to provide only partial per-location coherence
(in C++11 terms, dropping ``coherence-RR'' for plain accesses), consequently allowing this reordering.
The idea is to extend the notion of a view $\view$---both message views $\mrel$ and the
three component views of a thread $(\lcur,\lacq,\lrel)$---from being a single timemap to a pair of timemaps:
a ``normal'' one ($\view.\lrw$) as before, and one for plain accesses ($\view.\lur$).
The $\view.\lur$ timemap is generally smaller than the normal timemap ($\view.\lur \leq \view.\lrw$),
and restricts the possible timestamps available to plain reads.
A plain read from a message $m$ with location $x$ and time $t$ only consults this new timemap,
checking that $\lcur.\lur(x) \leq t$, and only updates $\lcur.\lrw(x)$ to include $t$.
 A plain write, on the other hand, cannot pick a timestamp smaller than $\lcur.\lrw(x)$ 
(since we do maintain the other coherence properties besides ``coherence-RR'').

Importantly, we do not exploit ``catch-fire'' semantics (\`a la C++) to accommodate our plain accesses,
but rather give a well-defined semantics to arbitrary racy programs.
In addition, we note that it is easy to decouple the two weaknesses of plain accesses compared to relaxed ones,
by introducing a middle access mode that allows synchronization
(together with release and acquire fences) but supports only partial per-location coherence.  

\begin{remark}
Our model handles only hardware-atomic memory accesses.
To handle non-atomic reads/writes, such as Java double and C struct accesses, our semantics could be extended 
by introducing ``garbage values'' (LLVM-style undefined values~\cite{llvm}) as in \cite{Soham17}.
% \derek{I never understood this paragraph.  Is the point that we are not handling mixed-size accesses?}
% \viktor{Updated.}
\end{remark}

\subsection{System Calls}
\label{sec:sys_call}
For the purpose of  defining the behaviors of programs (as needed to prove soundness of transformations),
we augment our language and semantics with system calls labeled with ``$\syscallt$''.
These are operations that  are visible to an external observer (\eg printing statements). 
For simplicity, we assume that these take one value (input or output),
and more importantly, that they do not access the memory, and serve as the strongest barrier for reordering.
Thus, we simply model system calls as SC fences.

\subsection{Modifying Existing Promises}

So far, our model does not allow promises, once made, to be changed.
However, our full model does allow two forms of promise adjustment,
both of which are defined in such a way that threads that have already
read from the promised message are unaffected.

% our full model does allow two kinds of modification of promised messages.
% Obviously, both are defined in a way that other threads, that already read the promised message,
% could have continued the same had they read from the modified promise.

\paragraph{Split}

The first form of promise adjustment is \emph{splitting}.
Consider the following example:
$$
\inarrII{ a := x ;  \comment{2} \\  \ite{a = 2}{\fai{y,1}; \fai{y,1};}{\fai{y,2};} }{ x := y; }
$$
%Here, $y\texttt{+=}2$ denotes a fetch-and-add instruction that increments $y$ by $2$.
We find it natural to allow the specified behavior, as it can be obtained by benign compiler optimizations:
first $\fai{y,1}; \fai{y,1}$ can be merged to $\fai{y,2}$, and then the whole if-then-else statement can be replaced by $\fai{y,2}$.
Nevertheless, the model described so far forbids this behavior.
Indeed, clearly, an execution obtaining this behavior must start with $T_1$ promising $\updmsg{y}{2}{f}{t}$.
Since this promise must be certified under an arbitrary future memory, $T_1$ must pick $f=\ts{0}$ (or else, it cannot fulfill its promise for a 
memory that includes, say, $\updmsg{y}{42}{\ts{0}}{\ts{5}}$). Then, $T_2$ can read the promise and add a message of the form $\updmsg{x}{2}{\_}{t_x}$ to the memory.
Now, $T_1$ would like to read this message.
However, if it does so, it will not be able to fulfill its promise $\updmsg{y}{2}{\ts{0}}{t}$, simply because there is no available timestamp
interval in which it can put the first $y=1$ message.
To solve this, we allow threads to split their own promises in two pieces, keeping the original promise with the same $m.\lto$
value.
For the example above, $T_1$ could proceed by splitting its promise $\updmsg{y}{2}{\ts{0}}{t}$ into 
$\updmsg{y}{1}{\ts{0}}{t/\ts{2}}$  and $\updmsg{y}{2}{t/\ts{2}}{t}$, reading the message $\updmsg{x}{2}{\_}{t_x}$ and fulfilling both promises.


% \ori{also: splits allows us to have simple simulation relation for proving soundness of merging that eliminate writes.}


\paragraph{Lower}

The second form of promise adjustment is \emph{lowering} of the promised message's view.
Note that by promising a message carrying a high view, a thread places \emph{more} restrictions on the readers of that promise.
Thus, changing the view of a promise $m$ to a view $\mrel'\leq m.\lmrel$ can never cause any harm.
Technically, including this option simplifies our simulation arguments used to prove the soundness of program transformations,
by allowing us to have a simpler simulation relation between the source and target memories.
%\ori{is this correct?}  \jeehoon{Yes; if the promised messages of the source and the target are fulfilled with different released views, lowering is essential in the current simulation technique.  It may be a future work to remove lowering in the model and yet prove optimizations.}
More generally speaking, it allows us to prove and use the following natural property:
if all the views included in some machine state $\mconf$ (in its memory's messages and its threads' views)
are less than or equal to all views in another machine state $\mconf'$, 
then every behavior of $\mconf'$ is also a behavior of $\mconf$.

\subsection{Formal Model}
\label{sec:full-formal}

\jeehoon{figure should be adjusted.}
\figfull

Finally, we formally present our full model,
combining and making precise all the ideas outlined above.
The model employs three modes for memory accesses, naturally ordered as follows:
\[\pln ~~\sqsubset~~\rlx~~\sqsubset~~\ra\]
%\[\pln ~~\sqsubset~~\rlx~~\sqsubset~~\ra~~\sqsubset~~\sco\]
We use $o$ as a metavariable for access mode.
The programming language is modeled by a transition system whose 
transition labels (see \Cref{sec:relaxed-formal}) are:
``$\silentt$'' for local transitions;
${\rlab(o,x,v)}$  for reads;
${\wlab(o,x,v)}$  for writes;
${\ulab(o_\lr,o_\lw,x,v_\lr,v_\lw)}$  for updates;
${\flab_\acqo},{\flab_\relo},{\flab_\sco}$ for fences;
and  ${\syscallt}$ for system calls.
Note that updates have two access modes, one for the read and one for the write;
and that only fences may have the $\sco$ mode.

\paragraph{View}
A \emph{view} is a pair $\view=\tup{T_\lur,T_\lrw}$ of timemaps (see \Cref{sec:relaxed-formal})
satisfying $T_\lur \leq T_\lrw$.
We denote by $\view.\lur$ and $\view.\lrw$ the components of $\view$.
$\View$ denotes the set of all views.
%A \emph{view} is a pair $\view=\tup{T_\lur,T_\lrw,T_\lsco}$ of timemaps (see \Cref{sec:relaxed-formal})
%satisfying $T_\lur \leq T_\lrw \leq T_\lsco$.
%We denote by $\view.\lur$, $\view.\lrw$ and $\view.\lsco$ the components of $\view$.
%$\View$ denotes the set of all views.

\paragraph{Messages}
A \emph{message} $m$ is a tuple $\msg{x}{v}{f}{t}{\mrel}$,
where $x\in\Loc$, $v\in\Val$, $f,t\in\Time$, and $\mrel\in \View$,
such that $f<t$ or $f=t=\ts{0}$, and $\mrel.\lrw(x) = t$ or $\mrel = \bot$.
We denote by $m.\lloc$, $m.\lval$, $m.\lfrom$, $m.\lto$, and $m.\lmrel$ the components of $m$.

\paragraph{Memory}
A \emph{memory} is a (nonempty) pairwise disjoint finite set of messages
(see \Cref{sec:updates} for def.\ of disjointness).
A memory $\mem$ supports the following insertions of a message $m=\msg{x}{v}{f}{t}{\mrel}$  :
\begin{itemize}
\item The \emph{additive insertion}, denoted by $\mem \insertadd m$,
 is only defined if $\setofz{m}\disj \mem$,  in which case it is given by
 $\setofz{m} \cup \mem$.
\item The \emph{splitting insertion}, denoted by $\mem \insertsplit m$,
is only defined if there exists $m'=\msg{x}{v'}{f}{t'}{\mrel'}$ with $t<t'$ in $\mem$, 
in which case it is given by $\mem{\setminus}\setofz{m'} \cup \setofz{m, \msg{x}{v'}{t}{t'}{\mrel'}}$.
\item The \emph{lowering insertion}, denoted by $\mem \insertupdate m$,
is only defined if there exists $m'=\msg{x}{v}{f}{t}{\mrel'}$ with $\mrel \leq \mrel'$ in $\mem$,
in which case it is given by $\mem{\setminus}\setofz{m'} \cup \setofz{m}$.
\end{itemize}
We write $M(x)$ for the sub-memory $\setof{m \in M \suchthat m.\lloc = x}$.
%We write ${\mem \astep{} \mem'}$ if $\mem'\in\{\mem\insertadd m,\mem \insertsplit m,\mem \insertupdate m\}$
%for some message $m$.
%Note that $\mem \astep{} \mem$ (by def.\ of $\mem \insertupdate m$).

\paragraph{Closed Memory}
Given a timemap $T$ and a memory $\mem$, we write $T\tmin\mem$ 
if, for every $x\in\Loc$, we have $T(x)= m.\lto$ for some 
$ m\in\mem$ with $m.\lloc = x$.
For a view $\view$, we write $\view \tmin \mem$ if $T \tmin \mem$ for each component timemap $T$ of $\view$.
A memory $\mem$ is \emph{closed} if $m.\lmrel \tmin \mem$ for every $m\in \mem$.

\paragraph{Future Memory}
For memories $\mem,\mem'$, we write ${\mem \astep{} \mem'}$ if 
$\mem'\in\{\mem\insertadd m,\mem \insertsplit m,\mem \insertupdate m\}$
for some message $m$, and $\mem'$ is closed.
We say that $\mem'$ is a \emph{future memory} of $\mem$ w.r.t. a memory $\lprom$,
if $\lprom \subseteq \mem'$ and  $\mem \astep{}^* \mem'$.

\paragraph{Threads}
A \emph{thread view} is a triple $\tcom=\tup{\cur,\acq,\rel}$,
where $\cur,\acq\in\View$ and $\rel\in\Loc\to\View$
satisfying $rel(x) \leq cur \leq acq$ for all $x\in\Loc$.
We denote by $\tcom.\lcur$, $\tcom.\lacq$, and $\tcom.\lrel$ the components of $\tcom$.
A \emph{thread state} is a triple $\lts=\tup{\sigma,\tcom,\lprom}$
defined just as in~\Cref{sec:relaxed-formal} except with a thread view $\tcom$ instead of
a single timemap ($\sigma$ is a local state and $\lprom$ is a memory).
We denote by $\lts.\lstate$, $\lts.\lview$, and $\lts.\lprmem$ the components of $\lts$.

\paragraph{Thread Configuration Steps}
A \emph{thread configuration} is a triple $\tup{\lts,\gsco,\mem}$,
where $\lts$ is a thread state, $\gsco$ is a timemap (the global SC timemap), and $\mem$ is a memory.

\Cref{fig:full-opsem-a} presents the full list of thread configuration steps.
To avoid repetition, we use the additional 
rules \textsc{read-helper}, \textsc{write-helper}, and \textsc{sc-fence-helper}.
These employ several helpful notations:
$\bot$ and $\sqcup$ denote the natural bottom elements and join operations for timemaps and for views 
(pointwise extensions of the initial timestamp $\ts{0}$ and the $\sqcup$---\ie $\max$---operation on timestamps);
$\{\vtup{x}{t}\}$ denotes the timemap assigning $t$ to $x$ and $\ts{0}$ to other locations;
and $\iteb{cond}{X}$ is defined to be $X$ if $cond$ holds, 
and $\bot$ otherwise.

%Promise steps use the \textsc{memory:promise} rule,
%that inserts a message to the global memory as well as to the set of promises (that is always a subset of the global memory).
%This insertion may be of any of the three insertion kinds mentioned earlier.

The write and the update steps cover two cases: a fresh write to memory (\textsc{memory:new}) and a fulfillment of an outstanding promise (\textsc{memory:fulfill}).
The latter allows to split the promise or lower its view before its fulfillment 
(note that when $m\in \lprom \suq \mem$, we have  
$\lprom=\lprom \insertupdate m$
and 
$\mem=\mem \insertupdate m$ by def.\ of $\insertupdate$).
%Both rules check that there are no outstanding promises for the same location when the write has mode $\sqsupseteq \ra$ (see \remarkref{explicit_np}).

\paragraph{Consistency}
A {thread configuration} $\tup{\lts,\gsco,\mem}$ is called \emph{consistent} if
for every future memory $\omem$ of $\mem$ w.r.t. $\lts.\lprmem$
and every timemap $\ogsco$ with ${\gsco \leq \ogsco \tmin \omem}$,
there exist $\lts'$, $\gsco'$, $\mem'$ such that:
\[
\tup{\lts,\ogsco,\omem} \astep{}^* \tup{\lts',\gsco',\mem'}
~~\land~~ \lts'.\lprmem = \emptyset
\]

\paragraph{Machine and Behaviors}

A \emph{machine state} is a triple $\mconf = \tup{\gts,\gsco,\mem}$ consisting of a
function $\gts$ assigning a thread state to every thread, an SC timemap $\gsco$, and a memory $\mem$.  
The initial state $\mconf^0$ (for a given program) consists of 
the function $\gts^0$ mapping each thread $i$ to its initial state $\sigma_i^0$,
the zero thread view (all timestamps in all timemaps are $\ts{0}$),
and an empty set of promises;
the zero timemap $\gsco^0$;
and the initial memory $\mem^0$ consisting of one message $\msg{x}{0}{\ts{0}}{\ts{0}}{\bot}$ for each location $x$.
The machine step is defined by the last rule in \Cref{fig:full-opsem-a}.
The variable $e$ in the final thread configuration step can either be a usual step ($e$ is empty),
or denote a system call ($e=\syscallt$).

To define the set of behaviors of a program $\prog$ 
(namely, what is externally observable during $\prog$'s executions), 
we use the system calls that $\prog$'s executions perform.
More precisely, every execution induces a sequence of system calls 
(each includes a specific value for input/output),
and the set of behaviors of $\prog$ is taken to be the set of all system call sequences induced by executions of $\prog$.

\paragraph{Promise-Free Machine}
In several of our results below, we make use of the fragment of our
model obtained by revoking the ability to make promises (\ie omitting
the \textsc{promise} rule).  We call this the \emph{promise-free
  machine}.




%%% Local Variables:
%%% mode: latex
%%% TeX-master: "main"
%%% TeX-command-extra-options: "-shell-escape"
%%% End:
