\section{Introduction}
\label{sec:sepcomp:introduction}

There was a key dimension in which the verification of CompCert, as well as that of the other
existing verified compilers, is not realistic---namely that, for simplicity, it only establishes the
correctness of \emph{whole-program} compilation: if CompCert is used to compile a self-contained C
program consisting of a single file, then the output of CompCert preserves the semantics of that
program.  But clearly this does not correspond to what many clients of a verified compiler would
expect.  For example, it is often essential in practice to be able to compile a client module
separately from the many standard libraries it depends on, yet be assured that linking the resulting
binaries together will result in an executable that preserves the semantics of the linked source
modules.  Furthermore, it is commonplace for different modules in a program to be compiled with
different sets of optimization passes turned on.  Although the CompCert compiler does indeed support
such forms of separate compilation, its verification statement says nothing about them.

The technical reason for this limitation is that it makes it possible for the CompCert verification
to be carried out straightforwardly using closed simulations: simulations between closed (\ie
self-contained, executable) programs.  For each pass of the compiler, the output of the pass is
shown to simulate the input of the pass, assuming and preserving whatever invariant the verifier
wishes to impose on the relation between the states of the input and output.  Working with closed
simulations simplifies life in two ways: $(1)$ the simulation proof for each pass can rely on
whatever state invariant it chooses, independent of what invariants are used in other passes, and
$(2)$ these independent simulations collectively imply the end-to-end correctness of the whole
compiler (this is sometimes called ``vertical compositionality'').  However, closed simulations are
by definition simulations over whole program states, thus seemingly confining their applicability to
the verification of whole-program compilation.

There has consequently been a great deal of work in the past several years attempting to prove
compiler correctness \emph{without} the whole-program restriction.  Indeed, it turns out that even
specifying, let alone verifying, when separate compilation is ``correct''---often referred to as
\emph{compositional compiler correctness}~\cite{benton+:icfp09}---is non-trivial, and has sparked a
variety of interesting proposals involving technically sophisticated techniques, such as Kripke
logical relations~\cite{hur+:popl11}, multi-language semantics~\cite{perconti+:esop14}, and
parametric simulations~\cite{hur+:popl12,neis+:icfp15}.  All these approaches aim to achieve a
highly flexible form of compositionality, guaranteeing for instance that the results of multiple
different verified compilers can be correctly linked together, and that it is safe to link those
results with hand-written assembly code.

It seems, however, that achieving such flexibility comes at the
expense of significant complication to the proof method.  For example,
Perconti and Ahmed's approach~\cite{perconti+:esop14} involves constructing logical
relations over a multi-language semantics encompassing all languages
used in a compiler, a considerable departure from CompCert-style
verification.  Neis~\etal's method~\cite{neis+:icfp15} employs a novel notion of
``parametric inter-language simulations (PILS)'', whose proof of the
aforementioned ``vertical compositionality'' is highly
involved~\cite{rts-trans}, whereas for closed simulations it is trivial.
% the recently proposed method of PILS~\cite{??} supports
% highly flexible compositionality
% proposed PILS~\cite{??}) offer an even stronger notion of
% compositionality, supporting compilers with wildly different source
% and target languages, but they are also much more technically
% sophisticated than ours, so porting existing compiler verifications to
% use them would require a substantial effort.
In the context of CompCert, Stewart~\etal\ recently developed
Compositional CompCert~\cite{stewart+:popl2015}, a re-engineering of
CompCert to support verified separate compilation, along with the
ability to link C modules with assembly modules.  Their approach,
however, relies on a novel notion of ``structured simulation'' on top
of a multi-language ``interaction semantics''~\cite{beringer+:esop14},
which is different enough from the closed simulations employed in the
CompCert verification that it required significant changes and
extensions to \mbox{the original proofs}.
% \todo{gil: the CompCert compiler is intact but the majority of the correctness verification was rewritten.}

In this paper, we ask the question: If we aim somewhat lower, can we
do a lot better?  That is, if we pursue a more restricted notion of
compositional correctness than prior work has done, can we develop a
much simpler, more lightweight proof method that enables us to
\emph{reuse} existing verifications of whole-program compilation as
much as possible instead of rewriting them?
% to , one which can 
% compiler verifications from whole-program  separate compilation, so that
% previous proof effort can be \emph{reused} as much as possible?

% Is there a way to support
% verification of separate compilation in a simpler, more lightweight
% manner, so that existing verifications of whole-program compilation
% can be reused as much as possible?

Indeed, we can.  In particular, we restrict attention here to
verifying separate compilation for a \emph{single} compiler.  Our goal
is to establish that, when different modules in a program are compiled
separately by the \emph{same} verified compiler, the linking of the
resulting assembly modules preserves the semantics of the linking of
the original source modules.  Within the scope of this more modest but
still important goal, we develop simple and effective techniques for
verifying two levels of compositional correctness:
%
% First, we develop an extremely simple
% technique for adapting a verification 
%
% We present several simple
% but effective techniques for establishing that, when different modules
% in a program are compiled separately by the \emph{same} verified
% compiler, the linking of the resulting assembly modules preserves the
% semantics of the linking of the original source modules.  Within the
% scope of this more limited goal, we actually consider two levels of
% compositionality.
%
% Our answer is: ``Yes, if one restricts attention to compositional
% correctness with respect to a \emph{single} compiler.''  That is,
% in this paper, rather than attempting to achieve maximally compositional
% compiler correctness as previous work has done, we focus on a more
% limited goal: proving that, when different modules in a program
% are compiled separately by the \emph{same} verified compiler, the
% linking of the resulting assembly modules preserves the semantics
% of the linking of the original source modules.  Within the scope
% of this more limited goal, we actually consider two levels of
% compositionality.
%  -- but the degree of simplicity depends
% (inversely) on the desired level of compositionality.''  In
% particular, we develop several verification techniques, at increasing
% levels of sophistication and supporting increasing levels of
% compositionality:
\begin{itemize}
\item \textbf{Compositional Correctness Level A:} 
Correctness of compilation is preserved when linking modules
that were compiled with \emph{the same exact compiler}.
% Linking separately
%   compiled modules is correct, assuming that all modules in the
%   program have been compiled by \emph{exactly the same compiler}.
\item \textbf{Compositional Correctness Level B:} Correctness of
  compilation is preserved when linking modules that were compiled
  with the same compiler, \emph{but possibly with different
    optimizations}---\ie different modules may be compiled with
  different optimization passes turned on.
% Linking separately
%   compiled modules is correct, assuming that all modules in the
%   program have been compiled by the same compiler, 
 % We first consider the modest goal of supporting
  % verified separate compilation for a \emph{single, fixed} compiler.
  % Given a compiler like CompCert, which supports separate compilation
  % but only verifies the correctness of single-file compilation, we
  % present an extremely simple way to adapt its correctness proof into
  % one that verifies separate compilation as well.
% Given a compiler like CompCert, which supports separate compilation
% but whose verification only accounts for single-file compilation, we
% present an extremely simple way to adapt its correctness proof into
% one that verifies 
% \item \textbf{Level 2: Separate compilation in the presence of
%     optimization flags.}  Still restricting attention to verification
%   of a single compiler, we strengthen the compositionality afforded by
%   our Level 1 method in order to enable different modules in a program
%   to be compiled with different optimization passes turned on.  This
%   strengthening requires a small amount of additional proof effort.
%   % Like our Level 1 method, the Level 2 method is very simple and
%   % should be
%   % applicable to a wide variety of compilers.
% \item \textbf{Level 3: Inter-compiler and inter-language linking.}
%   Lastly, we consider an even stronger form of compositional
%   correctness, by which different modules in a program may be compiled
%   with \emph{different} compilers, and which allows linking between
%   different languages (\eg C and assembly).  With Level 3
%   compositionality, we are targeting essentially the same goal as in
%   prior work by Stewart~\etal\ on Compositional CompCert~\cite{stewart+:popl2015},
%   and to do so we exploit their multi-language ``interaction''
%   semantics, which introduces some (but not much) complexity.
%   \mbox{Unlike} their approach, however, applying ours to CompCert
%   requires only minor changes to its existing verification (see below).
\end{itemize}
Level B correctness is stronger than Level A, and correspondingly
requires somewhat (but not much) more work to prove.

% closed-simulation
%   (whole-program) compiler verification into one that allows different
%   modules in a program to be compiled separately (and each with
%   potentially different optimization passes turned on).  We applied
%   this technique to CompCert and, in so doing, uncovered an (easily
%   fixable) bug in CompCert's existing (unverified) separate
%   compilation facility.  The adaptation took only X person-months and
%   required changes to only Y\% lines of proof.

% Like
%   their work, our Level 3 method is restricted in its applicability
%   because it assumes the source and target languages share a common
%   memory model (true for CompCert, but not for, say, ML compilers).

% In prior work (\eg \cite{hur+:popl11,stewart+:popl15,neis+:icfp15}),
% it seems to be mostly taken for granted that such a ``contextual''
% approach is only applicable to compositional compiler correctness when
% the source and target languages of the compiler are the same.  Our
% observation here is that this is not the case: one can gainfully
% employ contextual reasoning to simplify compositional compiler
% verification even if, as is usually the case, the source and target
% languages of the compiler are different.\footnote{Perconti and
%   Ahmed~\cite{perconti+:esop14} also employ contextual reasoning in
%   their account of compositional compiler correctness, but in a way
%   that depends fundamentally on the use of multi-language semantics in
%   order to embed the source and target languages of the compiler in
%   one common language.  Our Levels 1 and 2 show that contextual
%   reasoning is useful even without multi-language semantics.  For
%   further comparison, see \Cref{sec:related}.}

% That said, there is a tradeoff here: in return for the simplicity of
% our approach, its scope is somewhat narrower than that of some prior
% approaches.  In particular, Levels 1 and 2 only support proving the
% correctness of linking modules compiled by a \emph{single} compiler.
% Level 3 supports a stronger form of compositional correctness, but
% (like Stewart~\etal's work) it is restricted in its applicability
% because it assumes the compiler's source and target languages share a
% common memory model (true for CompCert but not for, say, ML
% compilers).  In contrast, some prior approaches (such as the recently
% proposed PILS~\cite{??}) offer an even stronger notion of
% compositionality, supporting compilers with wildly different source
% and target languages, but they are also much more technically
% sophisticated than ours, so porting existing compiler verifications to
% use them would require a substantial effort.  
% We believe the
% lightweight nature of our techniques---and the consequent ease of
% adapting existing verifications to use them---make them an attractive
% option for compiler verifiers when applicable.

The key idea spanning both levels is to formulate a compositional correctness statement about a
single \emph{module} $M$ in terms of a ``contextual'' correctness statement about how $M$ behaves
when linked with arbitrary other modules to form a complete program.  The latter has the advantage
of being a statement about closed (whole) programs, and as such, we can prove it using the kind of
simple, closed-simulation-style proofs employed in traditional verifications of whole-program
compilation.  This in turn enables us to significantly reuse existing compiler proofs.  We believe
the lightweight nature of our techniques---and the consequent ease of adapting existing
verifications to use them---will make them a highly attractive option for compiler verifiers.

We demonstrate the effectiveness of our techniques by applying them to CompCert 2.4.  In less than
two person-months total, we adapted the existing CompCert verification to support both Level A and
Level B compositional correctness, and much of that time was spent trying to understand the original
CompCert proof.  \textbf{The result of this effort is the first verification of separate compilation
  for the full CompCert compiler.}  Our verification (available online~\cite{kang-phd-thesis-web})
is mostly the same as the original CompCert verification, and is only 2\% (for Level A) or 3\% (for
Level B) larger than the original verification in terms of lines of Coq.  Furthermore, in the course
of doing so, we uncovered two bugs in CompCert: one in an invalid axiom, and one (in CompCert's
``value analysis'') that was outside the scope of CompCert's original verification because it only
showed up in the presence of separate compilation.  These have been confirmed and subsequently fixed
in CompCert 2.5.

% \footnote{The latest version of CompCert is 2.5.  It was released on June 12, 2015, after we had
%   already completed our verification of \sepcomp{}.  See \Cref{sec:related} for discussion of what
%   would be involved in porting \sepcomp{} to handle the new features of CompCert 2.5.}

% \todo{gil: say that CompCert 2.4 was the
%   latest version at the time we started this project.  The latest
%   version CompCert 2.5 was just released on 12 Jun 2015.}

% \footnote{The prior work on
%   Compositional CompCert~\cite{stewart+:popl2015} does not handle the full
%   \mbox{CompCert} compiler, and it employs a bespoke form of linking
%   that does not match the standard assembly-level linking.  See
%   \Cref{sec:related} for a detailed comparison.}

% The prior work on Compositional CompCert~\cite{stewart+:popl2015} only verified 10 of
% CompCert's 19 passes, and it established correctness w.r.t.\ a
% different notion of assembly-level linking than the one used by
% CompCert.

The remainder of this chapter is structured as follows.  In \Cref{sec:sepcomp:overview}, we give the
overview of our new techniques, which are presented in detail in the subsequent sections in the
context of our CompCert adaptation.  In \Cref{sec:sepcomp:discussion}, we conclude with a comparison
to related work and discussion on the generality and the impact of our techniques.

% We applied this technique to CompCert, verifying correctness of its
% separate compilation facility in a few weeks' time.  Along the way, we
% uncovered (and fixed) a bug in CompCert that was only triggered by
% separate compilation (and hence outside the scope of the original
% CompCert verification).

% In the case of CompCert, it performs six optimizations at the level of
% its RTL intermediate language (\ie translations from RTL to RTL).
% With Level 2 compositionality, we can verify the correctness of
% linking modules that were each compiled with arbitrary different
% subsets of these optimizations.






% , which aim at ``full
% compositionality'', \eg linking results of different verified compilers
% whose source and target languages are wildly different.  


% because the classic
% correctness notion of \emph{contextual equivalence/refinement} only
% applies to program transformations \emph{within} a single language,
% not between two different languages.  Our observation here is that,
% if one is willing to restrict the scope of one's compositional correctness
% statement---in ways that still provide a useful result---then contextual
% reasoning can be usefully integrated 




% This is
% intuitively similar to, but different in detail from, how one
% typically defines \emph{contextual refinement}, a common correctness
% property for program transformations within a \emph{single} language.
% Here, effectively, we show how one can reduce contextual reasoning is useful
% even when proving compositional correctness for compilers whose source
% and target languages are \emph{not} the same.  Since contextual reasoning
% defines compositional One key benefit this affords
% us is the ability to reuse existing closed-simulation proofs with very
% few modifications.


% Our approach, which is geared toward minimizing modification of existing

% Levels 1 and 2 provide more limited forms of compositional correctness
% for a single compiler, but the techniques are simple and general
% enough to be applicable to a wide variety of compilers.  In contrast,
% Level 3 supports a more flexible form of compositional correctness,
% but (like Stewart~\etal's work) it is restricted in its applicability
% because it assumes the compiler's source and target languages share a
% common memory model (true for CompCert but not for, say, ML compilers).


% In
% other words, we reduce the compositional correctness problem to a
% whole-program correctness program where a piece of the whole program
% is unknown.

% that, in order
% to simplify the porting of existing whole-program
% compiler verifications and reuse their proofs to the greatest extent
% possible, we want to  we aim to formulate a compositional correctness statement
% about a single module $M$ in terms of a ``contextual''
% as ``contextual'' statements about 




% demonstrate that---at least for a compiler like
% CompCert, which has been deliberately designed to be compatible with
% separate compilation, and whose source and target languages share a
% common memory model---there is a much simpler way to
% ``compositionalize'' its verification, \ie transforming its
% correctness guarantee for whole-program compilation into a correctness
% guarantee for separate compilation.


















% is realistic in the sense that it perfor This landmark
% project resulted in the first realistic, verified compiler.  A
% verified compiler is a compiler whose output has been formally
% guaranteed to preserve the semantics of its input.  Such a compiler
% can be used to establish that program analyses and proofs, conducted
% at the level of its source language, carry over to the lower-level
% code that actually runs on the machine.  As such, verified compilers
% have the potential to play an indispensable role in end-to-end
% verified software.  Moreover, compiler verification is


% therefore play a critical role in
% end-to-end software verification: role in end-to-end verified
% software.  Such a compiler is thus an indispensable building block for
% end-to-end verified software: it enables one to transfer program
% analyses and proofs, conducted at the level of the compiler's source
% language, to the code that actually runs at the machine level.  that
% program analyses and proofs, performed at the level of the compiler's
% source language, remain valid for the lower-level code that the
% compiler produces.  Such a guarantee Although compiler verification is
% a As recent projects like CompCert~\cite{??} and CakeML~\cite{??}
% demonstrate, interactive proof assistants like Coq and HOL4 have
% evolved into highly effective tools for the verification of realistic
% optimizing compilers.  Suchthey are increasingly becoming
% indispensable building blocks for end-to-end verified software.
% % of executable machine code
% % are increasingly becoming
% % indispensable building blocks for end-to-end verified software, because they make
% % it possible 
% % ensure that program
% % analyses and verifications performed at the source-language level do
% % in fact carry over to the code that actually executes at the machine
% % level.  As a result, they are increasingly becoming indispensable
% % building blocks for end-to-end verified software.

% Nonetheless, developing such a verified compiler from scratch tends to
% be an arduous multi-person-year enterprise.  Thus, for simplicity,
% these verification efforts have restricted attention to the
% correctness of \emph{whole-program} compilation---\eg if CompCert is
% used to compile a \emph{complete} C program, then the output of
% CompCert preserves the semantics of that complete program.
% Unfortunately, this does not correspond to what many clients of a
% verified compiler likely want or expect.  For example, it is essential
% in practice to be able to compile a client module separately from the
% many standard libraries it depends on, yet be assured that linking the
% resulting binaries together will result in an executable that
% preserves the semantics of the linked source modules.  Furthermore,
% the


%%% Local Variables:
%%% mode: latex
%%% TeX-master: "main"
%%% TeX-command-extra-options: "-shell-escape"
%%% End:
