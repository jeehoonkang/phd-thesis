\section{Related Work}\label{sec:related}

%\paragraph{The ``Out of Thin Air'' Problem}

There have been many proposals for solving the ``out of thin air''
problem.  Several of them have come with proofs of DRF guarantees, but
ours is the first to come with formal (and machine-checked) validation
of a wide range of essential local transformations
(\Cref{sec:transformations}) concerning a broad spectrum of features
from the C/C++ model.

%\ori{also say that we are the first to do that with the full set of features}

The first major attempt to solve the ``out of thin air'' problem was by the Java memory model (JMM)~\cite{jmm} (see also~\cite{Lochbihler2014TOPLAS}).
The JMM intended to validate all the compiler optimizations that Java compilers and just-in-time compilers might perform,
but its formal definition failed to validate them~\cite{sevcik:jmm}.
Subsequent fixes were proposed to the model, which improved the set of enabled optimizations, 
but still falling short of what actual Java compilers were performing.
%\derek{Citation?}

Interestingly, an early glimpse of our idea of promises may be seen in
version~1.0 of the JMM~\cite{jmm-1.0}, which describes a form of
``prescient store actions'' (\S{17.8}).  However, their description is
very brief and vague, and the feature was removed for JSR
133~\cite{jsr133}.

%\item 
To resolve some of the problems with the JMM definition,
\mbox{Jagadeesan~\etal}~\cite{Jagadeesan2010} proposed an operational
model following quite closely the intended behavior of the JMM, but
employing the notion of a \emph{speculation}.  Speculations are
similar to our notion of promises, but unlike promises they are not
certified thread-locally: whereas we model interference conservatively
by quantifying over all future memories during certification, they
model interference from other threads more precisely by executing
multiple threads together during certification.  We believe our
conservative approach is sufficient for justifying standard compiler
optimizations, which are typically thread-local, and moreover it
simplifies the presentation of the semantics and the development of
the metatheory because it avoids the need for nested certifications.

%  In this way, we allow very robust speculations only so
% that we can achieve better meta theories and easily verify compiler
% optimizations.

Jagadeesan~\etal's model satisfies the standard DRF theorem, as well
as a DRF theorem saying that speculations are unnecessary for programs
without read-write races.  They also develop a simulation proof
technique, with which they verify three optimizations: write-write
reordering, roach-motel reordering, and read-after-read elimination.
We have applied our simulation method to a much wider variety of
optimizations, and our proofs are machine-checked in Coq.  They also
do not provide any compilation correctness results, and their model
omits release-acquire accesses, updates, and fences.

% If it does not, then
% it cannot be compiled to Power and ARM without additional fences for
% plain accesses, which would render the model impractical.  \ori{Riely:
%   ``Upon discussion with Radha, I recalled that we left these results
%   out the paper not because they were difficult, but because they were
%   trivial.  Our model most certainly does validate these
%   reorderings.''}  Their model omits updates and fences.
% %
% \gil{It is worth giving an example showing that DRF does not hold in
%   the absence of re-certification if we allow release/acquire
%   accesses.  Related to this, we can say that Jagadeesan does not
%   support release/acquire accesses and has no notion of
%   re-certification.}


% \blue{From Gil's email: In Jagadeesan's, they model interference from other threads precisely by allowing executing all threads together during validation. But, in our model, we model such interference conservatively by quantifying over all possible future memories regardless of what the other threads are. This way we allow very robust speculations only so that we can achieve better meta theories and easily verify compiler optimizations.}

% %\item
% \blue{ Say something about the C/C++ standards. }

% %\item 
% \blue{ mention also \cite{Boehm2014}, it is a way to avoid thin-air...}

%\item 
More recently, Jeffrey and Riely~\cite{jeffrey:2016} presented a weak memory model based on event structures.
Their model admits a standard DRF theorem, but does not fully allow the reordering of independent memory accesses, 
and thus cannot be compiled to Power/ARM without extra fences.
The paper suggests an idea about how to fix the model to support such reorderings, 
but it is not known whether the suggested fixed model avoids OOTA behaviors.
Relating to our work, their model seems to be ``promising'' reads (instead of writes) and 
restricting the quantification over possible futures to only those that could arise from further execution of the current program.  The model only supports relaxed accesses and locks.

% \derek{Can we say anything more damning about Jeffrey and Riely?}
% \ori{i edited a bit, I suggest to remove the last sentence, but only bcs i dont understand it...}

% \blue{These two facts do not allow them to have "thread local" semantics.}

Pichon-Pharabod and Sewell~\cite{Pichon} introduced an event structure model with both a normal reduction rule, which executes an initial event of the event structure, and special reduction rules that mimic the effect of standard compiler optimizations on the event structure.
These optimization rules include a rather complex rule for non-thread-local optimizations that can declare a whole branch of the event structure unreachable.
The paper does not present any formal results about the model.
It is worth noting that the model does not support the weak behavior of the \ref{eq:ARMweak} program and thus may not be compiled to ARM without additional fences.  The model only handles relaxed and non-atomic accesses and locks.

Podkopaev~\etal~\cite{sergey:2016} proposed an operational model covering a large subset of the features of the C/C++ model.  They provide many litmus tests to demonstrate the suitability
of their model, but do not prove any formal results about it.
Their model ensures per-location coherence in a very similar way to our model: using timestamps.
In order to handle read-write reorderings, they allow reads to return symbolic values, which are then evaluated at a later point in time when their value is actually needed.
This approach gives the expected behaviors to the \ref{eq:LB} and \ref{eq:LBdep} programs, 
and may be extended with a set of \emph{syntactic} symbolic simplification rules to also give the expected result to the \ref{eq:LBfalsedep} program.
It seems, however, very difficult to extend this approach to enable code motion optimizations, where some common code is pulled out of two branches of a conditional. 
What makes code motion more challenging is that the common code may become apparent only after some earlier transformations, 
like for example the $y:=1$ assignment in the following code:
$$
\inarrII{ a := x; \comment{1} \\ \ite{a = 1}{y := a;}{(z := 1; y := z;)} }{ x := y; } 
$$
Our model allows the annotated behavior of the program above,
precisely because our promises are semantic in nature and thus avoid the brittle
tracking of syntactic data dependencies.

%\blue{The idea of "viewfronts" (for release/acquire syncronization) is a well-known technique in dynamic data-race detection (see, e.g., \cite{Pozniansky})}

Zhang and Feng~\cite{Zhang2016} suggested an operational model for Java accesses in which threads may re-execute some memory events.
The model admits a standard DRF theorem, and its replay mechanism enables it to support local transformations.
However, to avoid OOTA, this mechanism is limited by its tracking of syntactic dependencies between instructions,
and thus it fails to validate behaviors resulting from trace-preserving transformations like the one above.
%\ori{i am not 100\% sure here... try to make it vaguer?}

Other proposals for language-level memory models have tried not to solve the OOTA problem, but to avoid it,
by introducing stronger models where read-write reordering is not allowed.
For example, \sevcik~\etal~\cite{compcerttso} and Demange~\etal~\cite{bmm} proposed using TSO as the memory model for C and Java, respectively. 
These proposals may be reasonable compromises if the only target machines of interest also follow the TSO model,
but are prohibitively expensive on weaker architectures, such as Power and ARM,
because enforcing TSO on those machines requires essentially as many fences as enforcing SC.
In a similar line of work, Lahav~\etal~\cite{sra} introduced a strengthening of the release/acquire fragment of the C/C++ memory model, which they called SRA, together with an operational semantics for SRA\@.
Compiling SRA to Power and ARM is cheaper than TSO, but still requires some fences before or after every shared variable access, and may thus not be suitable for performance-critical code.

Another approach is to simply allow OOTA behaviors.
This was the approach taken by Batty~\etal's formalization of C/C++~\cite{Batty:2011}, and by the OpenCL model~\cite{opencl-model},
as well as by Crary and Sullivan~\cite{Crary2015}, who introduced a more fine-grained specification of the orders that the model is supposed to preserve.
All of these models allow the weak behavior of the \ref{eq:LBdep} example, thereby invalidating standard reasoning principles and DRF theorems.

Finally, Norris and Demsky~\cite{CDSchecker} presented a tool
that exhaustively enumerates the behaviors of concurrent C/C++ programs.
To account for speculative reads, the tool may establish ``promised
future values'', which a load can read from, and which must eventually
be written by a future store.  Norris and Demsky's promises look
superficially quite similar to ours, but their purpose is to support
practical model checking of C/C++ programs, not to change the semantics
of the language, so the paper does not present any formal model or
metatheory of promises.



%%% Local Variables:
%%% mode: latex
%%% TeX-master: "main"
%%% TeX-command-extra-options: "-shell-escape"
%%% End:
