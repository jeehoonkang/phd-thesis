\section{Discussion}
\label{sec:sepcomp:discussion}

% - comparison with compositional compcert and tahina's work

% - comparison with amal's work and pils

% - general applicability of the method (compcert 2.5, cakeml)

% Why we didn't do CompCert 2.5?
% Due to ``static variables and functions''.
% We need to handle local name space in the def of syntatic linking.
% Since there are several ways to define it, we leave it to Xavier.

\subsection{Related Work on Compositional Compiler Correctness}

\paragraph{Compositional CompCert}

The most closely related work to ours is Stewart~\etal's Compositional
CompCert~\cite{stewart+:popl2015}, which establishes compositional correctness for a
significant subset of CompCert 2.1.  Their approach builds on their
previous work on \emph{interaction semantics}~\cite{beringer+:esop14}, which defines
linking between modules in a (somewhat) language-independent way.
(The languages in question must share a common memory model, as is the
case for C, assembly, and all the intermediate languages of CompCert.)
Essentially, interaction semantics enables Compositional CompCert to
reduce the compiler verification problem to one of contextual
refinement.  The output of the compiler is proven to refine the source
under an arbitrary ``semantic context'', which may consist of a
linking of C and assembly modules.

On the one hand, Compositional CompCert is targeting a more ambitious
goal than separate compilation.  Their approach inherently supports the
possibility of linking results of multiple different compilers, as
well as the ability to link C modules with hand-coded assembly
modules, both of which are beyond the scope of our techniques.

On the other hand, as we explained in the introduction, the modesty of
our goal is quite deliberate---it enables our separate compilation verification
to use a considerably more lightweight approach than they do.  For
example, they report that their porting of CompCert passes to verify
compositional correctness took approximately 10 person-months and led
to more than a doubling in the size of each pass.  In contrast, our
porting took less than 2 person-months, and even if we look at just
the passes alone (ignoring the metatheory), they are on average less
than 2\% (for Level A) or 4\% (for Level B) larger than the original
CompCert passes.  The new metatheory backing up our proof technique is
also smaller than the corresponding new metatheory of Compositional
CompCert by roughly a factor of 7.

One reason for this, we believe, is that the \emph{structured
  simulations} employed in the Compositional CompCert proof require
all passes in the compiler to be verified using a common
``memory-injection'' invariant.  In contrast, in our verification, we were
able to essentially reuse the invariants from the original CompCert
proof, which are different for different passes.  Another potential
reason concerns the treatment of inter-module vs.\ external function
calls.  In CompCert, external functions are assumed to satisfy a
number of axioms, which are used in the verification to establish that
external function calls preserve the simulation relation in each pass.
In Compositional CompCert, inter-module function calls are treated the
same as external function calls, and as a result Stewart~\etal's
verification must additionally establish that functions compiled by
the compiler satisfy the external function axioms.  In contrast, in our verification,
we reduce the problem of verifying separate compilation to
that of verifying correctness of compilation for a whole
(multi-module) program.  Thus, for us, inter-module function calls are
\emph{not} treated as external calls---they merely shift control to
another part of the program---and there is no need for us to prove
that CompCert-compiled functions satisfy the external function axioms.

There are two other points of difference worth mentioning.  First, we
have ported over the entire CompCert 2.4 compiler, from C to assembly
(including the x86, Power, and ARM backends), whereas Compositional
CompCert omitted the front-end of CompCert 2.1 (from C to Clight),
along with three of its RTL-level optimizations (CSE, constant
propagation, and inlining).  Second, due to its use of interaction
semantics, Compositional CompCert employs a bespoke ``semantic''
notion of linking (even for linking assembly files), which has not yet
been related to the standard notion of syntactic linking (see
\Cref{sec:adaptconstprop}) that we employ in our
verification.  Syntactic linking corresponds much more closely to the
physical linking of machine code implemented by the \code{gcc} linker
that CompCert uses by default.  Proving this is outside the scope of
our verification effort, however, since CompCert only verifies
correctness of compilation down to the assembly level, not to the
machine-code level where linking is actually performed.

\paragraph{CompCertX}

Concurrently with Stewart~\etal's work, Ramananandro~\etal~\cite{ramananandro+:cpp2015}
developed a different approach to compositional correctness for
CompCert.  While similar in many ways to the approach of
Stewart~\etal, the \emph{compositional semantics} of
Ramananandro~\etal\ defines linking so that the linking of
assembly-level modules boils down to syntactic linking (essentially,
concatenation), as it does in our verification.  On the other hand, they
have only used their approach to compositionally verify a few passes
of CompCert.

Also concurrently with the above work, Gu~\etal~\cite{gu+:popl2015} have
developed CompCertX, a compositional adaptation of CompCert
specifically targeted for use in the compositional verification of OS
kernels.  CompCertX supports linking of compiled C code with
hand-written assembly code, and ports over all passes of CompCert, but
the source and target of the compiler are different from CompCert's.
In particular, the ClightX source language of CompCertX does not
generally allow functions to modify other functions' stack frames and
thus does not support stack-allocated data structures.  Although this
restriction is not problematic for their particular application to OS
kernel verification, it means that, as the authors themselves note,
``CompCertX can not be regarded as a full featured separate compiler
for CompCert.''

\paragraph{Multi-lanugage Semantics}

There have been several approaches proposed for compositional
correctness of compilers other than CompCert as well, although these
all involve \emph{de novo} verifications rather than ports of existing
whole-program compiler verifications.

Perconti and Ahmed~\cite{perconti+:esop14}  present an approach to compositional
compiler correctness for ML-like languages.  They use multi-language
semantics to combine all the languages of a compiler into one joint
language, with wrapping operations to coerce values in one language to
values of the appropriate type in the other languages.  Like
Compositional CompCert, their approach recasts the compiler
verification problem as a contextual refinement problem, except that
they model contexts syntactically rather than semantically and use
\emph{logical relations} as a proof technique for establishing
contextual refinement rather than structured simulations.  It is
difficult to gauge how well this approach scales as a practical
compiler verification method because it has not yet been mechanized or
applied to a full-blown compiler.

\paragraph{Cito}

Wang~\etal~\cite{cito} develop a compositional verification framework
for a compiler for Cito, a simple C-like language~\cite{cito}.  Their
approach is quite different from the others in that it characterizes
the compiler verification problem in terms of Hoare-style
specifications of assembly code.  Currently, the approach is limited
in its ability to talk about preservation of termination-sensitive
properties, and as its verification statement is so different from the
traditional end-to-end behavioral refinement result established by a
compiler like \mbox{CompCert}, it is not clear how Wang~\etal's method
could reuse existing CompCert-style verifications.

\paragraph{Parametric Inter-Language Simulation}

Most recently, Neis~\etal~\cite{neis+:icfp15} present \emph{parametric
  inter-language simulations (PILS)}, which they use to
compositionally verify Pilsner, a compiler for an ML-like core
language, in Coq.  PILS build on earlier work by Hur~\etal\ on logical
relations~\cite{benton+:icfp09,hur+:popl11} and parametric
bisimulations (aka relation transition
systems)~\cite{hur+:popl12,rts-trans}.  Pilsner supports a very strong
compositional correctness statement, but it also required a major
verification effort, involving several person-years of work and 55K
lines of Coq.


\subsection{Generality of Our Techniques}

In this paper, we have presented several simple techniques for
establishing Level A and Level B compositional correctness, and
demonstrated the feasibility and effectiveness of these techniques by
porting a major landmark compiler verification (CompCert 2.4) to
support compositional correctness without much difficulty at all.  We
hope the almost embarrassingly simple nature of these techniques will
encourage future compiler verifiers to consider proving at least a
restricted form of compositional correctness for their compilers from
the start.

One may wonder, however, how general our techniques are.  Are they
dependent on particular aspects of CompCert 2.4?  Can they be applied
to other verified compilers?  For C or for other languages?  Given the
landscape of compiler verification, dotted as it is with unique and
majestic mountains, it is difficult to give sweepingly general answers
to these questions.  But we can say the following.

We believe our techniques should be applicable to the most recent
version of CompCert (2.5), but a necessary first step is to determine
the appropriate notion of syntactic linking.  CompCert 2.5 introduces
support for \code{static} variables (which in C means variables that are
only locally visible within a single file).  The presence of
\code{static} variables means that the simple canonical definition of
syntactic linking we have used no longer works and must be revisited.
Assuming a reasonable definition can be found, as we expect, we do not
foresee any problems adapting our techniques to handle it.
% In private communication, Xavier Leroy has informed us that he is
% interested in developing such a redefinition of syntactic linking so
% that he can potentially incorporate (some of) our techniques into a
% future version of CompCert.

Regarding the application to compilers for other languages, we can
only speculate, but we also do not foresee any fundamental problems.
For instance, CakeML~\cite{cakeml} is a verified compiler for a
significant subset of Standard ML, implemented in HOL4.  CakeML's
end-to-end verification statement concerns the correctness of an x86
implementation of an interactive SML read-eval-print loop.  In that
sense, the verification is not exactly ``whole-program'' because new
code can be compiled and added to a global database interactively.
But it also does not support true separate compilation in the sense
that our verification does because modules cannot be compiled independently
of the other modules they depend on.

We believe in principle it should be possible to use our techniques to
adapt CakeML to verify correctness of separate compilation, because
CakeML is not an optimizing compiler and in particular does not
perform any optimizations that depend fundamentally on the
whole-program assumption.  The key challenge will be figuring out how
to define separate compilation and linking themselves.  The latter may
be especially interesting since CakeML (unlike CompCert) verifies
correctness of compilation all the way down to x86-64 machine code,
and thus linking will need to be defined at the machine-code level.


\subsection{Impact}

We verified separate compilation for the \emph{full} CompCert compiler for the first time.  In the
course of doing so, we uncovered two compiler bugs---one of which is on separate compilation and the
other is orthogonal to separate compilation---and our verification techniques are subsequently
adopted in the official CompCert 2.7.



%%% Local Variables:
%%% mode: latex
%%% TeX-master: "main"
%%% TeX-command-extra-options: "-shell-escape"
%%% End:
