\section{Introduction}
% dmitri: Is this a taugology? Does any kind of programming _not_ rely on
% some kind of understanding of semantics? I think the intention here is to
% emphasize the difficulty of writing correct programs in C. Also, the problem
% addressed by the paper, arbitrary manipulation of pointers as integer values,
% is not identified early enough.
%
% C programmers rely crucially on a mental model of their program's
% expected behavior:  they use their understanding of C semantics to
% guide program development, to help with debugging, and to
% predict the effects of program transformations and optimizations made
% by the compiler.

The ISO C standard~\cite{iso2011iec} famously does not give semantics to a significant
subset of syntactically valid C programs. Instead, many programs
exhibit implementation-defined, unspecified, or undefined behavior,
with the latter imposing no requirements on a conforming
implementation. This has led to the somewhat controversial practice of
sophisticated C compilers reasoning backwards from instances of
undefined behavior to conclude that, for example, certain code paths
must be dead. Such transformations can lead to surprising non-local
changes in program behavior and difficult-to-find bugs~\cite{wang2013towards,yang2011finding}.

Accordingly, there have been numerous efforts to capture the
subtleties of the C standard formally, either by giving an alternative
language definition or a conforming implementation~\cite{norrish1998c,leroy:compcert,ellison2012executable}.

 The C memory model has been of
particular interest: cross-platform low-level access to memory is a
defining feature of C-family languages and is essential for
applications such as operating system kernels and language
runtimes. However, subtle pointer aliasing
rules~\cite{krebbers2013aliasing}, reliance on implementation-specific
behavior, and the treatment of pointers to uninitialized memory makes
reasoning about even single-threaded programs non-trivial.

One popular extension of the standard C memory model that has not
previously been formalized is the \emph{unrestricted} manipulation of
pointers as integer values. While the language definition provides an integer
type \texttt{uintptr\_t} that may be legally cast to and from pointer
types, it does not require anything of the resulting values~\cite[\S7.20.1.4p1]{iso2011iec}. 
Nevertheless, there are many important use cases for manipulating
the representation of pointers in low-level code.

For example, casting pointers to integers is
widely used in the Linux kernel and JVM implementations to perform bitwise operations on pointers.
Another common usage pattern occurs in the C++ standard library (\texttt{std::hash}),
where the pointer's bit representation is used as a key for indexing into
a hash table.
This is useful since taking a pointer is a cheap way to get a unique key.

%As a consequence, most programmers probably rely on a fairly concrete
%model of C semantics in which pointer values are memory addresses,
%which are themselves ``just'' integers.  Together with some
%rules-of-thumb about data layout and the potential side effects of
%function calls, this model serves as a basis for reasoning about C
%programs.

% The extent to which the programmer's model of the C semantics agrees
% with that of the compiler implementor's is therefore an important
% aspect of C program development---incompatibility between the two
% perspectives can lead to hard-to-diagnose bugs or subtle dependencies
% on compiler implementation details.
%
% Unfortunately, the C11 standard, while it strives for precision and
% comprehensive coverage, is rather large, somewhat opaque, and
% relatively informal.  A significant amount of complexity in the
% standard is introduced to handle obscure cases or to justify the
% correctness of desirable compiler optimizations.


% The semantics in users' mind and that in compiler developer's must
% coincide. Otherwise, compiled program may behave in an unexpected way
% in a user's point of view. For example, compiler version up may make
% previously well behaved programs to crash.  Also, compiler verification
% result can be weakened.

%% \todo{Dead allocation elimination is another good example. It might be
%%   good to at least briefly explain the example.}


% dmitri: Would it be more accurate to say that the concrete memory model is the
% _simplest_ memory model that can support the arbitrary pointer manipulation
% that we're after rather than that all C memory models fall into these two categories?
% I'm also hesitant to make claims about what C programmers expect of
% the memory model.

The most straightforward way to support bit-level pointer 
manipulation is to adopt what is often called a \emph{concrete} memory
model. This approach most closely resembles what the machine is
actually doing: pointers have the same representation as integer
values of the appropriate width, and they simply index into a single
flat array representing memory.

However, the combination of finite memory and allowing casts
of arbitrary integers to pointers invalidates many basic compiler
optimizations. Consider, for example, a function \texttt{f} that
initializes a local variable \texttt{a} and then calls some unknown
external function~\texttt{g}. We might expect the compiler to deduce
that the value of \texttt{a} is unaffected by the call to \texttt{g}
and perform constant propagation:
%% On the formal side, there are two types of memory models for C\@.
% On the one hand, we have \emph{concrete} models.
% These represent pointers as 32-bit (or 64-bit) integers,
% which is close to what programmers expect, but do not support
% some of even the most basic optimizations that compilers perform.
% As a simple illustration, consider the following function \texttt{f} that
% initializes a local variable \texttt{a} and then calls some unknown external
% function \texttt{g}.
% The compiler might perform constant propagation and optimize this program
% as follows by assuming that the value of \texttt{a} is unaffected by the call to \texttt{g}.
\begin{center}
\begin{tabular}{@{}l@{}l@{}l@{}l@{}l@{}}
%% \small
\multicolumn{5}{c}{\texttt{extern void g();}}
\\
\begin{lstlisting}
int f(void) {
  int a = 0;
  g();
  return a;
}
\end{lstlisting}
&
$\quad\rightarrow\quad$
&
%% \small
\begin{lstlisting}
int f(void) {
  int a = 0;
  g();
  return 0;
}
\end{lstlisting}
&
$\quad\rightarrow\quad$
&
%% \small
\begin{lstlisting}
int f(void) {

  g();
  return 0;
}
\end{lstlisting}
\end{tabular}
\end{center}

Using a simple concrete memory model, however, we have to consider the
possibility that \texttt{g} is able to ``guess'' the location of
\texttt{a} in memory and changes its value. This can happen if, for
instance, our semantics allocates memory deterministically and 
the caller of~\texttt{f} sets up the state of memory appropriately for
\texttt{g}.

% We note, however, that this reasoning is not valid in the concrete model.
% The variable \texttt{a} is allocated somewhere in memory, and the external
% function might `guess' the address at which it is allocated and overwrite
% its value.
%

The compiler may further optimize the program by completely removing
the now unused local variable \texttt{a}. This latter transformation is again
disallowed by concrete models because it might change the behavior of
the program:
by virtue of there being one fewer memory cells allocated, the call to \texttt{g}
might succeed where initially it exhausted memory.



In order to enable such compiler optimizations, most work on verified
compilation instead relies on \emph{logical} memory models.  These
models represent pointers as pairs of an allocation block identifier
and an offset within that block, where typically the set of valid
allocation block identifiers is infinite.  In the above
example, the first optimization is allowed since there is no way to
forge the logical address of the variable \texttt{a} from those of
other blocks. Also, the second one is allowed since the
memory is infinitely large.

Logical models allow most compiler optimizations, but cannot
support many low-level C programming idioms using casts
between pointers and integers, treating programs containing them
as undefined (\ie erroneous).

%\todo{Show example with pointer to integer cast.}
%Gil:
%{%% \small
%\begin{lstlisting}
%void set_attribute (struct hash * h,
%     struct tree * v, struct attribute * w)
%{
%  hash_put(h, (void*) v, (void*) w)
%}
%\end{lstlisting}}

In this paper, we propose a \emph{quasi-concrete} memory model for C/C++ that
combines the strengths of the aforementioned approaches. It gives semantics to
programs that manipulate the bit-level representation of pointers, yet permits
the same optimizations as logical models for code not using these low-level
features. Crucially, we achieve this without substantially complicating the
proof techniques required for a verified compiler while retaining a model that is
simple for the programmer to reason about.

The key technical ingredient for making this work is having two
distinct representations of pointer values, a concrete and a logical one,
and a process for converting between the two.
By default, a pointer is represented logically, and only when it is
cast to an integer type, is the logical pointer value
\emph{realized} to a concrete 32-bit (or 64-bit) integer.
When an integer is cast (back) to a pointer value, 
it is mapped to the corresponding logical address.
%% it maintains its concrete
%% representation, and only mapped to the corresponding logical address
%% in memory accesses.

The quasi-concrete model conservatively extends the logical model.  It
gives semantics to strictly more programs than those supported by the
logical model without changing the semantics of those programs that do
have semantics under the logical model. Thus, any sound reasoning
about programs in the logical model also holds in the quasi-concrete
model, but the quasi-concrete model also supports reasoning about
pointer arithmetic as in the concrete model.

Finally, the quasi-concrete model is not intended to replace the
memory model in the C standard. Like the concrete and logical models,
it is a formal refinement of the (informal) C standard that can be
used for formally reasoning about programs and program transformations
(as in compiler verification).

\medskip \noindent
Our contributions are:
\begin{itemize}
\item The first formal memory model that \emph{fully} supports
  integer-pointer casts and yet allows the standard compiler
  optimizations.
\item A technique for proving program equivalence under our memory
model, and its application to verify a number of standard optimizations
that are difficult to verify in the presence of integer-pointer casts.
\end{itemize}

All the proofs reported in this chapter have been fully formalized in Coq and is available
online~\cite{kang-phd-thesis-web}.

%However, there is a big gap between the two.  The concrete semantics
%does not justify useful compiler optimizations.  So, the standard
%semantics, to justfiy such optimizations, have many informal, complex
%exceptional rules that users do not expect, and thus is hard to
%formalize. Compiler verification people use simple, formal logical
%semantics which justify optimizations.  However, the semantics does
%not support pointer-integer casting, a feature that users expect to
%use.
%
%
%The concrete semantics has a problem. use the following example to
%explain the lack of justifying optimizations.
%\begin{verbatim}
%void f(int i) {
%  int a[10];
%  int b = 0;
%  a[i] = 1;
%  return b;    -> return 0
%}
%\end{verbatim}
%
%The standard semantics solves the above problem using some
%exceptional rule. explain the rule and discuss the problems.
%


%%% Local Variables:
%%% mode: latex
%%% TeX-master: "main"
%%% TeX-command-extra-options: "-shell-escape"
%%% End:
