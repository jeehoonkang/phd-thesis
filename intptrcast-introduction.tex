\section{Introduction}
% dmitri: Is this a taugology? Does any kind of programming _not_ rely on
% some kind of understanding of semantics? I think the intention here is to
% emphasize the difficulty of writing correct programs in C. Also, the problem
% addressed by the paper, arbitrary manipulation of pointers as integer values,
% is not identified early enough.
%
% C programmers rely crucially on a mental model of their program's
% expected behavior:  they use their understanding of C semantics to
% guide program development, to help with debugging, and to
% predict the effects of program transformations and optimizations made
% by the compiler.

\paragraph{Context}

The ISO C standard~\cite{iso2011iec} famously does not give semantics to a significant
subset of syntactically valid C programs. Instead, many programs
exhibit implementation-defined, unspecified, or undefined behavior,
with the latter imposing no requirements on a conforming
implementation. This has led to the somewhat controversial practice of
sophisticated C compilers reasoning backwards from instances of
undefined behavior to conclude that, for example, certain code paths
must be dead. Such transformations can lead to surprising non-local
changes in program behavior and difficult-to-find bugs~\cite{wang2013towards,yang2011finding}.

Accordingly, there have been numerous efforts to capture the
subtleties of the C standard formally, either by giving an alternative
language definition or a conforming implementation~\cite{norrish1998c,leroy:compcert,ellison2012executable}.

 The C memory model has been of
particular interest: cross-platform low-level access to memory is a
defining feature of C-family languages and is essential for
applications such as operating system kernels and language
runtimes. However, subtle pointer aliasing
rules~\cite{krebbers2013aliasing}, reliance on implementation-specific
behavior, and the treatment of pointers to uninitialized memory makes
reasoning about even single-threaded programs non-trivial.

One popular extension of the standard C memory model that has not
previously been formalized is the \emph{unrestricted} manipulation of
pointers as integer values. While the language definition provides an integer
type $\mathtt{uintptr\_t}$ that may be legally cast to and from pointer
types, it does not require anything of the resulting values~\cite[\S7.20.1.4p1]{iso2011iec}. 
Nevertheless, there are many important use cases for manipulating
the representation of pointers in low-level code.

For example, casting pointers to integers is
widely used in the Linux kernel and JVM implementations to perform bitwise operations on pointers.
Another common usage pattern occurs in the C++ standard library (\code{std::hash}),
where the pointer's bit representation is used as a key for indexing into
a hash table.
This is useful since taking a pointer is a cheap way to get a unique key.

%As a consequence, most programmers probably rely on a fairly concrete
%model of C semantics in which pointer values are memory addresses,
%which are themselves ``just'' integers.  Together with some
%rules-of-thumb about data layout and the potential side effects of
%function calls, this model serves as a basis for reasoning about C
%programs.

% The extent to which the programmer's model of the C semantics agrees
% with that of the compiler implementor's is therefore an important
% aspect of C program development---incompatibility between the two
% perspectives can lead to hard-to-diagnose bugs or subtle dependencies
% on compiler implementation details.
%
% Unfortunately, the C11 standard, while it strives for precision and
% comprehensive coverage, is rather large, somewhat opaque, and
% relatively informal.  A significant amount of complexity in the
% standard is introduced to handle obscure cases or to justify the
% correctness of desirable compiler optimizations.


% The semantics in users' mind and that in compiler developer's must
% coincide. Otherwise, compiled program may behave in an unexpected way
% in a user's point of view. For example, compiler version up may make
% previously well behaved programs to crash.  Also, compiler verification
% result can be weakened.

%% \todo{Dead allocation elimination is another good example. It might be
%%   good to at least briefly explain the example.}


% dmitri: Would it be more accurate to say that the concrete memory model is the
% _simplest_ memory model that can support the arbitrary pointer manipulation
% that we're after rather than that all C memory models fall into these two categories?
% I'm also hesitant to make claims about what C programmers expect of
% the memory model.

\paragraph{Problem}

The most straightforward way to support bit-level pointer manipulation is to adopt the concrete
memory model, but as we have seen in \Cref{sec:introduction,sec:background:memory}, this model
invalidates many basic compiler optimizations such as constant propagation and dead allocation
elimination. \jeehoon{DAE?}

% due to the combination of finite memory and casts of arbitrary integers to pointers.

In order to enable such compiler optimizations, most work on verified compilation instead relies on
logical memory models, but as we have seen in \Cref{sec:background:memory}, they cannot support many
low-level C programming idioms using casts between pointers and integers, treating programs
containing them as undefined (\ie erroneous).  At the essence of the problem is that logical models
represent pointers as pairs of an allocation block identifier and an offset within that block, which
cannot be easily casted into and from an integer.

\jeehoon{Say the example in the introduction.}

%\todo{Show example with pointer to integer cast.}
%Gil:
%{%% \small
%\begin{lstlisting}
%void set_attribute (struct hash * h,
%     struct tree * v, struct attribute * w)
%{
%  hash_put(h, (void*) v, (void*) w)
%}
%\end{lstlisting}}

\paragraph{Our Solution}

In this chapter, we propose a \emph{hybrid memory model} for C/C++ that combines the strengths of the
aforementioned approaches. It gives semantics to programs that manipulate the bit-level
representation of pointers, yet permits the same optimizations as logical models for code not using
these low-level features. Crucially, we achieve this without substantially complicating the proof
techniques required for a verified compiler while retaining a model that is simple for the
programmer to reason about.

The key technical ingredient for making this work is having two
distinct representations of pointer values, a concrete and a logical one,
and a process for converting between the two.
By default, a pointer is represented logically, and only when it is
cast to an integer type, is the logical pointer value
\emph{concretized} to a concrete 32-bit integer (or 64-bit integer depending on the architecture).
When an integer is cast back to a pointer value, 
it is mapped to the corresponding logical address.
%% it maintains its concrete
%% representation, and only mapped to the corresponding logical address
%% in memory accesses.

The hybrid model conservatively extends the logical model.  It
gives semantics to strictly more programs than those supported by the
logical model without changing the semantics of those programs that do
have semantics under the logical model. Thus, any sound reasoning
about programs in the logical model also holds in the hybrid
model, but the hybrid model also supports reasoning about
pointer arithmetic as in the concrete model.

Finally, the hybrid model is not intended to replace the
memory model in the C standard. Like the concrete and logical models,
it is a formal refinement of the (informal) ISO C standard that can be
used for formally reasoning about programs and program transformations
(as in compiler verification).

\medskip \noindent
To summarize, our contributions are:
\begin{itemize}
\item The first formal semantics that \emph{fully} supports integer-pointer casts and yet allows the
  standard compiler optimizations (\Cref{sec:intptrcast:formal-semantics}).

\item A compiler verification technique for proving semantic preservation under our semantics, and its
  application to verify a number of standard optimizations that are difficult to verify in the
  presence of integer-pointer casts (\Cref{sec:intptrcast:compiler-verification}).
\end{itemize}

\noindent \Cref{sec:intptrcast:discussion} discusses the related work, the drawbacks, and the impact
of our memory model.  All the proofs reported in this chapter have been fully formalized in Coq and
is available online~\cite{kang-phd-thesis-web}.

%However, there is a big gap between the two.  The concrete semantics
%does not justify useful compiler optimizations.  So, the standard
%semantics, to justfiy such optimizations, have many informal, complex
%exceptional rules that users do not expect, and thus is hard to
%formalize. Compiler verification people use simple, formal logical
%semantics which justify optimizations.  However, the semantics does
%not support pointer-integer casting, a feature that users expect to
%use.
%
%
%The concrete semantics has a problem. use the following example to
%explain the lack of justifying optimizations.
%\begin{verbatim}
%void f(int i) {
%  int a[10];
%  int b = 0;
%  a[i] = 1;
%  return b;    -> return 0
%}
%\end{verbatim}
%
%The standard semantics solves the above problem using some
%exceptional rule. explain the rule and discuss the problems.
%


%%% Local Variables:
%%% mode: latex
%%% TeX-master: "main"
%%% TeX-command-extra-options: "-shell-escape"
%%% End:
