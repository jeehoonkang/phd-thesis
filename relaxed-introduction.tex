%!TEX root = main.tex
\section{Introduction}\label{sec:introduction}

\newcommand{\sevcik}{\v{S}ev\v{c}\'{\i}k}

What is the right semantics for concurrent shared-memory programs
written in higher-level languages?  For programmers, the simplest
answer would be a \emph{sequentially consistent (SC)} semantics, in
which all threads in a program share a single view of memory and
writes to memory take immediate global effect.
% Under an SC semantics, all threads in a program share a single global
% view of memory and witness writes to memory in the same order, which
% is consistent with the program order (the sequential order of writes
% within any one thread).
%

However, a naive SC semantics is costly to
implement.  % \ori{I
  % think it is better to start with the fact that no existing hardware
  % actually implements SC.  seems more convincing than compiler
  % optimizations}
First of all, commodity architectures (such as x86, Power, and ARM)
are not SC: they execute memory operations speculatively or out of
order, and they employ hierarchies of buffers to reduce memory
latency, with the effect that there is no globally consistent view of
memory shared by all threads.  To simulate SC semantics on these
architectures, one must therefore insert expensive fence instructions
to subvert the efforts of the hardware.  Secondly, a number of common
compiler optimizations---such as constant propagation---are rendered
unsound by a naive SC semantics because they effectively reorder
memory operations.
% Commodity
% architectures are not SC either: they execute memory operations out of
% order, so simulating SC behavior at the hardware level typically
% requires the insertion of expensive fence instructions to prevent
% reorderings.
% Thus, to provide SC semantics for all memory
% accesses would entail disabling many compiler optimizations and
% inserting expensive fence instructions throughout pSa naive SC model
% of memory is also costly to implement efficiently.  implementing such
% a semantics for all memory accesses would be costly: it would (1)
% preclude common compiler optimizations that merge or reorder memory
% operations, preserving SC semantics through compilation is costly.
% Compiler optimizations like constant propagation effectively merge or
% reorder memory operations, and commodity hardware executes
% instructions out of order, so to support SC semantics for all accesses
% would require one to disable common compiler optimizations and insert
% expensive fence instructions throughout compiled code.
Moreover, SC semantics is stronger (\ie more restrictive) than
necessary for many concurrent algorithms.

Hence, languages like Java and C++ have opted instead to provide
\emph{relaxed} (aka \emph{weak}) memory models~\cite{jmm,cppstandard},
which enable programmers to demand SC semantics when they need it, but
which also support a range of cheaper memory operations that trade off
strongly consistent and/or well-defined behavior for efficiency.

\subsection{Criteria for a Programming Language Memory Model}
\label{sec:criteria}

Unfortunately, despite many years of research, it has proven very
difficult to develop a memory model for concurrent programming
languages that adequately balances the conflicting desiderata of
programmers, compilers, and hardware.  In particular, we would like to
find a memory model that satisfies the following properties:
% In particular, some basic properties one expects of such a memory
% model are the following:
\vspace{-.5ex}
\begin{itemize}
\item The model should be \emph{implementable}, \ie it should validate
  common compiler optimizations, as well as standard compilation
  schemes to the major modern architectures.  To be implementable, it
  must justify many kinds of instruction reordering and merging.\\[-2.5ex]
\item The model should support \emph{high-level reasoning} principles
  that programmers and compiler analyses depend on.  At a bare
  minimum, it should validate simple invariant-based verification, and
  should provide some ``DRF'' guarantees~\cite{Adve:1990}, ensuring that
  programmers who employ sufficient synchronization need not understand
  the full complexities of relaxed-memory semantics.\\[-2.5ex]
\item The model should ideally \emph{avoid relying on undefined
    behavior} to define the semantics of racy programs.  This is a
  prerequisite for applicability to type-safe languages like Java, in
  which well-typed programs may contain data races but are
  nevertheless expected to have safe, well-defined semantics.
\end{itemize}
\vspace{-.5ex}
\mbox{}\indent Both Java and C++ fail to achieve some of these criteria.

In the case of Java, the memory model fails to validate a number of
common program transformations performed by real Java compilers, such
as redundant read-after-read elimination and ``roach motel''
reordering~\cite{sevcik:jmm}.  Although this problem has been known
for some time, a satisfactory solution has yet to be developed.

In the case of C++, the memory model relies crucially on undefined
behaviors to give semantics to racy programs.  Moreover, it permits
certain ``out-of-thin-air'' executions which violate basic
invariant-based reasoning (and DRF guarantees)~\cite{Boehm2014}.
%To illustrate the problem, let us give a concrete example.

% \begin{itemize}
% \item Our model supports a wide range of common compiler and hardware
%   reorderings (including read-write, give list here, etc.)
% \ori{should we give the list? maybe the first is enough}.  We
%   establish the validity of these reorderings using fairly
%   straightforward simulation arguments.

% \item Our model rules out ``bad'' OOTA behaviors, such as $a$
%   receiving $1$ in \ref{eq:LBdep}.  As a result, we are able to prove standard
%   (and some non-standard) ``DRF theorems''.  These theorems guarantee
%   that programmers need not understand the full complexities of our
%   relaxed-memory semantics if they adhere to a restricted programming
%   discipline (\eg only racing on SC accesses \ori{under Sc semantics?}). 
%   In addition, our model supports the basic reasoning principle of global non-relational
%   invariants.

% % \item Our model supports basic invariant-based reasoning for relaxed
% %   accesses by prohibiting ``bad'' out-of-thin-air executions that
% %   would violate such reasoning.  It \emph{does} permit some other
% %   out-of-thin-air-\emph{ish} executions, but only ones that do not
% %   break invariant-based reasoning and that are arguably desirable to
% %   allow in any case because the ARM model allows them. \gil{ARM together with common compiler optimizations}

% \item Our model is designed to serve as a foundation for both C/C++
%   and Java.  On the one hand, it supports a range of different memory
%   operations (\eg nonatomic, SC, release-acquire, relaxed, fences) in
%   the style of C++.\footnote{Say something about correspondence with
%     C++ here?}  On the other hand, we do not rely on undefined
%   behavior/values to account for data races---programs have a fully
%   defined semantics under our model, which is a prerequisite for
%   applicability to a type-safe language like Java.

% % \item Our model validates standard (and some non-standard) ``DRF
% %   theorems''.  These theorems ensure that programmers who use only a
% %   subset of the supported language features (\eg only SC and nonatomic
% %   accesses) need not understand the more complex aspects of the model
% %   related to features they do not use.  (Is that right? What do we
% %   want to say here?)

% % \item
% % Say something about liveness here?
% \end{itemize}

% \section{Introduction}\label{sec:introduction}

% What is the right semantics for concurrent shared-memory programs
% written in higher-level languages?  For programmers, the simplest
% answer would be a \emph{sequentially consistent (SC)} semantics, in
% which all threads in a program share a single view of memory and
% writes to memory take immediate global effect.
% % Under an SC semantics, all threads in a program share a single global
% % view of memory and witness writes to memory in the same order, which
% % is consistent with the program order (the sequential order of writes
% % within any one thread).  
% %
% However, a naive SC semantics is costly to
% implement.  % \ori{I
%   % think it is better to start with the fact that no existing hardware
%   % actually implements SC.  seems more convincing than compiler
%   % optimizations}
% First of all, commodity architectures (such as x86, Power, and ARM)
% are not SC: they execute memory operations speculatively or out of
% order, and they employ hierarchies of buffers to reduce memory
% latency, with the effect that there is no globally consistent view of
% memory shared by all threads.  To simulate SC semantics on these
% architectures, one must therefore insert expensive fence instructions
% to subvert the efforts of the hardware.  Secondly, a number of common
% compiler optimizations---such as constant propagation---are rendered
% unsound by a naive SC semantics because they effectively reorder
% memory operations.
% % Commodity
% % architectures are not SC either: they execute memory operations out of
% % order, so simulating SC behavior at the hardware level typically
% % requires the insertion of expensive fence instructions to prevent
% % reorderings.
% % Thus, to provide SC semantics for all memory
% % accesses would entail disabling many compiler optimizations and
% % inserting expensive fence instructions throughout pSa naive SC model
% % of memory is also costly to implement efficiently.  implementing such
% % a semantics for all memory accesses would be costly: it would (1)
% % preclude common compiler optimizations that merge or reorder memory
% % operations, preserving SC semantics through compilation is costly.
% % Compiler optimizations like constant propagation effectively merge or
% % reorder memory operations, and commodity hardware executes
% % instructions out of order, so to support SC semantics for all accesses
% % would require one to disable common compiler optimizations and insert
% % expensive fence instructions throughout compiled code.  
% Moreover, SC semantics is stronger (\ie more restrictive) than
% necessary for many concurrent algorithms.  Hence, languages like Java
% and C++ have opted instead to provide \emph{relaxed} (aka \emph{weak})
% memory models~\cite{jmm,cppstandard}.  These models make it possible to enforce
% SC semantics when programmers need it, but they also support a range
% of cheaper memory operations that trade off strongly consistent and/or
% well-defined behavior for efficiency.

\subsection{The ``Out of Thin Air'' Problem}
\label{sec:oota}

% Unfortunately, both the Java and C++ memory models suffer from serious
% flaws  (see also \cite{Batty:2015}).  In the case of
% Java, the memory model fails to justify \ori{validate?} a number of common program
% transformations performed by real Java compilers, such as redundant
% read-after-read elimination and ``roach motel''
% reordering~\cite{sevcik:jmm}.  In the case of C++, the memory model
% permits certain ``out-of-thin-air'' executions which violate basic
% reasoning principles that programmers and compiler analyses depend on~\cite{Boehm2014}.

% (Maybe something here about how the JMM
% effort has stalled?) 

%

% To understand why satisfying all these criteria is so hard, let us
% look at a concrete example.  
To illustrate the problem with C++, consider these two variants of the
classic ``load buffering'' litmus test (with two threads in parallel):
% Consider these two variants of the classic ``load buffering'' litmus
% test (with two threads running concurrently):
\begin{center}
\begin{minipage}{.46\columnwidth}
\begin{equation}\label{eq:LB}\tag{LB}
\inarrII{ a:=x; \\ y:=1; }{ x:=y; }\!\!\!
\end{equation}
\end{minipage}
\hfill
\begin{minipage}{.46\columnwidth}
\begin{equation}\label{eq:LBdep}\tag{LBd}
\inarrII{ a:=x; \\ y:=a; }{ x:=y; }
\end{equation}
\end{minipage}
\end{center}
Here, we assume that all variables are initially $0$, and that all
memory accesses are of the weakest consistency level, \ie they are
compiled down to plain loads and stores at the hardware level with no
additional synchronization (in C++ this is called ``relaxed'').  The
question is: should it be possible for these programs to assign $1$ to
$a$?  In the case of \ref{eq:LB}, the answer is yes: architectures
like Power and ARM may reorder the write of $y$ before the read of $x$
in the first thread (since these are accesses to distinct variables),
after which $a$ can be assigned $1$ by a standard interleaving
execution. In the case of \ref{eq:LBdep}, however, the answer
\emph{ought} to be no: all the operations simply copy one variable to
another and all are initially $0$, so if $a$ could receive $1$, it
would come ``out of thin air''.  No hardware reorderings or reasonable
compiler optimizations will produce this behavior.  If they did, it
would cause major problems: one would not be able to establish even
basic invariants (such as $x = y = 0$), and basic sanity results like
the aforementioned DRF theorems would cease to hold.  It is therefore
a serious problem that the formal memory model of C++ allows such
out-of-thin-air (OOTA) behavior.
% \footnote{Actually, the C++ model
%   disallows this behavior with a vague English side condition, but it
%   does not explain clearly what OOTA means~\cite{n3710}.}

Intuitively, the reason C++ allows OOTA behaviors is that it is not
clear how to distinguish them from acceptable behaviors.  The C++
model formalizes valid executions as graphs of memory access events
(think: partially-ordered traces) subject to a set of coherence
axioms, and the same coherent event graph that describes a valid
execution of \ref{eq:LB} in which $a$ receives $1$ also describes a valid
execution of \ref{eq:LBdep} in which $a$ receives $1$.

% Naively, one might hope to solve this problem by exploiting information
% about syntactic dependencies between instructions in the program.
% In particular, one might hope to prohibit the OOTA behavior in \ref{eq:LBdep}
% by observing that the write to $y$ has a data dependency on the previous
% instruction, so the reordered execution of \ref{eq:LB} should not be applicable
% to \ref{eq:LBdep}.  Indeed, 

Hardware memory models (\eg Power and ARM) handle this problem by
taking syntactic dependencies between instructions into account in
determining program semantics.  Under such models, the out-of-order
execution in \ref{eq:LB} is valid because the write to $y$ is independent of
the read from $x$, whereas in \ref{eq:LBdep} such out-of-order execution is
prevented by the syntactic dependency between the two instructions.
Although this approach is suitable for modeling hardware, it is too
brittle for a language-level semantics because it fails to validate
standard compiler optimizations that remove syntactic dependencies (see also \cite{Boehm2014}).
As a very simple example, consider the following variant of \ref{eq:LB} and
\ref{eq:LBdep}:
\begin{equation}\label{eq:LBfalsedep}\tag{LBfd}
\inarrII{ a:=x; \\ y:=a+1-a; }{ x:=y; }
\end{equation}
Under the hardware models, this \ref{eq:LBfalsedep} program would be treated
similarly to \ref{eq:LBdep} due to the syntactic data dependency, so $a$ could
not receive $1$.  But even a basic optimizing compiler could trivially
transform \ref{eq:LBfalsedep} to \ref{eq:LB}, in which case $a$ could receive $1$.
%\ori{a simple answer here is that one can bake in the compiler optimizations
%to the definition of dependency, should we explain why it is really not working?}
%\jeehoon{Ori, I think that is too much information in the introduction.  How about saying it in the result section?}

% A naive solution to this problem would be for the semantics to 
% loads and delay their execution to the point when their value is
% actually used.  Such a semantics would enable reordering in the case
% of \ref{eq:LB}, but would prevent the OOTA behavior in the case of \ref{eq:LBdep}
% because there is a syntactic dependency of $y := a$ on the previous
% instruction.  Indeed, this approach has been successfully applied in
% definitions of hardware memory models.  


%   so if one can assign $1$ to
% $a$, one can assign any value to $a$.  And, the C11 model allows
% precisely this.  Not only is this behavior unintuitive---it also means
% that one cannot establish even simple global invariants such as
% $x = y = 0$, and one cannot prove standard ``sanity'' theorems like
% the DRF property.



% it suffers from the so-called ``out-of-thin-air'' (OOTA)
% problem.  Roughly speaking, the OOTA problem is the problem of how to
% provide a model of relaxed memory accesses that is weak enough to
% account for the reorderings/optimizations that hardware and compilers
% may perform, yet also strong enough to support at least some very
% basic reasoning that compiler analyses and programmers depend on (\eg
% global invariants on single variables).

% , namely the so-called
% ``out-of-thin-air'' problem~\cite{??,??}.  Roughly speaking, the
% problem is how to provide a model of relaxed memory accesses that is
% weak enough to account for the reorderings/optimizations that hardware
% and compilers may perform, yet also strong enough to support at least
% some very basic reasoning that compiler analyses and programmers
% depend on (\eg global invariants on single variables).  The existing
% models are either too strong (disallowing common compiler
% optimizations) or too weak (permitting variables to receive arbitrary
% ``out-of-thin-air'' values that break even basic invariant-based
% reasoning). 
%


As a result, we still to this day lack a semantics for relaxed-memory
concurrency in Java and C++ that corresponds to how these languages
are implemented and that provides sufficient reasoning guarantees to
programmers and compiler-writers.  Several proposals have recently
been made for how to fix the Java and C++ memory models (some of which
are discussed in \cref{sec:related}), but none have been proven to
validate the full range of standard optimizations/reorderings
performed by Java and C++ compilers and by commodity hardware like
Power and ARM.  Furthermore, for most of the existing proposals, it is
known that indeed they do \emph{not} validate some important
reorderings.

\subsection{A ``Promising'' Semantics for Relaxed Memory}
\label{sec:promising}

In this paper, we present what we believe is a very promising way
forward: the first relaxed memory model to support a broad spectrum
of features from the C++ concurrency model while also satisfying all
three criteria listed in \cref{sec:criteria}.


% \begin{itemize}
% \item
% Accounts for a broad spectrum of features of the C++11 memory model.
% \item

% \end{itemize}


% \begin{itemize}
% \item
% accounts for nearly all the features of the C++11 memory model,
% \item
% provably validates a number of standard compiler
%   optimizations, as well as a wide range of memory access reorderings
%   that commodity hardware may perform,
% \item avoids bad
%   ``out-of-thin-air'' behaviors that break invariant-based reasoning
% and DRF guarantees, and
% \item defines the semantics of racy programs without relying on
%   undefined behaviors.
% \end{itemize}


% \begin{itemize}
% \item Our model supports a wide range of common compiler and hardware
%   reorderings (including read-write, give list here, etc.)
% \ori{should we give the list? maybe the first is enough}.  We
%   establish the validity of these reorderings using fairly
%   straightforward simulation arguments.

% \item Our model rules out ``bad'' OOTA behaviors, such as $a$
%   receiving $1$ in \ref{eq:LBdep}.  As a result, we are able to prove standard
%   (and some non-standard) ``DRF theorems''.  These theorems guarantee
%   that programmers need not understand the full complexities of our
%   relaxed-memory semantics if they adhere to a restricted programming
%   discipline (\eg only racing on SC accesses \ori{under Sc semantics?}). 
%   In addition, our model supports the basic reasoning principle of global non-relational
%   invariants.

% % \item Our model supports basic invariant-based reasoning for relaxed
% %   accesses by prohibiting ``bad'' out-of-thin-air executions that
% %   would violate such reasoning.  It \emph{does} permit some other
% %   out-of-thin-air-\emph{ish} executions, but only ones that do not
% %   break invariant-based reasoning and that are arguably desirable to
% %   allow in any case because the ARM model allows them. \gil{ARM together with common compiler optimizations}

% \item Our model is designed to serve as a foundation for both C/C++
%   and Java.  On the one hand, it supports a range of different memory
%   operations (\eg nonatomic, SC, release-acquire, relaxed, fences) in
%   the style of C++.\footnote{Say something about correspondence with
%     C++ here?}  On the other hand, we do not rely on undefined
%   behavior/values to account for data races---programs have a fully
%   defined semantics under our model, which is a prerequisite for
%   applicability to a type-safe language like Java.

% % \item Our model validates standard (and some non-standard) ``DRF
% %   theorems''.  These theorems ensure that programmers who use only a
% %   subset of the supported language features (\eg only SC and nonatomic
% %   accesses) need not understand the more complex aspects of the model
% %   related to features they do not use.  (Is that right? What do we
% %   want to say here?)

% % \item
% % Say something about liveness here?
% \end{itemize}


We achieve these ends through a combination of mechanisms (some
standard, some not), but the most important and novel idea for the
reader to take away from this paper is the notion of \emph{promises}.

Under our model, which is defined by an operational semantics, a thread $T$ may nondeterministically ``promise''
to write a value $v$ to a memory location $x$ at some point in the
future.  From the point of view of other threads, a promise is no
different from an ordinary write: once $T$ has promised to write $v$
to $x$, other threads can read from that write.  (In contrast, $T$
cannot read from its own promised write until $T$ has fulfilled the
promise: this is crucial to preserve basic sanity of the semantics.)
Intuitively, promises simulate the effect of read-write reorderings by
allowing write events to be visible to other threads before the point
at which they occur in the program order.

We must, however, ensure that promises do not introduce bad OOTA
behaviors.  Toward this end, we only allow $T$ to promise to write $v$
to $x$ if it is possible to \emph{thread-locally certify} that the
promise can be fulfilled in a finite number of steps.
That is, we must show that $T$
will be able to write $v$ to $x$ after some finite sequence of steps
of $T$'s execution (\ie with no help from other threads).  The
certification requirement guarantees absence of bad OOTA executions by
ensuring that $T$ can only promise to write a value $v$ to $x$ if $T$
could have written $v$ to $x$ anyway.

Returning to the examples from \cref{sec:oota}, it is easy to see how
promises give us the desired semantics:
\begin{itemize}
\item In \ref{eq:LB}, the first thread can promise to write $1$ to $y$ (since
  it will indeed write $1$ to $y$ no matter what value is assigned to
  $a$), after which the second thread can read from that promise and
  write $1$ to $x$.  Subsequently, the first thread can execute
  normally, reading $1$ from $x$ and assigning it to $a$.
\item The execution of \ref{eq:LBfalsedep} may proceed in exactly the same way.  The
  fact that the write of $y$ depends syntactically on $a$ is
  irrelevant, because during certification of the promised write of $1$
  to $y$, the expression $a+1-a$ will always evaluate to $1$.
\item By contrast, the OOTA behavior will not be allowed for \ref{eq:LBdep}.
  In order for the first thread to promise to write $1$ to $y$, it
  would need to certify that it can write $1$ to $y$ 
%in some finite number of steps 
without promises.  But since all variables are
  initially $0$, this is not possible.
\end{itemize}

Our model supports all features of C++ concurrency except consume
reads and SC accesses.  Consume reads are widely considered a
premature aspect of the C++11 standard and are currently implemented
the same as acquire reads in mainstream compilers.  In contrast, SC
accesses are a major feature of C++, and originally our model included
an account of SC accesses as well.  However, in the course of trying
to mechanize correctness of compilation to Power
(\cref{sec:compilation_POWER}), we discovered that our semantics of SC
accesses was flawed, and this led us to discover a flaw in the C++11
standard as well!  (See \cite{repairing-sc} for further details.)
Thus, a proper handling of SC accesses remains an open and important
problem for future work.


% \derek{I removed this paragraph.  It didn't seem to really add anything.}
% In this way, promises provide a solution to the OOTA problem that does
% not rely on tracking brittle syntactic dependencies.  Furthermore, as
% we will see, the thread-local nature of our promise certification greatly
% simplifies the formal development of our semantics and has enabled us
% to straightforwardly prove the validity of read-write reorderings that
% were out of reach for prior work~\cite{Jagadeesan2010}.

% it is possible for $t$ to thread-locally \emph{validate} that $t$
% itself will be able to write $v$ to $x$ through a sequence of steps of
% its own execution.  (Hence, the name ``promise'': a thread can only
% promise a write event will happen if it itself will be the one to
% carry it out.)  Once $t$ has promised to write $v$ to $x$, other
% threads can read from that write, but $t$ itself cannot.  The
% validation requirement guarantees absence of ``bad'' out-of-thin-air
% executions by ensuring that a thread can never promise to write a
% value $v$ to $x$ unless it is possible for the thread to write $v$ to
% $x$ without making any promises.

% Intuitively, promises simulate the effect of read-write reorderings by
% allowing write events to be visible to other threads before the point
% where they occur in the program order.  The validation requirement
% guarantees absence of ``bad'' out-of-thin-air executions by ensuring
% that a thread can never promise to write a value $v$ to $x$ unless it
% could have in fact written $v$ to $x$ in a promise-free execution.
% Furthermore, the \emph{thread-local} nature of our promise validation
% greatly simplifies the formal development of our semantics and has
% enabled us to straightforwardly prove the validity of read-write
% reorderings that were out of reach for prior work~\cite{esop10??}.

% XXX

% Under our semantics, a thread $t$ may nondeterministically ``promise''
% to write a value $v$ to a memory location $x$ at some point in the
% future, but only if it is possible for $t$ to thread-locally
% \emph{validate} that $t$ itself will be able to write $v$ to $x$
% through a sequence of steps of its own execution.  (Hence, the name
% ``promise'': a thread can only promise a write event will happen if it
% itself will be the one to carry it out.)  Once $t$ has promised to
% write $v$ to $x$, other threads can read from that write, but $t$
% itself cannot.

% Intuitively, promises simulate the effect of read-write reorderings by
% allowing write events to be visible to other threads before the point
% where they occur in the program order.  The validation requirement
% guarantees absence of ``bad'' out-of-thin-air executions by ensuring
% that a thread can never promise to write a value $v$ to $x$ unless it
% could have in fact written $v$ to $x$ in a promise-free execution.
% Furthermore, the \emph{thread-local} nature of our promise validation
% greatly simplifies the formal development of our semantics and has
% enabled us to straightforwardly prove the validity of read-write
% reorderings that were out of reach for prior work~\cite{esop10??}.









% C11, for instance, distinguishes between
% \emph{nonatomic} accesses (intended for normal data accesses) and
% \emph{atomic} accesses (intended to implement synchronization between
% threads).  The former can be compiled efficiently, and their semantics
% is SC, but their behavior is only safe and well-defined if the programmer
% can guarantee the absence of data races on them.  The latter are specifically
% intended for racy access, but are available at a range of consistency 

% memory accesses: the former may be compiled efficiently but
% their behavior is only defined\emph{release-acquire} accesses (useful
% for implementing message-passing between threads where the global
% synchronization of SC is overkill), \emph{relaxed} accesses (useful
% for racy accesses where order of execution is not important) and
% \emph{non-atomic} accesses (for ``normal'' variablesand \emph{fully
%   relaxed} accesses. but they also permit programmers to indicate when
% the strength of SC semantics is not needed


% \subsection{Overview of the Paper}
% \label{sec:overview}

In the rest of the paper, we will flesh out the idea of promises---as
well as the other elements of our model---in layers.  We begin in
\cref{sec:relaxed} by presenting the details of our model restricted
to relaxed reads and writes.  In \cref{sec:updates}, we extend this
base model further to support atomic updates (\ie read-modify-write
operations, like CAS).  Then, in \cref{sec:full}, we scale the model
up to handle most features of the C++ memory model.  In
\cref{sec:results}, we present our formal results---validating many
program transformations, compilation to x86-TSO and Power, DRF
theorems, and an invariant-based logic---most of which are fully
mechanized in the Coq proof assistant (totalling about 37K lines of
Coq).  In \cref{sec:related}, we compare with related work, and in
\cref{sec:discussion}, we conclude with discussion of future work.


%%% Local Variables:
%%% mode: latex
%%% TeX-master: "main"
%%% TeX-command-extra-options: "-shell-escape"
%%% End:
