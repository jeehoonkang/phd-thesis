\subsection{Declarative Presentation of the Promise-Free Machine}
\label{sec:axiomatic}

In this section, we provide a declarative presentation of our model, 
in the style of C++11, namely using sets of \emph{execution graphs} to describe the possible behaviors of programs.
This presentation abstracts away particular timestamp choices and thread views, 
and replaces them by formal conditions on partial orders in execution graphs.
The promise mechanism has an inherent operational nature, 
and thus our declarative presentation only applies to
the \emph{promise-free machine} (see end of \cref{sec:full-formal}).
Nevertheless, this presentation is useful for comparing our model to C++11, 
and it is used for establishing the correctness of compilation (see \cref{sec:compilation_POWER}). 
We also find this useful as a technical device for analyzing
possible behaviors of the promise-free machine and applying \cref{thm:drfpf}.

%The declarative presentation associates programs with graphs,
The nodes of execution graphs are called \emph{events}.
%and use $i$ as a metavariable for thread identifiers.
%
An {event} consists of
an identifier (natural number), 
a thread identifier (taken from a finite set $\Tid$ of thread identifiers,
or $0$ for initialization events), 
and a \emph{label}.
Labels have the form
${\rlab}(o,x,v)$ or ${\wlab}(o,x,v)$ 
(where $o$ is the access mode, 
$x$ is the location accessed, and
$v$ is the value read/written),
as well as $\flab_\acqo$, $\flab_\relo$, or $\flab_\sco$
(for fences).
The functions $\tid$, $\lab$, $\typ$, and $\loc$
return (when applicable) the thread identifier, 
label, type ($\rlab,\wlab,\flab$), and location of an event.

In turn, an \emph{execution graph} $G$  consists of:
\squishlist
\item a set $\lE$ of events.
This set always contains a set $\lE_0$ of initialization events,
consisting of one plain write event assigning the initial value for every location.
We assume that all initial values are $0$.
For $\tlab\in\{\rlab,\wlab,\flab\}$, 
we denote by $\tlab$ the set 
$\{a\in \lE \suchthat \typ(a)=\tlab\}$.
We also %concatenate the event sets notations, use subscripts to denote the accessed location,
use subscript for access modes
(\eg $\lW_{\sqsupseteq \rlx}$ denotes the set of all events $a\in \lW$ with 
whose access mode is at least $\rlx$).
\item a relation $\lSB$, called \emph{sequenced before}, 
which is a union of relations $(\lE_0\times (\lE\setminus \lE_0)) \cup \{\lSB_i \suchthat i\in\Tid\}$, 
where every $\lSB_i$ ($i\in\Tid$) is a strict total order on $\{a\in \lE \suchthat \tid(a)=i\}$.
\item  a relation $\lRMW$, called \emph{read-modify-write pairs}, consisting of 
 \emph{immediate} $\lSB$-edges
(\ie if $\tup{a,b}\in\lRMW$ then no $c$ satisfies both $\tup{a,c}\in\lSB$ and $\tup{c,b}\in\lSB$).
In addition, for every $\tup{a,b}\in\lRMW$, we have $\typ(a)=\rlab$, $\typ(b)=\wlab$, and
 $\loc(a)=\loc(b)$.
\item a relation $\lRF$, called \emph{reads-from}, which relates every read event in $\lE$
with one write event in $\lE$ that has the same location and value. 
\item a relation $\lMO$, called  \emph{modification order},
which is a disjoint union of relations $\{\lMOx\}_{x\in\Loc}$, 
such that each relation $\lMOx$ is a strict total order on $\lW_x$.
\item  a relation $\lSC$, called \emph{SC order}, which is a strict total order on $\lF_\sco$ 
(the set of all SC fence events in $\lE$).
\squishend

\begin{notation}
$R^?$, $R^+$, and $R^*$ respectively denote the reflexive, transitive, and reflexive-transitive closures of a binary relation $R$. 
$R^{-1}$ denotes its inverse relation.
%$R\rst{x}$ denotes the restriction of $R$  to events accessing location $x$,
%while  $R\rst{\LOC}$ denotes the restriction of $R$ to events accessing the same location
% (\ie $R\rst{x} = \{\tup{a,b} \in R \suchthat \loc(a)=\loc(b)=x\}$ and
% $R\rst{\LOC} = \{\tup{a,b} \in R \suchthat \loc(a)=\loc(b)\}$).
We denote by $R_1;R_2$ the left composition of two relations $R_1,R_2$.
Finally, $[A]$ denotes the identity relation on a set $A$.
In particular, $[A];R;[B] = R\cap (A\times B)$.
\end{notation}

Now, to define which execution graphs are allowed, 
we derive an \emph{happens-before} order $\lHB$,
which is defined as in C++11
(after applying the correction suggested in \cite{c11comp} for release sequences):
\begin{align*}
\lSB\rst{\LOC} &=  \{\tup{a,b} \in \lSB \suchthat \loc(a)=\loc(b)\} \tag{\emph{sb-loc}}
\\ \lRSEQ &=  [\lW];\lSB\rst{\LOC}^?;[\lW_{\sqsupseteq\rlx}];(\lRF;\lRMW;[\lW_{\sqsupseteq\rlx}])^* \tag{\emph{release-seq}}
\\ \lREL &= ([\lW_\ra] \cup ([\lF_\relo \cup \lF_\sco];\lSB));\lRSEQ \tag{\emph{to-be-released}}
\\ \lSYN &= \lREL;\lRF; ([\lR_\ra] \cup ([\lR_{\sqsupseteq \rlx}];\lSB;[\lF_\acqo\cup \lF_\sco])) \tag{\emph{sync}}
\\ \lHB &= (\lSB \cup \lSYN)^+  \tag{\emph{happens-before}}
\end{align*}

Given the definition of $\lHB$, 
an {execution graph} $G$  is \emph{consistent} if the following hold:
\begin{itemize}
\item $\lMO;\lHB$ is irreflexive.  \hfill(\emph{WW-coherence})
\item $\lMO;\lRF;\lHB$ is irreflexive.  \hfill(\emph{RW-coherence})
\item $\lMO;\lHB ; \lRF^{-1}$ is irreflexive.  \hfill(\emph{WR-coherence})
\item $\lMO;\lRF;R;\lRF^{-1}$ is irreflexive,   \hfill(\emph{RR-coherence})
\\ where $R=([\lR_{\sqsupseteq \rlx}];\lHB) \cup 
(\lHB;[\lR_{\sqsupseteq \rlx}]) \cup
(\lHB;[\lF_\sco];\lHB)$.
\item $(\lRF;\lRMW) \cap (\lMO;\lMO) = \emptyset$.  \hfill(\emph{Atomicity})
\item $\lMO;\lRF^?;\lHB;\lSC; \lHB;(\lRF^{-1})^?$  is irreflexive.  \hfill(\emph{SC})
\item $\lSB \cup \lRF \cup \lSC$ is acyclic. \hfill(\emph{No-promises})
\end{itemize}

%\jeehoon{I think RW-coherence should cover rf; hb.}
%\ori{$\lRF;\lHB$ is covered due to ``no-promises''}
%\jeehoon{Ah, I see.  And yet I think it would be better for RW-coherence to require $\lMO^?;\lRF;\lHB$, as it better matches with C++11 definitions.  Though I am fine with the current version, too.}
%
%\jeehoon{I think it is easier to read if the relation begins with hb.  It matches with the rule names.}
%\ori{I'm not sure what do you mean here.
%Perhaps write your alternative definition, and then we decide.}
%\viktor{I think I understand. Here it is:
%\begin{itemize}
%\item $\lHB;\lMO$ is irreflexive.  \hfill(\emph{WW-coherence})
%\item $\lHB;\lMO;\lRF$ is irreflexive.  \hfill(\emph{RW-coherence})
%\item $\lHB;\lRF^{-1};\lMO;$ is irreflexive.  \hfill(\emph{WR-coherence})
%\item $R;\lRF^{-1};\lMO;\lRF$ is irreflexive,   \hfill(\emph{RR-coherence})
%\\ where $R=([\lR_{\sqsupseteq \rlx}];\lHB) \cup 
%(\lHB;[\lR_{\sqsupseteq \rlx}]) \cup
%(\lHB;[\lF_\sco];\lHB)$.
%\end{itemize}}
%\noindent \jeehoon{Yes, Viktor said what I meant.  Again, I am fine with the current version, too.}


Putting aside  differences in presentation (\eg instead of a primitive RMW event,
we have two events related by an $\lRMW$-edge), 
there are three essential differences between this model and the C++11 one~\cite{Batty:2011}:
\begin{itemize}
\item Our model disallows cycles in $\lSB\cup\lRF\cup\lSC$, and hence it does not
permit load buffering behaviors (which are not possible in the promise-free machine).
\item Our model lacks SC accesses, and its condition for SC fences is stronger than the one
of C++11. In particular, unlike in C++11, placing an SC fence between every two commands
\emph{does} guarantee SC semantics.
%\jeehoon{I am worried if it may hamper the contribution of our scfix paper.}
%\ori{hopefully, it'd be fine. Anyway its mimicking the operational semantics...}
\item Our model does not have non-atomic accesses on which races imply undefined behavior.
It does have plain accesses for which RR-coherence does not apply 
(unless an SC fence is $\lHB$-between).
\end{itemize}

Now, to define behaviors of programs using consistent 
execution graphs and programs, we present a ``declarative machine'', 
which incrementally builds a consistent execution graph
following some interleaving of the program's threads operations.
Formally, the declarative machine state is a pair $\tup{\Sigma,G}$, 
where $\Sigma$ assigns a local state $\sigma$ to every thread
(as described in \cref{sec:relaxed}), and $G$ is some consistent execution graph.
The possible steps of this machine are given in \cref{fig:axiomatic-opsem},
assuming the same abstract programming language discussed  in \cref{sec:full-formal}.
To define these steps, we use the following notation for execution graph extension.
% \jeehoon{How about replacing ``assigns a local state $\sigma$ to every thread'' with ``assigns a local state to every thread''?}
% \ori{I think that $\sigma$, even though not used, will help the reader to
% recall (or check) what do we mean by ``local state'', since there was also a notion
% of thread state that is irrelevant here}

\begin{notation}
For two execution graphs $G$ and $G'$,
we write $G'\in\add{G,a}$
if $G'$ extends $G$ with one event $a$,
which is $\lSB\cup\lRF\cup\lSC$ maximal in $G'$.
We write $G'\in\addatomic{G,a_\lr,a_\lw}$
if $G'$ extends some $G_{\text{mid}}\in \add{G,a_\lr}$ with 
one event $a_\lw$ and an $\lRMW$-edge $\tup{a_\lr,a_\lw}$,
and $a_\lw$ is $\lSB\cup\lRF\cup\lSC$ maximal in $G'$.
\end{notation}

The initial machine state is given by $\tup{\lambda i.\sigma_i^0, G_0}$,
where $G_0$ contains only the initialization events $\lE_0$ (its relations are empty).
A behavior of a program under the declarative machine 
is again taken to be the set of sequences of system calls induced by its executions.

\begin{remark}
In a purely declarative style, one considers only \emph{full} runs of a program and checks consistency only once at the end.
However, the definition of consistent execution graphs above is ``prefix-closed''
(that is, every $\lSB\cup\lRF\cup\lSC$-prefix of a consistent execution graph forms a consistent execution graph).  As a result, we were able to present our declarative
semantics in a more operational style,
which can be conveniently related to the promise-free machine.
\end{remark}

\begin{figure}[t]
\small
\begin{mathpar}
%\inferrule[(silent)]
%
\vspace*{-1mm}
%
\inferrule*
{\sigma \astep{\silentt} \sigma'}  
{\tup{\sigma,G} \astep{i} \tup{\sigma',G}}
~~
%\inferrule[(read/write/fence)]
\inferrule*
{\sigma \astep{\lab(a)} \sigma' \\\\
\typ(a) \in \{\rlab,\wlab,\flab\} \\\\
\tid(a)=i \\\\
G' \in \add{G,a}}
{\tup{\sigma,G} \astep{i} \tup{\sigma',G'}}
~~
%\inferrule[(update)]
\inferrule*
{\sigma \astep{\ulab(x,v_\lr,v_\lw,o_\lr,o_\lw)} \sigma' \\\\
\lab(a_\lr)=\rlab(o_\lr,x,v_\lr) \\\\
\lab(a_\lw)=\wlab(o_\lw,x,v_\lw) \\\\
\tid(a_\lr)=\tid(a_\lw)=i \\\\
 G' \in \addatomic{G,a_\lr,a_\lw}}
 {\tup{\sigma,G} \astep{i} \tup{\sigma',G'}}
%
\\\vspace*{-1mm}
%
%\inferrule[(system call)]
\inferrule*
{\sigma \astep{\syscallt} \sigma' \\
\lab(a)=\flab_\sco \\\\
\tid(a)=i \\
G' \in \add{G,a}}
{\tup{\sigma,G} \astep{i,\syscallt} \tup{\sigma',G'}}
\and
\inferrule[(machine step)]
%\inferrule*
{\tup{\Sigma(i),G} \astep{i,e} \tup{\sigma',G'}\\\\
\text{$G'$ is consistent}}
{\tup{\Sigma,G} \astep{e} \tup{\Sigma[i\mapsto \sigma'],G'}}
\end{mathpar}
\caption{Operational semantics based on the declarative model.}
\label{fig:axiomatic-opsem}
\end{figure}

\begin{theorem}
\label{thm:axiomatic}
For every program $\prog$, the behaviors of $\prog$ according to the promise-free machine
coincide with the behaviors of $\prog$ according to the declarative machine.
\end{theorem}

The Coq proof of this theorem involves a non-trivial simulation argument.
The simulation relation is presented in~\cite{appendix}.

%%% Local Variables:
%%% mode: latex
%%% TeX-master: "main"
%%% TeX-command-extra-options: "-shell-escape"
%%% End:
