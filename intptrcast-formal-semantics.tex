\section{Formal Semantics of Hybrid Model}
\label{sec:intptrcast:formal-semantics}

Our model is simply a hybrid of the fully concrete
model and the fully logical model. However, there are several issues
with how to combine the two models to minimize their disadvantages.
In this section, we introduce the hybrid model and discuss how
we address the design issues at a high level.  % More detailed
% discussions with concrete examples will follow in the subsequent
% sections.  

All the optimization examples presented in this section are performed by \code{clang -O2}.
Furthermore, unless specified otherwise, integer variables have type \code{uintptr\_t}, and pointer
variables have type \code{int*}.  Recall that \code{uintptr\_t} is an integer type that is able to
hold a pointer value.


\subsection{Hybrid of Concrete and Logical Blocks}
\label{sec:intptrcast:formal-semantics:representation}

Our hybrid model slightly generalizes the logical model to allow
both concrete blocks (as in the concrete model) and logical blocks (as
in the logical model). For this, we add one more attribute $p$ to a
logical block presented in \Cref{sec:background:memory},
which is either \emph{undefined} or a concrete address.
The attribute $p$ indicates whether the block is logical (when $p$ is
undefined) or it is a concrete block starting at the address $p$ (when
$p$ is defined).
\[
\begin{array}{@{}l@{~}c@{~}l@{}}
\mathrm{Block} &\defeq&
\setof{(v, p, n, c) \suchthat
  p \in \mathtt{int32}\uplus\setofz{\mathrm{undef}} \\
&&\qquad\qquad\quad~~
  {}\land v \in \setofz{\rtlvalid,\rtlfreed} \land n \in \Nat \land c \in \rtlVal^n }
\end{array}
\]
We say that an address $(l,i)$ is concrete when the block $l$ is a
concrete block starting at an address $p$. 
%% In this case, the
%% hybrid model does not distinguish between the address $(l,i)$
%% and the integer $p+i$ (\ie they are completely identified).
In this case, the address $(l,i)$ can be cast to the integer $p+i$
and vice versa (see \Cref{sec:intptrcast:formal-semantics:langsem} for details).

As in the concrete model, the list of valid (\ie allocated) blocks with concrete addresses must be
consistent: they should not include zero (\ie{} \code{nullptr}) or the maximum address, and their
ranges should be disjoint. Logical blocks have no such requirement, since they are non-overlapping
by construction.
%\label{model:issues}

In the subsequent subsections, we discuss several issues that arose during the design of our hybrid
model and justify our solutions to the issues.

\subsection{Combining Logical and Concrete Blocks}
\label{sec:intptrcast:formal-semantics:hybrid}

%\todo{Give a forward reference to related work with the APLAS paper.
%and say that the work is based on logical model and does not fully support
%pointer-integer casting (e.g. hash table does not work).
%}

Our hybrid model allows both concrete and logical blocks to coexist, suffering from the
disadvantages of both kinds of blocks: concrete ones do not provide exclusive ownership and logical
ones do not allow casting to integers.  Then why do we not develop a new notion of block that has
the advantages of both concrete and logical blocks instead?

Because such a new notion of block would not justify other important optimizations such as
simplification of integer operations.  For instance, consider a model in which some blocks have both
concrete addresses and some extra permission information, so that we can tell when a block is
exclusively owned. In such a model, we would like to know that we do not lose permission information
when a pointer is cast to an integer, even if integer operations are performed on it (\eg
\code{base64\_enc}oding a pointer and then \code{base64\_dec}oding it).  However, this prevents the
optimization presented in \Cref{fig:intptrcast:formal-semantics:arith1}.  Suppose the variable
\code{b} contains an integer with permission to access some valid block $l$, and \code{a} contains
an integer without any permission that is equal to the concrete address of the block $l$. Then the
source program successfully stores 123 into the block $l$ because \code{q} has the relevant
permission, whereas the target program fails because \code{q} does not have the permission.
%
\begin{figure}[t]
\center
\begin{tabular}{lll}
%\small
\begin{minipage}{0.4\textwidth}
\begin{minted}{c}
a = (a - b) + (2 * b - b);
q = (int *) a;
*q = 123;
\end{minted}
\end{minipage}
&
$\quad\rightarrow\quad$
&
%\small
\begin{minipage}{0.2\textwidth}
\begin{minted}{c}

q = (int *) a;
*q = 123;
\end{minted}
\end{minipage}
\end{tabular}
\caption{Arithmetic optimization example I}\label{fig:intptrcast:formal-semantics:arith1}
\end{figure}
%

Notice that this optimization is sound in our model.  See
\Cref{sec:intptrcast:compiler-verification:examples} for how to verify it in our model.


\subsection{Choosing Concrete Blocks}
\label{sec:intptrcast:formal-semantics:cast}

As discussed in \Cref{sec:background:memory}, using concrete addresses for memory locations provides
no guarantees of ownership, and thus prevents certain optimizations. In the worst case, one function
could guess the concrete address of a supposedly-private resource of another function, and then
forge a pointer to that address and modify it.

In order to maximize the range of optimizations that can be performed, our hybrid model assigns
concrete addresses to only those blocks whose concrete addresses are requested via
pointer-to-integer casts.  This gives a simple semantics of pointer-to-integer casts in that $(1)$
it clearly defines when a block should be made concrete, and $(2)$ the semantics of integer
operations is still independent from the memory in the presence of pointer-to-integer casts.

The reader may wonder if we can further maximize the optimization opportunity by making concrete
only those blocks whose concrete addresses are really used in some operation.  If we perform some
computation with the value of a pointer that only makes sense when that value is an integer
(\eg % using it as an offset into another block
comparing it with an integer value) then the target of that pointer must have a concrete address.
In all other cases, even if the address of the block is taken, we could conceivably use a logical
value and maintain the ownership guarantees of the logical model.

However, this approach has a serious problem: it does not justify some important integer
optimizations, such as a dead code elimination presented in
\Cref{fig:intptrcast:formal-semantics:dce}.  Suppose the pointer \code{p} contains a logical block
$l$. In the source program, since its concrete address is used in the function \code{foo}, the block
$l$ must be given a concrete address. In the target, the read-only call to \code{foo} is optimized
away, and the block $l$ may not be given a concrete address. That is, the source may have more
concrete blocks than the target. Thus, if \code{bar()} accesses an arbitrary concrete memory
location, then that access might succeed in the source but fail in the target. Since a failure is
possible in the target that did not exist in the source, the optimization has introduced new
behavior, and is invalid.

\begin{figure}[t]
\small
\center
\begin{tabular}{lll}
%\small
\begin{minipage}{0.3\textwidth}
\begin{minted}{c}
void foo(uintptr_t a) {
    a = a & 123;
}
...
a = (uintptr_t) p;
foo(a);
bar();
\end{minted}
\end{minipage}
&
$\quad\rightarrow\quad$
&
%\small
\begin{minipage}{0.3\textwidth}
\begin{minted}{c}
foo(uintptr_t a) {
    a = a & 123;
}
...
a = (uintptr_t) p;

bar();
\end{minted}
\end{minipage}
\end{tabular}
\caption{Dead code elimination example}
\label{fig:intptrcast:formal-semantics:dce}
\end{figure}


Notice that this optimization is sound in our hybrid model, because it makes concrete all the blocks
whose addresses are cast to integers, even if the cast integers are not used in any operation.  See
\Cref{sec:intptrcast:compiler-verification:examples} for how to verify this example in our model.
Furthermore, our model allows most of the optimizations in practice that would be performed in the
minimally-concrete model presented above (see \Cref{sec:intptrcast:formal-semantics:drawbacks} for
details).


\subsection{Assigning Concrete Addresses}
\label{sec:intptrcast:formal-semantics:ownership}

Once we know which blocks will need concrete addresses, we need to
decide when during a program's execution to assign those addresses.

One approach would be to make such a decision as early as possible (\ie at allocation time). We
allocate blocks as either logical or concrete, and cause concrete operations (namely integer casts)
on logical blocks to raise \emph{out-of-memory-type behavior}.  (For its precise meaning, we refer
the reader to \Cref{sec:intptrcast:formal-semantics:oom}.)  Since it is difficult to determine
whether a block will need a concrete address,
%% (and we would like to be able to compile our programs to more concrete ones),
we would need to choose the kind of block to allocate
non-deterministically. However, this would add unintuitive failures to
our model, effectively allowing out-of-memory-type behavior when the
allocator chooses the wrong kind of block even when concrete blocks
are available.

Our solution is to instead allocate all blocks as logical blocks, and assign concrete addresses to logical blocks at casting time. This casting can result in out of memory only when there is not enough free concrete space. By waiting to make blocks concrete until we reach the casting point,
we can remove the non-determinism about whether the blocks are
concrete or logical. That is, blocks are always logical until the
first casting point, and concrete afterward.
 %% at which they must be concrete, we reduce the non-determinism of the model. 

\begin{figure}[t]
\small
\center
\begin{tabular}{lll}
%\small
\begin{minipage}{0.3\textwidth}
\begin{minted}{c}
p = (int *) malloc(4);
*p = 123;
bar();
a = *p;
hash_put(h, p, a);
\end{minted}
\end{minipage}
&
$\quad\rightarrow\quad$
&
%\small
\begin{minipage}{0.3\textwidth}
\begin{minted}{c}
p = (int *) malloc(4);
*p = 123;
bar();
a = *p;
hash_put(h, p, 123);
\end{minted}
\end{minipage}
\end{tabular}
\caption{Ownership transfer example}\label{fig:intptrcast:formal-semantics:ownership}
\end{figure}

This also allows \emph{ownership transfer} optimizations such as the
constant propagation example in \Cref{fig:intptrcast:formal-semantics:ownership}, in which
pointers are privately owned up until some point and then become
publicly available.  
% On the other hand, in the hybrid model,
This optimization is performed by \code{clang -O2} and higher.
In this example,
the allocated block is initially logical and becomes concrete when cast to an
integer (possibly in the call to \code{hash\_put}). At this point, the ownership of the block is transferred from
private to public. Since \code{a} is treated as logical up until the
call to \code{hash\_put}, we can perform constant propagation as
normal before the call. (For formal details, see \Cref{sec:intptrcast:compiler-verification:examples}.)

On the other hand, the above model with
non-deterministic allocation does not allow such optimizations for the following reasons.
When the allocated block in the target is concrete, the
  corresponding allocation in the source must be concrete; otherwise,
  when the function \code{hash\_put} casts the address of the block to
  an integer the source program raises out-of-memory-type behavior, while the target succeeds.
Thus, you lose the ownership over the block and cannot justify the constant propagation due to the presence of \code{bar()}.

   % Note that this may not be a decisive example, however, since this
   % optimization is not used in all real-world compilers (it is
   % performed by \code{clang -O2} and higher, but not by \code{gcc
   %   -O2} or higher).


\subsection{Operations on Pointers}

In \Cref{sec:intptrcast:formal-semantics:cast}, we explained that the fewer blocks we make concrete, the more we can take advantage of the ownership guarantees provided by logical blocks. We have seen that operations that require pointers to have integer values force a lower bound on the number of blocks that must be made concrete. As such, we can improve our model by reducing the number of operations that require concrete addresses. We take our cue from CompCert's memory model, in which several arithmetic operations, such as integer-pointer addition and subtraction of pointers from pointers in the
same block, are well-defined even in the absence of a concrete address (see \Cref{sec:background:memory}). 

One disadvantage of CompCert's approach to arithmetic operations is that it invalidates some important arithmetic optimizations, such as the optimization presented in \Cref{fig:intptrcast:formal-semantics:arith2}, by introducing logical addresses as possible values for integer-typed variables. 
To see this, suppose that the integer variables \code{a}, \code{b}, \code{c1}, and \code{c2} contain the same logical address $(l,0)$. The source program shown will successfully assign $(l,0)$ to the variables \code{d1} and \code{d2}, because \code{b-c1} and \code{b-c2} evaluate to $0$. However, in the target, the variable $\code{t}$ gets an unspecified value, because the addition of two logical addresses is undefined. Thus, the target has more behaviors than the source, and the optimization is invalid.

\begin{figure}[t]
\center
\begin{tabular}{lll}
%\small
\begin{minipage}{0.3\textwidth}
\begin{minted}{c}

d1 = a + (b - c1);
d2 = a + (b - c2);
\end{minted}
\end{minipage}
&
$\quad\rightarrow\quad$
&
%\small
\begin{minipage}{0.3\textwidth}
\begin{minted}{c}
t = a + b;
d1 = t - c1;
d2 = t - c2;
\end{minted}
\end{minipage}
\end{tabular}
\caption{Arithmetic optimization example II}\label{fig:intptrcast:formal-semantics:arith2}
\end{figure}

We avoid this disadvantage through the use of type checking. 
As in the LLVM IR, we use types to ensure that integer variables contain only integer values.
This allows us to justify the full range of arithmetic optimizations on integer variables (see \Cref{sec:intptrcast:compiler-verification:examples} for details), 
while also giving semantics to the operations on logical addresses when possible.
See \Cref{sec:intptrcast:formal-semantics:langsem} for an example of type-dependent semantics of arithmetic operations in our model.

\subsection{Dead Cast Elimination}
\label{sec:intptrcast:formal-semantics:deadcast}

As a result of our design decisions thus far, casts have become
important effectful operations in our model, determining the points at
which logical blocks are given concrete addresses. This leads to a
potential problem with dead code elimination.  Since casting a pointer
to an integer has a side effect in memory, removing dead cast
operations is not obviously justified in the hybrid model.

However, in fact,
we can solve this problem and support dead cast elimination in our
framework. 
The solution stems from our model's place in a broader compilation
framework. We expect the hybrid model to be used for mid-level
intermediate representations in a compiler, while the back-end
low-level language will use a fully concrete model. In the
hybrid model, casting a pointer to an integer has a side
effect on memory, and we cannot eliminate cast operations. In the
fully concrete model, however, a cast from pointer to integer is a 
no-op, and such casts can always be eliminated. Thus, we can
perform dead-cast-elimination optimizations in the backend.

However, we still have a problem when dead cast is combined with dead
allocation. In the concrete model allocations of dead blocks cannot be
removed, because otherwise an arbitrary access in the source may
succeed but fail in the target. 

Our solution to this problem is to remove dead casts combined with dead blocks during the
translation from the hybrid to the fully concrete model.  For instance, consider the dead call
elimination optimization presented in \Cref{fig:intptrcast:formal-semantics:deadcast}. This
optimization is not valid when both the source and the target use the hybrid model, due to the cast
operation in the function. It is also not valid when both the source and the target use the concrete
memory, due to the allocation in the function. However, the optimization is valid when the source
uses the hybrid model and the target uses the fully concrete model (see
\Cref{sec:intptrcast:compiler-verification:examples} for formal details).

\begin{figure}[t]
\small \center
\begin{tabular}{@{}l@{}l@{~~}l}
%\small
\begin{minipage}{0.4\textwidth}
\begin{minted}{c}
void foo(int *p, int n) {
  auto q = malloc(n);
  auto a = (uintptr_t) p;
  auto r = a * 123;
}
\end{minted}
\end{minipage}
&
$\quad\rightarrow\quad$
&
%\small
\begin{minipage}{0.4\textwidth}
\begin{minted}{c}
void foo(int *p, int n) {
  auto q = malloc(n);
  auto a = (uintptr_t) p;
  auto r = a * 123;
}
\end{minted}
\end{minipage}
\\
\\
%\small
\begin{minipage}{0.4\textwidth}
\begin{minted}{c}
...
foo(p, n);
bar();
\end{minted}
\end{minipage}
&
$\quad\rightarrow\quad$
&
%\small
\begin{minipage}{0.4\textwidth}
\begin{minted}{c}
...

bar();
\end{minted}
\end{minipage}
\end{tabular}
\caption{Dead cast elimination example}
\label{fig:intptrcast:formal-semantics:deadcast}
\end{figure}

Although our solution does not justify the removal of all dead casts,
it should cover most of them in practice (see \Cref{sec:intptrcast:formal-semantics:drawbacks}).

%% To sum up, ...  

%% We do not have any practically meaningful optimizations justified in
%% symbolic model but not justified in our concolic model
%% (see \Cref{sec:intptrcast:formal-semantics:drawbacks} for details).

\subsection{Drawbacks of the Hybrid Model}
\label{sec:intptrcast:formal-semantics:drawbacks}

% Furthermore, while our hybrid model does disallow some optimizations based on exclusive ownership
% (since once a block has become concrete it is non-exclusive for the remainder of the execution), we
% expect that our model would not lose many optimization opportunities in practice. This is because
% exclusively owned blocks are mostly local or temporary ones, so that their concrete addresses are
% unlikely to be used by integer operations. See \Cref{sec:intptrcast:formal-semantics:drawbacks} for further details.

Although our hybrid model is designed to allow as many optimizations as possible, it still disallows some reasonable optimizations. 
In particular, if a function newly allocates a block and casts its address to an
integer, then it loses the ownership guarantees on the block. 
Even if the block is still effectively locally owned, 
once its address is cast to an integer, we can no longer perform optimizations that rely on its locality, such as dead code elimination or constant propagation. 

However, we think that blocks whose addresses are cast to
integers in actual programs are unlikely to be completely local. There are few reasons to cast a local pointer to an integer unless the address will be shared with other code sections. 

For instance, consider the following example of a simple program in which we might want to perform a locality optimization even after casting a pointer to an integer.
The program is a variant of 
the example in \Cref{sec:intptrcast:formal-semantics:deadcast}, in which we cast \code{q} into an integer instead of~\code{p},
so that in our model the local block becomes concrete and cannot be eliminated.
Although this optimization cannot be performed in our framework, the function \code{foo} is nothing but an %silly pseudo random 
unpredictable number generator, and is unlikely to occur in real programs.

\begin{center}
\small
\begin{tabular}{@{}l@{}l@{}l@{}}
%\small
\begin{minipage}{0.4\textwidth}
\begin{minted}{c}
void foo(ptr p, int n) {
  auto q = malloc(n);
  auto a = (uintptr_t) q;
  auto r = a * 123;
}
\end{minted}
\end{minipage}
&
$\quad\rightarrow\quad$
&
%\small
\begin{minipage}{0.4\textwidth}
\begin{minted}{c}
void foo(ptr p, int n) {
  auto q = malloc(n);
  auto a = (uintptr_t) q;
  auto r = a * 123;
}
\end{minted}
\end{minipage}
\\
\\
%\small
\begin{minipage}{0.4\textwidth}
\begin{minted}{c}
...
foo(p, n);
bar();
\end{minted}
\end{minipage}
&
$\quad\rightarrow\quad$
&
%\small
\begin{minipage}{0.4\textwidth}
\begin{minted}{c}
...

bar();
\end{minted}
\end{minipage}
\end{tabular}
\end{center}

Another, more reasonable limitation of our model occurs when one privately uses a local block for some time, then casts its address to an integer and releases it to the
public (\eg by using the integer as a key for hash table).  Consider the following example, which is a variant of the example in \Cref{sec:intptrcast:formal-semantics:ownership},
where we cast the address of a newly allocated block into an integer
and use the integer as a key for hash table:

\vskip 0.2cm

\begin{center}
\small
\begin{tabular}{@{}l@{}l@{}l@{}}
%\small
\begin{minipage}{0.3\textwidth}
\begin{minted}{c}
...
p = (int *) malloc(4);
*p = 123;
b = (uintptr_t) p;
bar();
a = *p;
hash_put(h, b, a);
...
\end{minted}
\end{minipage}
&
$\qquad\rightarrow\qquad$
&
%\small
\begin{minipage}{0.3\textwidth}
\begin{minted}{c}
...
p = (int *) malloc(4);
*p = 123;
b = (uintptr_t) p;
bar();
a = *p;
hash_put(h, b, 123);
...
\end{minted}
\end{minipage}
\\
\end{tabular}
\end{center}

This constant propagation optimization is invalid in the
hybrid model because the newly allocated block is cast to an
integer before the call to \code{bar}.  (It becomes valid if the
cast is moved after the call to \code{bar}, though.)  However, while
ownership transfer optimizations of this sort are indeed performed by
\code{clang -O2}, they are not performed by \code{gcc -O2} or
higher, and can be viewed as minor optimizations that are not often
used.


%% because if they are going to be used temporarily, 
%% public blocks (\ie imported from the environment, or
%% exported to the environment).

%% as we will see below, such optimizations are very unlikely to happen in
%% usual C programs anyway, and thus it is not a serious drawback.

%% For example, consider a variant of the dead cast elimination example
%% in \Cref{ex:deadcast} (LHS below).  In this example, the function
%% \code{foo} allocates a block and cast its address to an integer.  In
%% our framework, the function \code{foo} is never read-only and thus
%% we cannot eliminate call to \textt{foo} even when the source has a
%% concolic memory and the target has a concrete memory.  However, in
%% some intuitive sense, this function \code{foo} can be seen as
%% read-only and indeed both ``clang -O2'' and ``gcc -O2'' optimizes away
%% a call to \code{foo} when the return value is ignored.


%% \begin{minipage}{0.3\textwidth}
%% \begin{minted}{c}
%% foo() {
%%   var ptr p, int r;
%%   p = malloc (1);
%%   3 -> p;
%%   _ = (int) p;
%%   bar();
%%   r <- p;
%%   printf(r);           $\to$ printf(3);
%% }
%% \end{minted}
%% \end{minipage}

%% \begin{itemize}
%% \item The above example, a constant propagation of the value of a
%%   casted pointer across a function call, is not justified in our
%%   model.  Since \code{foo} does not have exclusive ownership of
%%   \code{p} due to a casting to integer, \code{bar} may change the
%%   content of \code{p}.
%% \end{itemize}

%% \begin{tabular}{lll}
%% \begin{minipage}{0.3\textwidth}
%% \begin{minted}{c}
%% var ptr p,int r;
%% p = malloc (1);
%% 3 -> p;
%% r = (int) p;
%% r = r * 5;
%% free (p);
%% \end{minted}
%% \end{minipage}
%% &
%% \begin{minipage}{0.3\textwidth}
%% \begin{minted}{c}
%% var ptr p;
%% p = malloc (1);

%% _ = (int) p;

%% free (p);
%% \end{minted}
%% \end{minipage}
%% &
%% \begin{minipage}{0.3\textwidth}
%% \begin{minted}{c}
%% var ptr p;
%% p = malloc (1);



%% free (p);
%% \end{minted}
%% \end{minipage}
%% \end{tabular}

%% \begin{itemize}
%% \item Unused casting is \emph{almost} dead, but we cannot eliminate
%%   out the \code{malloc} and \code{free} of the casted block.
%%   Between the logical languages, we can optimize out everything except
%%   for \code{malloc}, \code{free}, and a cast.  From the logical
%%   langauge to the concrete language, we can remove the cast, too.
%% \item In both cases, we performed the int casting of short-term
%%   memory.  Intuitively, this is very unnatural, hard to find use cases
%%   and hard to expect such usecases to occur often.  So, we do not lose
%%   much optimization opportunities.
%% \end{itemize}

%% \mbox{}\\
%% -----------------------\\
%% We first discuss the advantages of the concological model.

%% The key idea is that when an unrealized address $(l,i)$ is cast to an
%% integer, we find a free concrete space (say at $p$) and realize the
%% block $l$ in the free space, so that the address $(l,i)$ is identified
%% with the integer $p+i$. On the other hand, casting an already realized
%% address is no operation. Thus, in this model, the cast of a pointer to
%% an integer is an operation with side effect on the memory, which is
%% the key difference from the logical model.

%% The cast operation is non-deterministic, whereas the memory allocation
%% is deterministic.

%% If the address $(l,i)$ is already realized,
%% the cast is no operation.

%% operation

%% cast -> realize

%% reasoning principle -> visualize

%% special operations

%% Explain how we turn the logical model into hybrid one.

%% explain the invariant visually.

%% Main idea is to model casting as realization
%% (migration of logical block to concrete one).

%% Explain about handling OOM as stuck, not error (explain why)

%% examples.

%% revisit the hash table example.

%% \begin{itemize}
%% \item Explain its simplicity.
%% \item Explain the problem using one of the following two examples.
%% \begin{verbatim}
%% void f(int i) {
%%   int a[10];
%%   int b = 0;

%%   a[i] = 1;

%%   return b;
%% }
%% ->
%% void f(int i) {
%%   int a[10];
%%   int b = 0;

%%   a[i] = 1;

%%   return 0;
%% }

%% Or

%% extern void g(void);
%% void f(void) {
%%   int a = 0;
%%   g();
%%   return a;
%% }
%% ->
%% extern void g(void);
%% void f(void) {
%%   int a = 0;
%%   g();
%%   return 0;
%% }
%% \end{verbatim}
%% \item Explain why it does not work using one of the following.
%% \begin{verbatim}
%% f(10);

%% Or

%% void g(void) {
%%   int* p = (int *) 0x0000ff00;
%%   *p = 1;
%%   return;
%% }
%% \end{verbatim}

%% \end{itemize}


\subsection{Language Semantics}
\label{sec:intptrcast:formal-semantics:langsem}

This section describes how to use the ideas of the hybrid memory model to give semantics to C-like
languages, including CompCert's RTL language presented in \Cref{sec:background:rtl}.


\paragraph{NULL pointer} 
We represent the NULL pointer as the logical address $(0,0)$ and 
initialize the block $0$ as follows:
\[
m(0) = (v,p,n,c) \text{ with } v=\code{true}, p=0, n=1.
\]
The only special treatment of the block $0$ is that we $(1)$ raise
undefined behavior when accessing it via a load or a store;
and $(2)$ do nothing when freeing it (because \code{free(0)} is allowed in C).


\paragraph{Casting between Integers and Pointers}

We first define casting between integers and logical addresses via \emph{reification} and \emph{validity checking}.
The reification function $\toint{}{m}$ under memory $m$
converts a logical address to a corresponding integer if its block in memory has a concrete address.
The validity predicate $\text{valid}_m$ checks if a logical address
is inside the range of a valid block.
\[
\begin{array}{@{}r@{~~}c@{~~}ll@{}}
\toint{(l, i)}{m} &\defeq& p + i &\text{if }~ m(l) = (v, p, n, c) \land p\text{ is defined} \\[2mm]
\text{valid}_m(l,i) &\text{iff}& 
\multicolumn{2}{@{}l@{}}{
m(l) = (v, p, n, c) \land v=\mathrm{true} \land (0\le i < n)}
\\
\end{array}
\]

Casting a logical address $(l,i)$ into an integer first concretizes the
block $l$ and then reifies the address $(l,i)$ if it is valid;
otherwise, raises undefined behavior.  Casting an integer $i$ yields a
valid logical address $(l,j)$ that is reified to $i$ if there is such
an address; otherwise, raises undefined behavior. Note that 
an integer is cast to a unique address if it succeeds.
\[
\begin{array}{@{}l@{~}c@{~}ll@{}}
\code{cast2int}_m(l,i) &\defeq& \toint{(l,i)}{m} &\text{if }~ \text{valid}_m(l,i)
~~~\text{\{after concretizing $l$\}}
\\
\code{cast2ptr}_m(i)   &\defeq& (l,j) & \text{if }~ \text{valid}_m(l,j) \land \toint{(l,j)}{m}=i \\
\end{array}
\]


\paragraph{Computing with Logical Values}
%% In order to use the hybrid model in our language, we must first clarify the treatment of logical values in the language. Recall that logical addresses are pairs $(l, i)$ where $l$ is a block identifier and $i$ is an integer offset. 

%% We identify logical addresses with integers via \emph{normalization}.
%% The normalization function ${|-|_m}$ under memory $m$
%% converts a logical address into a corresponding integer if its block in memory has a concrete address:%%normalization syntax?
%% \[
%% \begin{array}{@{}r@{~}c@{~}ll@{}}
%% \toint{(l, i)}{m} &\defeq& p + i &\text{if }~ m(l) = (v, p, n, c) \land p\text{ is defined} \\
%% \toptr{i}{m} &\defeq& (l,j) &\text{if }~ i = \toint{(l,j)}{m} \land \text{valid}_m(l,j)\\[2mm]
%% \multicolumn{4}{@{}l@{}}{
%% \text{where } \text{valid}_m(l,i) \text{ iff } 
%% m(l) = (v, p, n, c) \land v=\mathrm{true} \land (0\le i < n).} \\
%% \end{array}
%% \]

We now give semantics to the binary operations based on the static types of their operands. 
When both operands are of type \code{int}, we perform ordinary integer addition, subtraction, etc. 
When one or more arguments are of type \code{ptr}, we give the operations special semantics for the well-defined cases
and raise undefined behavior otherwise: %%lines
\[
\begin{array}{@{}l@{~}l@{}}
%% (\code{p + a}, m) \Downarrow i_1 + i_2 & \text{ if } |\code{p}|_m = i_1 \land
%%  \code{a} = i_2\\
(\code{p + a}, m) \Downarrow (l, i_1 + i_2) & \text{ if }~ \code{p} = (l, i_1) \land \code{a} = i_2\\
%% (\code{a + p}, m) \Downarrow i_1 + i_2 & \text{ if } \code{a} = i_1 \land |\code{p}|_m = i_2\\ %%We can remove the symmetric cases if we need the space.
(\code{a + p}, m) \Downarrow (l, i_1 + i_2) & \text{ if }~ \code{a} = i_1 \land \code{p} = (l, i_2)\\
%% \end{array}
%% \]
%% \[
%% \begin{array}{@{}ll@{}}
%% (\code{p - a}, m) \Downarrow i_1 - i_2 & \text{ if } |\code{p}|_m = i_1 \lan
  %% d \code{a} = i_2\\
(\code{p - a}, m) \Downarrow (l, i_1 - i_2) & \text{ if }~ \code{p} = (l, i_1) \land \code{a} = i_2\\
%% (\mathtt{p_1 \mathbin{\code{-}} p_2}, m) \Downarrow i_1 - i_2 & \text{ if } |\mathtt{p_1}|_m = i_1 \land |\mathtt{p_2}|_m = i_2\\
(\mathtt{p_1 \mathbin{\code{-}} p_2}, m) \Downarrow i_1 - i_2 & \text{ if }~ \mathtt{p_1} = (l, i_1) \land \mathtt{p_2} = (l, i_2)\\
%% \end{array}
%% \]
%% The question of pointer equality is slightly more complicated. In the
%% presence of logical blocks, we will not always be able to tell whether
%% two pointers are equal: in particular, when the offset of a logical
%% address is outside the range of the block, its actual target is
%% undetermined. In defining the semantics of equality, then, we identify
%% as many cases as possible in which we can be certain that two pointers
%% are not equal: 
%% When both pointers normalize to integers or to logical
%% pointers in the same block, we can compare them
%% straightforwardly. Finally, when an operand is a logical pointer with
%% an offset within the range of its block, we know that it is not equal
%% to the null pointer or any pointer in the range of a different block.
%% \[
%% \begin{array}{@{}l@{~~~}l@{}}
%% (\mathtt{p_1 \mathbin{\code{=}} p_2}, m) \Downarrow i_1 = i_2 & \text{ if } |\mathtt{p_1}|_m = i_1 \land |\mathtt{p_2}|_m = i_2\\
(\mathtt{p_1 \mathbin{\code{=}} p_2}, m) \Downarrow i_1 = i_2 & \text{ if }~ \mathtt{p_1} = (l, i_1) \land \mathtt{p_2} = (l, i_2) \\
%% (\mathtt{p_1 \mathbin{\code{=}} p_2}, m) \Downarrow \code{false} & \text{ if } |\mathtt{p_1}|_m = \code{NULL} \land \mathtt{p_2}\hookrightarrow_m l\\
%% (\mathtt{p_1 \mathbin{\code{=}} p_2}, m) \Downarrow \code{false} & \text{ if }\; \mathtt{p_1}\hookrightarrow_m l \land |\mathtt{p_2}|_m = \code{NULL} \\
(\mathtt{p_1 \mathbin{\code{=}} p_2}, m) \Downarrow \code{false} & \text{ if }~ 
\mathtt{p_1} \!=\! (l_1, i_1) \land \mathtt{p_2} \!=\! (l_2, i_2) \land l_1 \!\neq\! l_2 \\
%% \mathtt{p_1}\hookrightarrow_m l_1 \land \mathtt{p_2}\hookrightarrow_m l_2  \land l_1\neq l_2 \\
  & \quad{}~ \land \text{valid}_m(l_1,i_1) \land \text{valid}_m(l_2,i_2) \\
\end{array}
\]
%% where $\code{p} \hookrightarrow_m l$ denotes that the address in
%% \code{p} (either concrete or logical) is inside the range of the
%% block~$l$ in the memory $m$; and $\text{valid}_m(l)$ denotes that $l$
%% is a valid (\ie allocated) block in $m$. 
This definition of equality is a refinement of the pointer equality given in the ISO C18
standard~\cite[\S6.5.9p6]{c18}; for instance, it allows us to conclude that $\mathtt{p = p}$ even
when $\mathtt{p}$ is not a pointer to an allocated block, while in the C standard the result of this
comparison is undefined.


\paragraph{Hybrid and Concrete Semantics}
Using these definitions, we can give the usual operational semantic definitions to our language constructs, and perform memory operations (loads, stores, allocations, and casts) in the hybrid memory model. Static type checking allows us to split variables into pointer-typed variables (whose values are %normalized 
always logical addresses
and treated as described above) and integer-typed variables (whose values are always ordinary integers and require no special handling). 

We also give the language a purely concrete semantics and use it as a
low-level intermediate language with the concrete memory.  In this
semantics, all memory blocks are concretized and all values are just
integers, interpreted as either integer values or physical addresses
of memory cells.

%% We also give the language a purely concrete semantics, in which all
%% blocks are always concrete. The concrete semantics is derived from the
%% hybrid semantics simply by performing a realization step after
%% each instruction, in which all blocks without concrete addresses are
%% assigned arbitrary (consistent) concrete addresses. This allows the
%% same syntax to serve as both a high-level intermediate language with
%% the hybrid memory model, and a low-level intermediate language
%% in which all memory locations are concrete addresses.



\section{Fix to the LLVM and GCC miscompilation Bugs}
\label{sec:intptrcast:fix}

According to our model, the LLVM bug presented in \Cref{fig:introduction:bug} is due to the first
and the fourth optimizations:\footnote{Our proposal coincides with the consensus in the bug tracker:
  \url{https://bugs.llvm.org/show_bug.cgi?id=34548}.}
\begin{enumerate}
\item[1.] The compiler should not replace the integer comparison \code{pi != yi} at line \code{4}
  with the pointer comparison \code{\&x != y+1}, because comparing integers is always safe but
  comparing pointers with different origins invokes undefined behavior.
\item[4.] The compiler should not replace \code{(int*)pi} at line \code{13} with \code{y+1}, because
  \code{y+1} is out of range.
\end{enumerate}

On the other hand, the GCC bug presented in \Cref{fig:intptrcast-introduction:bug} is due to the too
aggressive alias analysis:
\begin{enumerate}
\item[2.] The compiler should not conclude that \code{p} is an invalid pointer.  In the presence of
  casts between pointers and integers, casting a pure integer to a pointer may results in a valid
  pointer to a concrete block.  Actually, \code{p} is alised with \code{\&x} and thus the constant
  propagation optimization should not kick in.
\end{enumerate}


%%% Local Variables:
%%% mode: latex
%%% TeX-master: "main"
%%% TeX-command-extra-options: "-shell-escape"
%%% End:
