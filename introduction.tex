\section{Introduction}
\label{sec:introduction}

\subsection{The Context: The C Programming Language}

\paragraph{A Hardware Abstraction}

The C programming language is the \emph{lingua franca} for systems programming, mainly due to its
two notable advantages: \emph{portability} and \emph{control} over hardware.  C is portable in that
C programs can be compiled and then executed in most of the existing hardware.  At the same time, C
has a precise control over hardware in that C allows programmers to access low-level hardware
features such as memory layout and concurrency.  These advantages have attracted system programmers
for decades, resulting in a giant ecosystem around the C language itself and tools such as
optimizing compilers, linkers, and program analyzers.

C enjoys portability and control---seemingly conflicting properties---at the same time thanks to the
fact that it is a balanced abstraction over various hardware assembly languages.  If C were exposing
too much detail of hardware, then it would have not been able to support some assembly languages
that mismatch with the exposed details, losing a significant degree of portability; on the other
hand, if C were exposing too little detail of hardware, then it would have lost the precise control
over them.  The design choice of C as a hardware abstraction is so popular that other systems
programming languages---such as C++, D, Objective C, Swift, and Rust---largely follow the design of
C and are often called ``C-like''.

% To summarize, the C programming language is an abstraction that should satisfy the desiderata for
% three different ``masters'': portability for programmers, control for hardware, and optimization for
% compilers.

% C---like all the other programming languages---serves multiple ``masters'', namely programmers,
% compilers, and hardware.  From programmer's point of view, C should support \emph{reasoning
%   principles} that are powerful enough to reason about real-world C programs and guarantee their
% safety and functional correctness.  On the other hand, C should validate compiler and hardware
% \emph{optimizations} that may vastly accelerate the execution of C programs and are therefore
% actually performed in the real-world compilers and hardware.  What is particularly interesting about
% C is that ...


\paragraph{Compiler Optimization}

However, C is not just a thin wrapper around assembly languages because of compiler optimizations.
They have been so crucial for the performance of systems since the early days that every system
programmer expect a compiler to perform, \eg{} register promotion and register
allocation~\cite{reg-prom, reg-alloc}, which are very effective and yet quite sophisticated compiler
optimizations.  Optimizations are becoming more and more important these days because recent trends
offer potential for them to further improve the performance of systems.  Since they are an essential
ingredient of the real-world practice of C, the language should be an abstraction not only over just
hardware assembly languages but also over compiler optimizations.

% system programmers are building bigger systems, which they cannot hand-optimize on their own; and
% hardware vendors are introducing complex features, which need a special attention for maximal
% utilization.

% These days the mainstream compilers are becoming so aggressive in these days that they are
% performing even subtle optimizations that cannot be immediately justified.

Some compiler optimizations are so subtle that they cannot be justified without an artifact.  For
example, consider the following optimization that is actually performed by the mainstream compilers
such as GCC~\cite{gcc} and LLVM~\cite{llvm}:

\[\begin{array}{rcl}
\begin{minipage}{0.27\textwidth}
\begin{minted}{c}
void f() {
  10: int x = 42;
  20: g();
  30: print(x);
}
\end{minted}
\end{minipage}
&
\optarrow
&
\begin{minipage}{0.4\textwidth}
\begin{minted}{c}
void f() {
  10: int x = 42;
  20: g();
  30: print(42); // const. prop.
}
\end{minted}
\end{minipage}
\end{array}\]

\noindent Suppose \code{g()} is an external function whose body is unknown to the compiler, and
\code{print($x$)} prints the value of $x$ to the screen.  The function \code{f()} first assigns
\code{42} to \code{x} (line \code{10}), calls \code{g()} (line \code{20}), and then prints \code{x}
(line \code{30}).  Mainstream compilers replaces \code{x} with \code{42} at line \code{30},
effectively propagating the constant \code{42} at line \code{10} to line \code{30}.  Compilers
perform such a \emph{constant propagation} optimization because they analyze that the variable
\code{x} is accessible only within the function \code{f()}, since its address is not leaked to
elsewhere, and thus \code{g()} cannot modify the content of \code{x}.

But what if an adversarial \code{g()} tries to \emph{guess} the address of \code{x} as follows?
%
\[
\begin{minipage}{0.8\textwidth}
\begin{minted}{c}
void g() {
  10: int anchor;
  20: int *guess = &anchor + 10; // guessing &x
  30: *guess = 666;
}
\end{minted}
\end{minipage}
\]
%
\noindent The function \code{g()} tries to guess the address of \code{x} by exploiting the fact that
stack usually grows downwards: it first declares a variable \code{anchor}, and guesses that \code{x}
is located 10 words later than than \code{anchor} is.  Sometimes, the guess happens to be correct
when compiled with GCC: when the adversarial \code{g()} is linked with the original \code{f()},
\code{f()} will surprisingly print the evil value 666; on the other hand, when \code{g()} is linked
with the optimized \code{f()}, \code{f()} will print the propagated value 42 as
expected.\footnote{We got this result with GCC 8.2.1 and compile option \code{-fno-stack-protector}
  in Linux 4.18.} This example invalidates the compiler's analysis that \code{x} is accessible only
within the function \code{f()}, putting the soundness of the optimization in danger.

In order to rescue the soundness of constant propagation, C blames the adversarial \code{g()} by
marking it as invoking \emph{undefined behavior}~\cite{undefined-behavior}: \code{g()} is not
following the rule of C so that compilers can do anything it chooses, even ``to make demons fly out
of your nose''~\cite{nasal-demons}.  Specifically, \code{g()} invokes undefined behavior in the ISO
C18 standard~\cite{c18} because the pointer \code{guess} is outside of the valid range of the memory
allocation of \code{anchor}, from which $\code{guess}$ is derived from, and thus \code{guess} is an
invalid address~\cite{c11-6.5.6p8}.  In other words, compilers may safely assume that all the
pointers derived from \code{anchor} \emph{shall} point to \code{anchor}, because a failure to
conform with such an assumption invokes undefined behavior.

What is particularly interesting about C is the widespread use of undefined behavior in defining
language semantics.  The constant propagation example above shows that the low-level control over
memory layout via pointer manipulation conflicts with a simple compiler optimization, and C resolves
the conflict by marking adversarial programs as invoking undefined behavior.  C typically applies
this strategy for resolving the conflicts between control over various low-level details of hardware
and various compiler optimizations implemented in mainstream C compilers, introducing a lot of
undefined behavior instances in the language semantics.  It is worth comparing C with higher-level
languages---such as Java, C\#, OCaml, Haskell---that do not need undefined behaviors to justify
compiler optimizations thanks to their lack of precise control over low-level details of hardware,
\eg{} constant propagation is immediately justified in Haskell without resorting to undefined
behavior thanks to its lack of raw pointer.


\subsection{The Problem: Reckless Development of Semantics and Compilers}

The problem is that the C language and its compilers have evolved in such an unplanned way that even
experienced system programmers disagree on the semantics of several language features and the
soundness of various compiler optimizations.  In order to improve performance and energy consumption
of systems, compiler writers have introduced dozens of subtle optimizations even though their
soundness is justified solely by intuition; in turn, to justify those optimizations afterwards, the
recent ISO C standards mark certain programs as invoking undefined behavior with a variety of ad-hoc
exceptions, making the already informal semantics more ambiguous and confusing.  In order to
mitigate the problem caused by reckless development of C semantics and compilers, ISO revises the C
semantics in a series of standards---C89, C99, C11, and C18---but they are still complex and are not
widely accepted in the systems programming community, \eg{} the Linux community defines its own
dialect of C that supports much less compiler optimizations and is closer to the assembly language
than ISO C18.

TODO: explain the problems in more details.  Too many undefined behaviors, conflicts among
optimizations.  ISO C18 is English prose.


\subsection{The Prior Art: Formal Semantics and Compiler Verification}

In order to systematically address the problems caused by the unplanned evolution of C semantics and
compilers, researchers have proposed to \emph{define the formal semantics} of C and \emph{prove the
  soundness of compiler optimizations} w.r.t. the formal semantics.  In this research agenda, we
describe the C semantics no longer in an informal English prose (as ISO C18) but in a mathematically
clear formalization, thereby completely removing the ambiguity in the semantics; furthermore, based
on the formalized semantics, we prove that compiler optimizations preserve the semantics of source
program, conclusively vindicating them from miscompilation bugs.

A landmark in this agenda is the CompCert C compiler~\cite{compcert}, which was initiated by Xavier
Leroy over ten years ago and grows as the first realistic verified compiler.  The CompCert compiler
is realistic in the sense that it ``could realistically be used in the context of production of
critical software''.  In particular, it compiles a significant subset of ISO C99 down to assembly,
and it performs a number of common and useful optimizations.  It is verified in the sense that it
``is accompanied by a machine-checked proof [in Coq] of a semantic preservation property: the
generated machine code behaves as prescribed by the semantics of the source program.''  As such,
CompCert guarantees that program analyses and verifications performed on its input carry over
soundly to its machine-level output.  It has served as a fundamental building block in academic work
on end-to-end verified software~\cite{TODO}, as well as receiving significant interest from the
avionics industry~\cite{TODO}.

In the same spirit as CompCert, Vellvm~\cite{vellvm} by Steve Zdancewic and his collaborators
formalizes a significant subset of the LLVM IR and verifies several compiler transformations and
optimizations performed at the IR level.  What is particularly interesting about Vellvm is its
formalization of static single assignment (SSA).  Vellvm formalizes a dominance analysis---which
serves as the basis for SSA form---and verifies an SSA type checker and an SSA-aware register
promotion algorithm that is simplified from the \code{mem2reg} pass in LLVM.

However, these projects make big simplifying assumptions on C semantics and compilers, skating over
the complexity of the real-world practice of C in the wild.  While CompCert and Vellvm support a
significant subset of C99 and LLVM IR, respectively, they lack support for various low-level
features that are crucially used in many system programs, such as memory layout, concurrency, and
processor register manipulation.  Furthermore, CompCert and Vellvm perform only straightforward
transformations and optimizations that are way less sophisticated than those in mainstream compilers
such as GCC and LLVM, and support only limited use cases of compilers, \eg{} they did not verify
linking.  Because of these simplifications, CompCert and Vellvm are currently suitable only for
niche use cases such as safety-critical embedded systems.


\subsection{Our Contribution: Towards Formalization of C in the Wild}

In this dissertation, pursuing the research agenda of the formalization of real-world practice of C
semantics and compilers in the wild, we aim to make an improvement upon the prior work by lifting
their simplifying assumptions in several dimensions.  Specifically, we make the following
contributions in this dissertation:

% Specifically, we propose the formal semantics of several low-level features that are the defining
% characteristics of C and yet are omitted from the prior work.  Furthermore, we propose
% verification techniques for real-world compiler use cases.

%
\begin{itemize}
\item In \Cref{chap:intptrcast}, we propose the first formal semantics of \textbf{casts between
    integers and pointers}.  While integer-pointer cast is one of the defining characteristics of
  the C programming language, the feature has not been formalized in CompCert and Vellvm because it
  drastically conflicts with major compiler optimizations.  ISO C standard tries to reconcile the
  feature and the optimizations using the notion of \emph{provenance}, but it fails to support
  certain optimizations and requires an intrusive change to the language semantics.  We propose a
  new formal semantics using the notion of \emph{concretization}, and verify the soundness of major
  memory optimizations in the proposed formal semantics.  This chapter draws heavily on the work and
  writing in the following paper:

  {\small \cite{intptrcast} \textbf{Jeehoon Kang}, Chung-Kil Hur, William Mansky, Dmitri Garbuzov,
    Steve Zdancewic, Viktor Vafeiadis.  \emph{A Formal C Memory Model Supporting Integer-Pointer
      Casts}.  \textbf{PLDI 2015}.}

\item In \Cref{chap:promising}, we propose the first formal semantics of \textbf{relaxed-memory
    concurrency} that accounts for a broad spectrum of low-level concurrency features in C.  TODO.
  This chapter draws heavily on the work and writing in the following paper:

  {\small \cite{promising} \textbf{Jeehoon Kang}, Chung-Kil Hur, Ori Lahav, Viktor Vafeiadis, Derek
    Dreyer.  \emph{A Promising Semantics for Relaxed-Memory Concurrency}.  \textbf{POPL 2017}.}

\item In \Cref{chap:sepcomp}, we propose a lightweight technique for verifying \textbf{separate
    compilation}.  TODO.  This chapter draws heavily on the work and writing in the following paper:

  {\small \cite{sepcomp} \textbf{Jeehoon Kang}, Yoonseung Kim, Chung-Kil Hur, Derek Dreyer, Viktor
    Vafeiadis.  \emph{Lightweight Verification of Separate Compilation}.  \textbf{POPL 2016}.}
\end{itemize}

TODO: As a next step, we will propose a revision to the ISO C standard and mainstream compilers
based on the formalized semantics and compiler verification techniques, thereby closing the gap
between the ISO C standard and the real-world practice of C in the wild.

TODO: ``towards'': it's not the end.  in some sense, it's just a beginning...



\paragraph*{}

Before delving into the main contributions, we first provide the technical background that informs
the rest of the dissertation in \Cref{sec:background}.  Specifically, we explain CompCert's formal
semantics of C, assembly languages, and intermediate representations (IR), and the soundness proof
of its transformation and optimization passes, based on which we develop the main body of this
dissertation.


% TODO: how to mention these papers?
%
% \begin{itemize}
% \item[\cite{scfix}] Ori Lahav, Viktor Vafeiadis, \textbf{Jeehoon Kang}, Chung-Kil Hur, Derek Dreyer.
%   \emph{Repairing Sequential Consistency in C/C++11}.  \textbf{PLDI 2017}.
% \item[\cite{crellvm}] \textbf{Jeehoon Kang}*, Yoonseung Kim*, Youngju Song*, Juneyoung Lee, Sanghoon
%   Park, Mark Dongyeon Shin, Yonghyun Kim, Sungkeun Cho, Joonwon Choi, Chung-Kil Hur, Kwangkeun Yi.
%   (*The first three authors contributed equally and are listed alphabetically.)  \emph{Crellvm:
%     Verified Credible Compilation for LLVM}.  \textbf{PLDI 2018}.
% \end{itemize}


%%% Local Variables:
%%% mode: latex
%%% TeX-master: "main"
%%% TeX-command-extra-options: "-shell-escape"
%%% End:
