\section{Introduction}
\label{sec:introduction}

\subsection{The Context: The C Programming Language}

The C programming language is the \emph{lingua franca} for systems programming, mainly because of
its \emph{portability} and \emph{control} over low-level hardware.  C is portable in that most of
the real-world systems support C API and compilers.  C has control over hardware in that its
programmers can manually instruct hardware what to do.  These advantages have attracted system
programmers for decades, resulting in a giant ecosystem around the C language itself and language
tools such as optimizing compilers, linkers, and program verifiers.  While many languages---such as
C++, D, Objective C, Swift, Rust---have been proposed to be used in building systems instead of
using C, they have never replaced C completely and even are called ``C-like'' in order to
acknowledge the influence of C on those languages.

C---like all the other programming languages---serves multiple ``masters'', namely programmers,
compilers, and hardware.  From programmer's point of view, C should support \emph{reasoning
  principles} that are powerful enough to reason about real-world C programs and guarantee their
safety and functional correctness.  On the other hand, C should validate compiler and hardware
\emph{optimizations} that may vastly accelerate the execution of C programs and are therefore
actually performed in the real-world compilers and hardware.  What is particularly interesting about
C is that TODO.


\subsection{The Gap: Masters are Conflicting with Each Other}

TODO


\subsection{Our Innovation: Formal Semantics and Formal Verification}

TODO


\subsection{Organization}

TODO


\subsection{Previously Published Material}

This thesis draws heavily on the work and writing in the following papers:

\begin{itemize}
\item[\cite{intptrcast}] \textbf{Jeehoon Kang}, Chung-Kil Hur, William Mansky, Dmitri Garbuzov,
  Steve Zdancewic, Viktor Vafeiadis.  \emph{A Formal C Memory Model Supporting Integer-Pointer
    Casts}.  \textbf{PLDI 2015}.
\item[\cite{promising}] \textbf{Jeehoon Kang}, Chung-Kil Hur, Ori Lahav, Viktor Vafeiadis, Derek
  Dreyer.  \emph{A Promising Semantics for Relaxed-Memory Concurrency}.  \textbf{POPL 2017}.
\item[\cite{sepcomp}] \textbf{Jeehoon Kang}, Yoonseung Kim, Chung-Kil Hur, Derek Dreyer, Viktor
  Vafeiadis.  \emph{Lightweight Verification of Separate Compilation}.  \textbf{POPL 2016}.
\end{itemize}

TODO

\begin{itemize}
\item[\cite{scfix}] Ori Lahav, Viktor Vafeiadis, \textbf{Jeehoon Kang}, Chung-Kil Hur, Derek Dreyer.
  \emph{Repairing Sequential Consistency in C/C++11}.  \textbf{PLDI 2017}.
\item[\cite{crellvm}] \textbf{Jeehoon Kang}*, Yoonseung Kim*, Youngju Song*, Juneyoung Lee, Sanghoon
  Park, Mark Dongyeon Shin, Yonghyun Kim, Sungkeun Cho, Joonwon Choi, Chung-Kil Hur, Kwangkeun Yi.
  (*The first three authors contributed equally and are listed alphabetically.)  \emph{Crellvm:
    Verified Credible Compilation for LLVM}.  \textbf{PLDI 2018}.
\end{itemize}


%%% Local Variables:
%%% mode: latex
%%% TeX-master: "main"
%%% End:
